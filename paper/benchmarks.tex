\section{Benchmarks}

In this section, we will look at the expected and measured performance of the
permutation loop in detail. In principle, because this is the innermost loop,
the optimizations done here should have the most effect. We focus on the inner
loop and leave the outer loop for future work as it is relatively easy to
optimize, and therefore not as interesting.

\subsection{Register usage}

Before looking at our implementations speed, it is interesting to look at the
number of registers our implementation uses. Although variations exist, most
RISC-V microarchitectures offer 31 general-purpose registers. One of these is
always needed to point to the stack, leaving us with 30 registers for our
implementation. We will still try to keep register usage to a minimum however,
so that the techniques used in this implementation may be easily ported to other
microarchitectures.

The 320-bit Ascon state needs ten 32-bit registers at all times in order to
avoid loading and storing it from memory. Our implementation of the substitution
layer uses three temporary registers during its computation. Because these are
temporary registers, they are available again after the substitution layer
finishes.

Our linear diffusion layer uses the same three temporary registers for its own
computation, however, because the substitution layer moved the state words
around, two more registers are needed to move them back into place again.

The current round constant uses another register and the final round constant
uses one more. Because the final round constant is small and constant, it can be
traded for a single-cycle instruction to generate it, but on RISC-V, there is no
need.

In total, we use 17 registers out of 30, which leaves 13 registers to implement
the outer loop. This should allow the implementation of the outer loop to avoid
loading or storing anything but the inputs and outputs of the algorithm.
Figure~\ref{registers} gives an overview of these allocations.

\begin{figure}
\caption{Register allocation during the permutation}
\begin{center}
\begin{tabular}{l c}
Purpose & Registers used \\ \hline
Ascon state & 10 \\
Temporary & 3 \\
Shuffle into place & 2 \\
Current round constant & 1 \\
Final round constant & 1 \\ \hline
Total & 17
\end{tabular}
\end{center}
\label{registers}
\end{figure}

\subsection{Expected performance}

The substitution layer was optimized by using different binary formulas to
compute the S-box. The original formulas used 22 single-cycle instructions while
the improved formulas use 17 single-cycle instructions. We have made an attempt
to prove this is the smallest number possible, but the method we used was not
sufficiently powerful.

Because we can only operate on machine words and the Ascon state words are twice
the size of a machine word, we expect the substitution layer to take 34 cycles.

As described in the section on optimization, the linear diffusion layer must
take at least 80 cycles. We give an implementation that reaches this minimum,
while also moving the state words back into their original registers, to reverse
the shuffling from the substitution layer. This is possible thanks to RISC-V's
abundant number of registers.

Finally, the addition of the round constant takes a single cycle, and its
computation from the previous round constant takes one more. This round constant
is also used to terminate the loop, by comparing it to the final round constant.
As long as this constant is loaded in a register, comparing to it and jumping
back should take on cycle. Of course, due to branch misprediction, this may take
more cycles in the first few rounds and at the end of the last round. On all
other rounds, the addition of the round constant and the loop together should
take 3 cycles. Summing these, one round of the permutation is expected to take
117 cycles.

\subsection{Method}

We take several measures in order to get highly reliable benchmarks. First of
all, we run the processor at a low frequency, approximately 18 MHz, much lower
than this processor's maximum frequency of around 300 MHz. This assures other
components, like memory and instruction cache can keep up. Second, we run each
benchmark 32 times. We report the median of these results.

Next, in order to test only the speed of rounds after the branch predictor and
other factors have stabilized, we report the difference between the timings of
running the target number of rounds and twice that number of rounds in a single
permutation.

Finally, we choose the number of permutations to be the product of several small
factors that could be the periods of various kinds of fluctuations. We use 120
rounds, which is well above realistic usage of Ascon, but the computation
remains the same, so the speed should not be affected. After taking all these
measures, we find the results are completely consistent between runs.

\subsection{Results}

Figure~\ref{roundbench} shows the measured cycles for a single round of the Ascon
permutation. While a single round was expected to take 117 cycles, in practice
it takes 118 cycles. It is unclear what causes this. The most likely explanation
is that the compare-and-jump-if-not-equal instruction used at the end of the
loop takes two cycles rather than one. Unfortunately, the documentation from
SiFive does not specify clear cycle counts for each instruction.

Figure~\ref{asconbench} shows the resulting increase in speed for encryption and
decryption across the different implementations. To encrypt 4096 bytes using
Ascon128, it is first padded to 4104 bytes and then split into 513 blocks. To
separate these blocks, 6 rounds of the permutation are run between them, 512
times in total. Initialization and finalization cost another 12 rounds each, for
a total of 3096 rounds. These rounds cost 365328 cycles, before branch
mispredictions. This means encryption spends 66\% of its time on the Ascon
permutation and decryption spends 76\% of its time on the permutation. This
difference is due to the time spent copying and padding the plaintext during
encryption, which is not needed during decryption.

\begin{figure}
\caption{Number of cycles each phase of an Ascon permutation round takes after
the branch predictor has stabilized.}
\begin{center}
\begin{tabular}{l c c}
Phase & Expected & Measured \\ \hline
Substitution layer & 34 & 34 \\
Linear diffusion layer & 80 & 80 \\
Addition of constant and loop & 3 & 4 \\ \hline
Total & 117 & 118
\end{tabular}
\end{center}
\label{roundbench}
\end{figure}

\begin{figure}
\begin{center}
\begin{tabular}{l c c c}
    Implementation & Encryption & Decryption & Relative speed \\ \hline
    Reference implementation & 701340 & 627863 & 100\% \\
    Big endian state & 612095 & 538205 & 116\% \\
    Inner loop in assembly & 552076 & 478262 & 129\% \\
\end{tabular}
\end{center}
\caption{Number of cycles for encrypting and decrypting 4096 bytes for different
implementations.}
\label{asconbench}
\end{figure}
