\documentclass{article}

\usepackage{amsmath}
\usepackage{amssymb}
\usepackage[page]{appendix}
\usepackage{caption}
\usepackage[scaled=0.8]{DejaVuSansMono}
\usepackage{listings}
\usepackage{textcomp}
\usepackage{tikz}
\usepackage{url}
\usepackage{xcolor}

\newcommand{\hex}[1]{$\mathtt{#1}_h$}
\newcommand{\xor}[0]{\oplus}
\newcommand{\cat}[0]{\parallel}
\newcommand{\ror}[0]{\ggg}

\definecolor{comment}{gray}{0.4}

\lstdefinestyle{customasm}{,
    language={[x86masm]Assembler},
    belowcaptionskip=1\baselineskip,
    basicstyle=\ttfamily,
    columns=fullflexible,
    keepspaces=true,
    upquote=true,
    numbers=none,
    numberstyle=\ttfamily\color{comment},
    keywords={.endm,.macro,.text,.globl},
    keywordstyle=\bfseries,
    morecomment=[f][\color{comment}]{\#},
}

%\renewcommand{\ttdefault}{dejavumono}

\title{Optimizing Ascon on RISC-V}
\author{Lars Jellema \\\\
    \textbf{\small Supervisors} \\
    \small Peter Schwabe \\
    \small Christoph Dobraunig}

\date{\today}

\begin{document}

\maketitle

\begin{abstract}

Ascon is a family of authenticated encryption schemes which appears in the final
portfolio of the CAESAR competition and as a candidate in the NIST lightweight
cryptography competition. It uses a cryptographic permutation as main building
block that is designed to have a straightforward and efficient implementation.

RISC-V is a hardware instruction set architecture which is completely
open-source and based RISC principles. It offers a small and simple instruction
set with little space for optimization.

We optimize the Ascon's permutation for the RISC-V RV32IMAC architecture. We
find improvements over the reference implementation in several parts and get
very close to the theoretical optimal speed. At the same time, we keep register
usage to a minimum to allow our techniques to be ported to other architectures
easily.

\end{abstract}

\clearpage

\section{Introduction}

\subsection{Symmetric Encryption}

In communication, it is often desirable to keep one's messages hidden from third
parties. This property is called \emph{confidentiality}. In more precise terms,
confidentiality means that a message is transformed in such a way that
authorized parties can recover the original message from it, while unauthorized
parties can not. In symmetric encryption, the authorized parties are defined as
the parties that know a secret number, called the \emph{key}. The original
message is called the \emph{plaintext} and the transformed message is called the
\emph{ciphertext}. The transformation from plaintext to ciphertext is called
\emph{encryption}, while the transformation back is called \emph{decryption}.
For symmetric encryption, both of these transformations need the key, which is
what prevents unauthorized parties from recovering the plaintext.

Although unauthorized parties can't recover the plaintext from the ciphertext,
they may be able to modify the ciphertext, resulting in a modified plaintext
after decryption. It is not always possible to prevent such modifications, but
it is possible to detect them. The property that it is detectable whether a
message has been changed, is called \emph{integrity}.

\subsection{Ascon}

\subsection{Related Work}

\section{Preliminaries}

\subsection{Notation}

The following table specifies the symbols and notation used in this document.

\begin{center}
\begin{tabular}{c l}
    \hex{1337} & hexadecimal number
    \\ $\bot$ & verification failure
    \\ $x \cat y$ & concatenation of bitstrings $x$ and $y$
    \\ $x \xor y$ & bitwise addition of bitstrings $x$ and $y$
    \\ $x \ror{32} n$ & 32-bit word $x$ rotated right by $n$ bits
    \\ $x \ror{64} n$ & 64-bit word $x$ rotated right by $n$ bits
\end{tabular}
\end{center}

\subsection{Symmetric encryption}

In communication, it is often desirable to keep one's messages hidden from third
parties. This property is called \emph{confidentiality}. In more precise terms,
confidentiality means that a message is transformed in such a way that
authorized parties can recover the original message from it, while unauthorized
parties can not.

In symmetric encryption, the authorized parties are defined as
the parties that know some secret information, called the \emph{key}. The
original message is called the \emph{plaintext} and the transformed message is
called the \emph{ciphertext}. The transformation from plaintext to ciphertext is
called \emph{encryption}, while the transformation back is called
\emph{decryption}. When the same secret key is needed for both encryption and
decryption, it is called \emph{symmetric encryption}. Because the key is needed
for decryption, unauthorized parties are prevented from recovering the
plaintext.

\subsubsection{Nonces}

While unauthorized parties are unable to decrypt ciphertexts, similar plaintext
may result in similar ciphertexts, and unauthorized parties may therefore detect
when similar messages are sent. In order to prevent this, during every
encryption, a different number is used to modify the resulting ciphertext. It is
needed again during decryption in order to revert that modification. Because a
different number is used every time, the resulting ciphertexts will also differ.

This number is called the \emph{nonce}, short for number used once. If it is
used more than once, unauthorized parties may be able to infer something about
the difference or similarity of the messages it was used for. Usually, this
number is made from a sufficiently large amount of random bits that the chance
of it occuring twice is negligable. When it is chosen randomly, it must be
attached to the ciphertext unencrypted, as decryption will fail without it.

If there is no space to attach the nonce to the ciphertext, it is not possible
to use a random nonce. In this case, information that is available during both
the encryption and decryption should be used to construct a unique nonce. For
example, when adding encryption support to existing filesystems, there may be no
space for nonces in the existing data structures and these probably need to stay
compatible with earlier versions. In this case, a nonce for a block can be
constructed by combining the inode and offset within the file into a nonce. With
this construction, uniqueness cannot be guaranteed, so security is reduced, but
if encryption is otherwise not possible, it is still an improvement over
unencrypted data or encryption without nonces.

\subsubsection{Authentication}

Although unauthorized parties cannot recover the plaintext from the ciphertext,
they may be able to modify the ciphertext, resulting in a modified plaintext
after decryption. It is not always possible to prevent such modifications, but
it is possible to detect them. The property that it is detectable whether a
message has been changed, is called \emph{integrity}.

Unauthorized parties may also attempt to construct messages from scratch.
Because the key is needed for encryption, they will be unable to encrypt a
specific plaintext, however, they will be able to send specific ciphertexts,
even if they don't know what plaintexts corresponds to them. In order to prevent
this, it is desirable for the receiving party to be able to verify that a
message comes from an authorized party. Together with integrity, this property
is called \emph{authenticity}. Encryption that provides both confidentiality and
authenticity to the plaintext is called \emph{authenticated encryption} or
\emph{AE}.

Authenticated encryption is usually implemented by generating an
\emph{authentication tag}. Just like the ciphertext, this tag is based on the
key, the nonce and the plaintext. This tag is needed again during decryption.
The decryption algorithm checks if the tag is correct and only returns a
plaintext if it is. If the ciphertext is modified, it will decrypt to a
different plaintext from the one used to generate the tag and cause the
decryption to fail. Because the tag also depends on the key, valid tags cannot
be created without it.

\subsubsection{Associated data}

Even when confidentiality and authenticity are assured, unauthorized parties may
still repeat a message they have seen before in a different context. To prevent
this, some data about the context in which a message is allowed to appear can be
associated with it. This is one of the uses of \emph{associated data}, which is
defined as information that requires authentication but not confidentiality.
Just like nonces, this \emph{associated data} is used during encryption to
modify the resulting ciphertext and is needed during decryption in order to
revert that modification. It does not need to be stored or sent with the
ciphertext if it can be inferred during both encryption and decryption.

Associated data can prevent a ciphertext from being decrypted in unintended
contexts. An authenticated encryption scheme that supports this is called
\emph{authenticated encryption with associated data} or \emph{AEAD}.

\subsubsection{Formal definition}

An authenticated encryption with associated data scheme is defined as a tuple of
functions $(E, D)$ with the following properties:

$E$ is a function that takes a key, a nonce, some associated data and a message
and produces a ciphertext and an authentication tag.
\begin{equation}
    E \colon K \times N \times A \times M \to C \times T
\end{equation}

$D$ is a function that takes a key, a nonce, some associated data, a ciphertext
and an authentication tag and produces either a failure or a message.
\begin{equation}
    D \colon K \times N \times A \times C \times T \to M \cup \{ \bot \}
\end{equation}

Let $k$ be a key, $n$ be a nonce, $a$ be any associated data and $m$ be a
message. The result of decryption after encryption with the same nonce, key and
associated data is the original message.
\begin{equation}
    D(k, n, a, E(k, n, a, m)) = m
\end{equation}

% TODO: No algorithm constructs valid decryption inputs different from ones seen
% before without the key.
% TODO: No algorithm decrypts without the correct key, nonce, tag, associated
% data and ciphertext.
% TODO: Different nonces imply unrelated ciphertexts
% TODO: Incorrect tag, key or associated data implies decryption failure

Let $k$ be a key, $n$ be a nonce, $a$ be any associated data and $m$ be a
message and let $c, t = E(k, n, a, m)$. Authenticity means the probability is
negligable that a decryption succeeds for a different ciphertext and the same
nonce, key, associated data and tag.

\begin{equation}
    P(D(k, n, a, c', t) \neq \bot \mid c \neq c') < \frac{1}{2^S}
\end{equation}

\subsection{Ascon internals}

Ascon~\cite{ascon} is an authenticated encryption cipher designed for use in
resource-constrained environments, like embedded devices. It has an internal
state of just 320 bits, which can be kept in registers on most architectures.
This ensures moving data between registers and memory is kept to a minimum,
which is important, as embedded devices usually do not have the same amount of
cache available as larger systems.

\subsubsection{Mode of operation}

Ascon aims to provide 128 bits of security. To that end, its key, nonce, and
authentication tag are 128 bits in size each. The plaintext, ciphertext and
associated data can all be of any length and are processed in \emph{blocks}.
There are multiple variants of Ascon, we implement two of them: Ascon-128, which
processes 64-bit blocks and Ascon-128a, which processes 128-bit blocks.

Ascon uses a 320-bit state and a permutation that mixes the state in a way that
is hard to reverse. This permutation consists of a transformation that is
applied in multiple rounds, each with a different round constant. Both variants
of Ascon use permutation $p^a$, which consists of 12 rounds, during
initialization and finalization. After processing each block, Ascon uses
permutation $p^b$, which consists of 6 rounds in Ascon-128 and 8 rounds in
Ascon-128a. Figure~\ref{encdec} gives an overview of encryption and decryption.

%%%%%%%%%%%%%%%%%%%%%%%%%%%%%%%%%%%%%%%%%%%%%%%%%%%%%%%%%%%%%%%%%%%%%%%%%%%%%%%%%%
% The Ascon encryption mode
%
% public domain (CC0 1.0 https://creativecommons.org/publicdomain/zero/1.0/)
%%%%%%%%%%%%%%%%%%%%%%%%%%%%%%%%%%%%%%%%%%%%%%%%%%%%%%%%%%%%%%%%%%%%%%%%%%%%%%%%%%

\newif\ifsans
\newif\iftext
\newif\ifdetails

%%% CONFIGURATION %%%%%%%%%%%%%%%%%%%%%%%%%%%%%%%%%%%%%%%%%%%%%%%%%%%%%%%%%%%%%%%%
% preferably use build.py

%\sanstrue  % for sans-serif fonts (slides, web)
\sansfalse  % for serif fonts (article)

%\texttrue  % include phase description
\textfalse  % no phase description

\detailsfalse % simplify (eg, 'const' instead of 'kab0'
%%%%%%%%%%%%%%%%%%%%%%%%%%%%%%%%%%%%%%%%%%%%%%%%%%%%%%%%%%%%%%%%%%%%%%%%%%%%%%%%%%

\ifsans
\renewcommand*\familydefault{\sfdefault}
\usepackage{sfmath}
\fi

\tikzset{sparsam/.style={inner sep=1pt}}
\tikzset{bitwidth/.style={above=-1pt, font=\tiny}}
\tikzset{next/.style={->, >=latex}}

\begin{tikzpicture}[scale=0.75, every node/.style={scale=0.75}]
  \newcommand{\conc}{\ensuremath{\Vert}}
  \newcommand{\perm}[1]{node[rectangle, rounded corners=3pt, minimum width=.5cm, minimum height=1.8cm, draw, sparsam] {$p^{#1}$}}
  \ifsans
    \newcommand{\tikzxor}[1]{node[circle, inner sep=-1.3pt, name={#1}] {\tikz{\draw[] (0,0) circle (3.75pt) +(3.75pt,0) -- +(-3.75pt,0) +(0,3.75pt) -- +(0,-3.75pt);}}}
  \else
    \newcommand{\tikzxor}[1]{node[circle, inner sep=-1.3pt, name={#1}] {$\oplus$}}
  \fi
  \newcommand{\bitwidth}{\tikz{\draw[-] (-2pt,-2pt) -- (2pt, 2pt);}}
  \newcommand{\rate}{.5cm}
  \newcommand{\msg}{.6cm}
  \newcommand{\phase}{1.7cm}
  \newcommand{\minnext}{.4cm}

  \clip(-3,-2) rectangle (6,2);

  % --- init up to p^a ---
  \begin{scope}[xshift=0cm]
    \ifdetails
      \draw (0,\rate) node[left, sparsam] {$k \conc r \conc a \conc b \conc 0^*$};
      \draw (0,-\rate) node[left, sparsam] {$K \conc N$};

      \draw[next] (0,\rate) -- node {\bitwidth} node[bitwidth] {$r$} (.7,\rate);
      \draw[next] (0,-\rate) -- node {\bitwidth} node[bitwidth] {$c$} (.7,-\rate);
    \else
      \draw (0,0) node[left, sparsam] {$\mathrm{IV} \conc K \conc N$};
      \draw[next] (0,0) -- node {\bitwidth} node[bitwidth] {$320$} (.7,0);
    \fi

    \draw (.95,0) \perm{a};
  \end{scope}

  % --- init after p^a and auth A1 ---
  \begin{scope}[xshift=1.2cm]
    \draw (.4,-\rate) \tikzxor{Ki};
    \draw[next] (0,-\rate) -- (Ki);
    \draw[next] (Ki) +(0,-\msg) node[below] {$0^* \conc K$ \hspace*{.3cm}} -- (Ki);
    \draw[next] (Ki) -- node[pos=0.6] {\bitwidth} node [pos=0.6, bitwidth] {$c$} +(1.3,0);

    \draw[dashdotted] (.8,1.5) -- (.8,-1.5);

    \draw (1.3,\rate) \tikzxor{A1};
    \draw[next] (0,\rate) -- node[near start] {\bitwidth} node[near start, bitwidth] {$r$} (A1);
    \draw[next] (A1) +(0,\msg) node[above] {$A_1$} -- (A1);
    \draw[next] (A1) -- +(\minnext,0);

    \draw (1.95,0) \perm{b};
  \end{scope}

  % --- auth As ---
  \begin{scope}[xshift=3.4cm]
    \draw[dotted] (0,\rate) -- (\minnext,\rate)
                  (0,-\rate) -- (\minnext,-\rate);

    \draw (.9,\rate) \tikzxor{As};
    \draw[next] (\minnext,\rate) -- (As);
    \draw[next] (As) +(0,\msg) node[above] {$A_s$} -- (As);
    \draw[next] (As) -- +(\minnext,0);

    \draw[next] (\minnext,-\rate) -- node {\bitwidth} node[bitwidth] {$c$} (1.3,-\rate);

    \draw (1.55,0) \perm{b};
  \end{scope}

  % --- enc P1 ---
  \begin{scope}[xshift=5.2cm]
    \draw (.4,-\rate) \tikzxor{AuthPad};
    \draw[next] (0,-\rate) -- (AuthPad);
    \draw[next] (AuthPad) +(0,-\msg) node[below] {$0^* \conc 1$ \hspace*{.3cm}} -- (AuthPad);
    \draw[next] (AuthPad) -- node {\bitwidth} node [bitwidth] {$c$} +(1.7,0);

    \draw[dashdotted] (.8,1.5) -- (.8,-1.5);

    \draw (1.3,\rate) \tikzxor{P1};
    \draw[next] (0,\rate) -- node[near start] {\bitwidth} node[near start, bitwidth] {$r$} (P1);
    \draw[next] (P1) +(0,\msg) node[above] {$P_1$} -- (P1);
    \draw[next] (P1) -- +(2*\minnext,0);
    \draw[next] (P1) ++(.5,0) -- +(0,\msg) node[above] {$C_1$};

    \draw (2.35,0) \perm{b};
  \end{scope}

  %% --- enc Pt-1 ---
  %\begin{scope}[xshift=7.8cm]
  %  \draw[dotted] (0,\rate) -- (\minnext,\rate)
  %                (0,-\rate) -- (\minnext,-\rate);

  %  \draw[next] (\minnext,-\rate) -- node[pos=.4] {\bitwidth} node [pos=.4,bitwidth] {$c$} +(1.3,0);

  %  \draw (.9,\rate) \tikzxor{Pt1};
  %  \draw[next] (\minnext,\rate) -- (Pt1);
  %  \draw[next] (Pt1) +(0,\msg) node[above] {$P_{t\!-\!1}$ \hspace*{.15cm}} -- (Pt1);
  %  \draw[next] (Pt1) -- +(.8,0);
  %  \draw[next] (Pt1) ++(.5,0) -- +(0,\msg) node[above] {\hspace*{.2cm} $C_{t\!-\!1}$};

  %  \draw (1.95,0) \perm{b};
  %\end{scope}

  %% --- enc Pt and finalize ---
  %\begin{scope}[xshift=10.0cm]
  %  \draw (.5,\rate) \tikzxor{Pt};
  %  \draw[next] (0,\rate) -- (Pt);
  %  \draw[next] (Pt) +(0,\msg) node[above] {$P_t$} -- (Pt);
  %  \draw[next] (Pt) ++(.5,0) -- +(0,\msg) node[above] {$C_t$};
  %  \draw[next] (Pt) -- node[pos=.7] {\bitwidth} node[pos=.7,bitwidth] {$r$} +(1.6,0);

  %  \draw[dashdotted] (1.3,1.5) -- (1.3,-1.5);

  %  \draw (1.7,-\rate) \tikzxor{Kf};
  %  \draw[next] (Kf) +(0,-\msg) node[below] {\hspace*{.3cm} $K \conc 0^*$} -- (Kf);
  %  \draw[next] (0,-\rate) -- node[pos=.3] {\bitwidth} node[pos=.3,bitwidth] {$c$} (Kf);
  %  \draw[next] (Kf) -- (2.1,-\rate);

  %  \draw (2.35,0) \perm{a};

  %  \draw (3.3,-\rate) \tikzxor{Kt};
  %  \draw[next] (Kt) +(0,-\msg) node[below] {$K$} -- (Kt);
  %  \draw[next] (2.6,-\rate) -- node[pos=.4] {\bitwidth} node[pos=.4,bitwidth] {$k$} (Kt);
  %  \draw (4.0,-\rate) node[name=T,sparsam] {$T$};
  %  \draw[next] (Kt) -- (T);
  %\end{scope}

  % --- phase descriptions ---
  \iftext
    \draw (.5,-\rate-\phase) node {Initialization};
    \draw (4.0,-\rate-\phase) node {Associated Data};
    \draw (8.5,-\rate-\phase) node {Plaintext};
    \draw (12.8,-\rate-\phase) node {Finalization};
  \fi
\end{tikzpicture}

%\begin{tikzpicture}[scale=0.6, every node/.style={scale=0.6}]
%  \newcommand{\conc}{\ensuremath{\Vert}}
%  \newcommand{\perm}[1]{node[rectangle, rounded corners=3pt, minimum width=.5cm, minimum height=1.8cm, draw, sparsam] {$p^{#1}$}}
%  \ifsans
%    \newcommand{\tikzxor}[1]{node[circle, inner sep=-1.3pt, name={#1}] {\tikz{\draw[] (0,0) circle (3.75pt) +(3.75pt,0) -- +(-3.75pt,0) +(0,3.75pt) -- +(0,-3.75pt);}}}
%  \else
%    \newcommand{\tikzxor}[1]{node[circle, inner sep=-1.3pt, name={#1}] {$\oplus$}}
%  \fi
%  \newcommand{\bitwidth}{\tikz{\draw[-] (-2pt,-2pt) -- (2pt, 2pt);}}
%  \newcommand{\rate}{.5cm}
%  \newcommand{\msg}{.6cm}
%  \newcommand{\phase}{1.7cm}
%  \newcommand{\minnext}{.4cm}
%
%  % --- init up to p^a ---
%  \begin{scope}[xshift=0cm]
%    \ifdetails
%      \draw (0,\rate) node[left, sparsam] {$k \conc r \conc a \conc b \conc 0^*$};
%      \draw (0,-\rate) node[left, sparsam] {$K \conc N$};
%
%      \draw[next] (0,\rate) -- node {\bitwidth} node[bitwidth] {$r$} (.7,\rate);
%      \draw[next] (0,-\rate) -- node {\bitwidth} node[bitwidth] {$c$} (.7,-\rate);
%    \else
%      \draw (0,0) node[left, sparsam] {$\mathrm{IV} \conc K \conc N$};
%      \draw[next] (0,0) -- node {\bitwidth} node[bitwidth] {$320$} (.7,0);
%    \fi
%
%    \draw (.95,0) \perm{a};
%  \end{scope}
%
%  % --- init after p^a and auth A1 ---
%  \begin{scope}[xshift=1.2cm]
%    \draw (.4,-\rate) \tikzxor{Ki};
%    \draw[next] (0,-\rate) -- (Ki);
%    \draw[next] (Ki) +(0,-\msg) node[below] {$0^* \conc K$ \hspace*{.3cm}} -- (Ki);
%    \draw[next] (Ki) -- node[pos=0.6] {\bitwidth} node [pos=0.6, bitwidth] {$c$} +(1.3,0);
%
%    \draw[dashdotted] (.8,1.5) -- (.8,-1.5);
%
%    \draw (1.3,\rate) \tikzxor{A1};
%    \draw[next] (0,\rate) -- node[near start] {\bitwidth} node[near start, bitwidth] {$r$} (A1);
%    \draw[next] (A1) +(0,\msg) node[above] {$A_1$} -- (A1);
%    \draw[next] (A1) -- +(\minnext,0);
%
%    \draw (1.95,0) \perm{b};
%  \end{scope}
%
%  % --- auth As ---
%  \begin{scope}[xshift=3.4cm]
%    \draw[dotted] (0,\rate) -- (\minnext,\rate)
%                  (0,-\rate) -- (\minnext,-\rate);
%
%    \draw (.9,\rate) \tikzxor{As};
%    \draw[next] (\minnext,\rate) -- (As);
%    \draw[next] (As) +(0,\msg) node[above] {$A_s$} -- (As);
%    \draw[next] (As) -- +(\minnext,0);
%
%    \draw[next] (\minnext,-\rate) -- node {\bitwidth} node[bitwidth] {$c$} (1.3,-\rate);
%
%    \draw (1.55,0) \perm{b};
%  \end{scope}
%
%  % --- enc P1 ---
%  \begin{scope}[xshift=5.2cm]
%    \draw (.4,-\rate) \tikzxor{AuthPad};
%    \draw[next] (0,-\rate) -- (AuthPad);
%    \draw[next] (AuthPad) +(0,-\msg) node[below] {$0^* \conc 1$ \hspace*{.3cm}} -- (AuthPad);
%    \draw[next] (AuthPad) -- node {\bitwidth} node [bitwidth] {$c$} +(1.7,0);
%
%    \draw[dashdotted] (.8,1.5) -- (.8,-1.5);
%
%    \draw (1.3,\rate) \tikzxor{P1};
%    \draw[next] (0,\rate) -- node[near start] {\bitwidth} node[near start, bitwidth] {$r$} (P1);
%    \draw[next] (P1) -- +(0,\msg) node[above] {$P_1$};
%    \draw[<->,>=latex] (P1) -- +(2*\minnext,0);
%    \draw[next] (P1) ++(.5,\msg) node[above] {$C_1$} -- +(0,-\msg);
%
%    \draw (2.35,0) \perm{b};
%  \end{scope}
%
%  % --- enc Pt-1 ---
%  \begin{scope}[xshift=7.8cm]
%    \draw[dotted] (0,\rate) -- (\minnext,\rate)
%                  (0,-\rate) -- (\minnext,-\rate);
%
%    \draw[next] (\minnext,-\rate) -- node[pos=.4] {\bitwidth} node [pos=.4,bitwidth] {$c$} +(1.3,0);
%
%    \draw (.9,\rate) \tikzxor{Pt1};
%    \draw[next] (\minnext,\rate) -- (Pt1);
%    \draw[next] (Pt1) -- +(0,\msg) node[above] {$P_{t\!-\!1}$ \hspace*{.15cm}};
%    \draw[<->,>=latex] (Pt1) -- +(.8,0);
%    \draw[next] (Pt1) ++(.5,\msg) node[above] {\hspace*{.2cm} $C_{t\!-\!1}$} -- +(0,-\msg);
%
%    \draw (1.95,0) \perm{b};
%  \end{scope}
%
%  % --- enc Pt and finalize ---
%  \begin{scope}[xshift=10.0cm]
%    \draw (.5,\rate) \tikzxor{Pt};
%    \draw[next] (0,\rate) -- (Pt);
%    \draw[next] (Pt) -- +(0,\msg) node[above] {$P_t$};
%    \draw[next] (Pt) ++(.5,\msg) node[above] {$C_t$} -- +(0,-\msg);
%    \draw[<->,>=latex] (Pt) -- node[pos=.7] {\bitwidth} node[pos=.7,bitwidth] {$r$} +(1.6,0);
%
%    \draw[dashdotted] (1.3,1.5) -- (1.3,-1.5);
%
%    \draw (1.7,-\rate) \tikzxor{Kf};
%    \draw[next] (Kf) +(0,-\msg) node[below] {\hspace*{.3cm} $K \conc 0^*$} -- (Kf);
%    \draw[next] (0,-\rate) -- node[pos=.3] {\bitwidth} node[pos=.3,bitwidth] {$c$} (Kf);
%    \draw[next] (Kf) -- (2.1,-\rate);
%
%    \draw (2.35,0) \perm{a};
%
%    \draw (3.3,-\rate) \tikzxor{Kt};
%    \draw[next] (Kt) +(0,-\msg) node[below] {$K$} -- (Kt);
%    \draw[next] (2.6,-\rate) -- node[pos=.4] {\bitwidth} node[pos=.4,bitwidth] {$k$} (Kt);
%    \draw (4.0,-\rate) node[name=T,sparsam] {$T$};
%    \draw[next] (Kt) -- (T);
%  \end{scope}
%
%  % --- phase descriptions ---
%  \iftext
%    \draw (.5,-\rate-\phase) node {Initialization};
%    \draw (4.0,-\rate-\phase) node {Associated Data};
%    \draw (8.5,-\rate-\phase) node {Plaintext};
%    \draw (12.8,-\rate-\phase) node {Finalization};
%  \fi
%\end{tikzpicture}


Encryption starts with the initialization of the state. The state is initialized
to the concatenation of an initialization vector, the key and the nonce. The
initialization vector encodes the parameters of the variant of Ascon that it is
used for. This initial state is then passed through the first permutation,
$p^b$. Finally, the key is bitwise added to the least significant bits of the
state again.

After initialization, the associated data is mixed into the state. If there is
no associated data, the state is left unmodified. Otherwise, the associated data
is padded and split into blocks. Each of the blocks is bitwise added to the most
significant bits of the state, followed by applying the second permutation,
$p^b$. At the end, the least significant bit is inverted. This is done to
separate the associated data from the plaintext, which comes next.

The plaintext is padded and split into blocks in the same way as the associated
data. Each block of plaintext is bitwise added to the most significant bits of
the state. The result of this bitwise addition also forms the corresponding
block of the ciphertext. After adding each block, except for the last,
permutation $p^b$ is applied again.

Finalization begins with bitwise adding the key to the most significant bits of
the state that were not used to add the plaintext. After that, the permutation
$p^a$ is applied again and finally, the key is bitwise added to the least
significant bits of the state again. The result of this last operation is used
as the authentication tag.

Decryption is identical to encryption, except the ciphertext is processed in a
slightly different way from the plaintext: Each block of the ciphertext is again
bitwise added to the most significant bits of the state to form the plaintext,
but instead of using this result in the state, the most significant bits of the
state are replaced with the ciphertext before the bitwise addition.

\subsubsection{Permutation}

Ascon's main component is a permutation consisting of three phases, which are
applied in several rounds. Ascon-128 uses 6 rounds to process blocks of 64 bits
at a time. Ascon-128a uses 8 rounds to process blocks of 128 bits at a time.
Both use keys of 128 bits and 12 rounds to initialize and finalize the state. A
round consists of the following three phases: The addition of the round
constant, the substitution layer and the linear diffusion layer. Each of these
phases modify the internal state in a different way. The internal state consists
of 320 bits, logically split into five 64-bit words called $x_0$, $x_1$, $x_2$,
$x_3$ and $x_4$.

The round constant is a single byte that changes from round to round and is
added bitwise to the least significant 8 bits of $x_2$. It ensures rounds are
not all identical. The final round constant is always \hex{4b}. The round
constant changes linearly, decreasing by \hex{f} every round. This means
the round constant can be computed easily based on the number of rounds that are
left.

The substitution layer applies a 5-bit lookup function in parallel to the five
state words. It provides non-linear mixing and mixing between the five state
words. It is usually implemented as a binary formula applied bitwise to five
machine words at a time.

The linear diffusion layer provides linear mixing between the bits within each
state word. It consists of a bitwise addition of the original state word with
the same state word rotated by two different amounts. Each of the five state
words uses different rotation amounts.

\subsection{RISC-V}
RISC-V~\cite{riscv} is an open-source hardware instruction set architecture.

% TODO: introduce add means bitwise exclusive or
% TODO: List all used instructions
% TODO: Describe simplicity / RISCness which makes formal proofs easier

\section{Optimization}

\subsection{Permutation}

Ascon keeps an internal state of 320 bits. On 64-bit platforms, this is usually
kept in 5 registers, on 32-bit platforms, it needs to be kept in 10.

The core of Ascon consists of a transformation applied in several rounds. Each
round consists of three phases: The addition of the round constant, the
substitution layer and the linear diffusion layer.

\subsection{Substitution layer}

The substitution layer in Ascon is a 5-bit S-box which is applied in parallel to
64 sets of 5 bits. The paper describes an instruction sequence of bitwise
operations that will compute this S-box on an entire machine word in parallel.
It takes 5 input words, one for each bit in the S-box, and produces 5 output
words. The instruction sequence translates to 22 single-cycle instructions.
Because these can only be applied to 32 bits at a time on 32-bit platforms, they
need to be run twice. Therefor, a straightforward implementation takes 44
cycles. The instruction sequence is as follows:

\begin{samepage}
\begin{verbatim}
x0 ^=  x4;  x4 ^=  x3;  x2 ^=  x1;
t0  = ~x0;  t1  = ~x1;  t2  = ~x2;  t3  = ~x3;  t4  = ~x4;
t0 &=  x1;  t1 &=  x2;  t2 &=  x3;  t3 &=  x4;  t4 &=  x0;
x0 ^=  t1;  x1 ^=  t2;  x2 ^=  t3;  x3 ^=  t4;  x4 ^=  t0;
x1 ^=  x0;  x0 ^=  x4;  x3 ^=  x2;  x2  = ~x2;
\end{verbatim}
\end{samepage}

The substitution layer can be optimized by computing the same S-box with a
different formula. I derived shorter binary formulas by first writing down the
the bit sequences that occur for each of the 5 output bits for all 32 inputs.
This allowed us to recognize patterns in the bit sequences. I first eliminated
input bits that did not affect the output, or only affected the output through a
single exclusive or operation. Then, I looked at the remaining bit patterns and
found short and overlapping binary formulas for them. Here, $o_n$ indicates
output bit $n$ and $i_n$ indicates input bit $n$.

\begin{samepage}
\begin{align*}
   o_0 & = i_3 \oplus i_4 \oplus (i_1 \vee (i_0 \oplus i_2 \oplus i_4))
\\ o_1 & = i_0 \oplus i_4 \oplus ((i_2 \oplus i_1) \vee (i_3 \oplus i_2))
\\ o_2 & = i_1 \oplus i_2 \oplus (i_3 \vee \neg i_4)
\\ o_3 & = i_1 \oplus i_2 \oplus (i_0 \vee (i_4 \oplus i_3))
\\ o_4 & = i_3 \oplus i_4 \oplus (i_1 \wedge \neg (i_4 \oplus i_0))
\end{align*}
\end{samepage}

From these, an instruction sequence can be produced like the following:

\begin{samepage}
\begin{verbatim}
t12  =  x1 ^ x2; t04  = x0 ^ x4; t34  = x3 ^ x4;
x4   = ~x4;      x4  |= x3;      x4  ^= t12;
x3  ^=  x1;      x3  |= t12;     x3  ^= t04;
x2  ^=  t04;     x2  |= x1;      x2  ^= t34;
t04  = ~t04;     x1  &= t04;     x1  ^= t34;
                 x0  |= t34;     x0  ^= t12;
\end{verbatim}
\end{samepage}

This instruction sequence computes all five of the above formulas, but the
results end up in different registers. It is possible to compensate for this in
the linear diffusion layer.

\subsubsection{Superoptimization}

As these formulas were constructed by hand, it may be possible to do better.
There are many possible formulas, so it is infeasable to find the best option by
hand. One option is to use something like the GNU Superoptimizer which tries all
possible instruction sequence of a certain length in order to see if any of them
computes a specific formula. Unfortunately, according to its README, the longest
instruction sequence it was able to find for anything was seven instructions
long. This is not enough, since it's expected that at least five XOR operations
and at least five AND operations need to be computed, leading to a minimum of 10
instructions.

Ko Stoffelen\cite{sat} attempted to optimize the binary formulas using a SAT
solver. He found ways to compile an S-box to a satisfiability problem
determining whether it can be computed in a given number of instructions. I made
use of his project to prove that it is not possible to compute the Ascon S-box
in 10 instructions. Unfortunately, the project did not finish within reasonable
time for instruction counts larger than 10, so I was unable to verify whether 17
instructions is the best number possible.

\subsection{Linear diffusion layer}

The linear diffusion layer in Ascon mixes the bits with each of the five state
words. Each of them is rotated by two different amounts and added with the
results. For each of the five state words, two different rotation amounts are
used.

The state words are 64 bits in size, so 64-bit rotations are used. On most
64-bit architectures, this can be done with just one instruction, but 32-bit
architectures generally don't have 64-bit rotate instructions.

The most straightforward method of simulating a 64-bit rotate instruction is
using shifts. By combining a left shift and a right shift, all bits of a single
machine word can be placed at the correct offset. Because each shift throws
away some of the bits, it is not possible to simulate a rotate instruction with
one shift. Because a state word is two machine words in size, four shifts are
needed in total. After this, there are four intermediate results that need to be
combined to two resulting machine words. There are several options to combine
these intermediate results:

Since each bit will be zero in at least one of the two operands, the combining
instructions can be bitwise or, bitwise exclusive or, or integer addition
instructions. Since the results will be combined using bitwise exclusive or
operations, we will use these for combining intermediate shift results as well.
Because bitwise exclusive or operations are associative, it doesn't matter in
what order all intermediate results are combined, which gives us more freedom
during implementation. Bitwise exclusive or can also be seen as bitwise
addition, or addition without any carries between the bits. I will use
\emph{bitwise addition} to refer to this operation from here on.

Using shifts means throwing away bits, which is why four of them are needed. It
would be preferable to use rotate instructions, since they don't throw away
bits. Unfortunately, rotating both halves of a state word independently does not
have the same effect as rotating the word as a whole.

This is usually resolved using bit interleaving: By storing even-numbered bits
in one word and odd-numbered bits in another, a 64-bit rotation can be simulated
using two 32-bit rotations. For even rotation amounts, both words are rotated by
half the amount. For odd rotation amounts, the words are rotated by half the
amount, rounded up and down respectively, and then swapped. The disadvantage is
that some extra cycles are needed to convert between bit-interleaved
representation and normal representation.

Unfortunately, the microcontroller we're targeting does not have rotate
instructions at all. The SiFive HiFive1 does not include the bit manipulation
extension, which is where RISC-V introduces its rotate instructions. This means
we're stuck using the straightforward method using four shifts and two bitwise
additions. The result also needs to be bitwise added back into the original,
which takes another two instructions, one for each of the machine words. This
means that handling one rotation amount takes eight instructions in total.

There are five state words which each have two rotations applied to them. This
means a total of 80 instructions are needed for the linear diffusion layer. On
32-bit RISC-V, without the bit manipulation extension, it is not possible to do
better. This is easy to see:

The only method of moving bits to different offsets is the shift instruction,
and because it throws away bits, there are always two needed to handle 32 bits,
or one machine word. Therefor, to move all bits into two separate new positions
for five state words, or ten machine word, a total of 40 shifts are needed. Each
of these shifts produces an intermediate result and all those intermediate
results need to be merged back into the unrotated state words.

Each merging instruction takes two operands and produces one result, reducing
the number of intermediate results by one, so a total of 40 merging instructions
is needed to merge all intermediate results. On the HiFive1, there are no
instructions that can merge more than two machine words at a time. This means it
is not possible to do the linear diffusion layer in less than 80 instructions.
Figure~\ref{xorror} shows how the rotations for a single state word are
implemented in RISC-V assembly.

\begin{figure}
\begin{verbatim}
.macro xorror dl, dh, sl, sh, sl0, sh0, r0, sl1, sh1, r1, t0, t1, t2
        slli \t0, \sl0, (32 - \r0)
        srli \t2, \sh0, \r0
        xor \t0, \t0, \t2
        slli \t2, \sl1, (32 - \r1)
        xor \t0, \t0, \t2
        srli \t2, \sh1, \r1
        xor \t0, \t0, \t2
        slli \t1, \sh0, (32 - \r0)
        srli \t2, \sl0, \r0
        xor \t1, \t1, \t2
        slli \t2, \sh1, (32 - \r1)
        xor \t1, \t1, \t2
        srli \t2, \sl1, \r1
        xor \t1, \t1, \t2
        xor \dl, \sl, \t1
        xor \dh, \sh, \t0
.endm
\end{verbatim}

\caption{The \texttt{xorror} macro applies the rotates and bitwise additions of
the linear diffusion layer to a single state word. Because of the limitations of
assembly macros, it takes many arguments:
\texttt{dl} and \texttt{dh} are the registers for storing the low and high part
of the result. \texttt{sl}, \texttt{sh}, \texttt{sl0}, \texttt{sh0},
\texttt{sl1} and \texttt{sh1} are the source registers for the state word,
without rotation, with rotation \texttt{r0} and with rotation \texttt{r1}
respectively. Because the shift amounts can only be 32 at most, representing
rotate amounts above 64 is done by subtracting 32 and swapping the respective
source registers. This is why the source registers must be supplied thrice.
\texttt{t0}, \texttt{t1} and \texttt{t2} are three temporary registers used to
store intermediate results.}
\label{xorror}
\end{figure}

\section{Benchmarks}

In this section, we will look at the expected and measured performance of the
permutation loop in detail. In principle, because this is the innermost loop,
the optimizations done here should have the most effect. We focus on the inner
loop and leave the outer loop for future work as it is relatively easy to
optimize, and therefore not as interesting.

\subsection{Register usage}

Before looking at our implementations speed, it is interesting to look at the
number of registers our implementation uses. Although variations exist, most
RISC-V microarchitectures offer 31 general-purpose registers. One of these is
always needed to point to the stack, leaving us with 30 registers for our
implementation. We will still try to keep register usage to a minimum however,
so that the techniques used in this implementation may be easily ported to other
microarchitectures.

The 320-bit Ascon state needs ten 32-bit registers at all times in order to
avoid loading and storing it from memory. Our implementation of the substitution
layer uses three temporary registers during its computation. Because these are
temporary registers, they are available again after the substitution layer
finishes.

Our linear diffusion layer uses the same three temporary registers for its own
computation, however, because the substitution layer moved the state words
around, two more registers are needed to move them back into place again.

The current round constant uses another register and the final round constant
uses one more. Because the final round constant is small and constant, it can be
traded for a single-cycle instruction to generate it, but on RISC-V, there is no
need.

In total, we use 17 registers out of 30, which leaves 13 registers to implement
the outer loop. This should allow the implementation of the outer loop to avoid
loading or storing anything but the inputs and outputs of the algorithm.
Figure~\ref{registers} gives an overview of these allocations.

\begin{figure}
\caption{Register allocation during the permutation}
\begin{center}
\begin{tabular}{l c}
Purpose & Registers used \\ \hline
Ascon state & 10 \\
Temporary & 3 \\
Shuffle into place & 2 \\
Current round constant & 1 \\
Final round constant & 1 \\ \hline
Total & 17
\end{tabular}
\end{center}
\label{registers}
\end{figure}

\subsection{Expected performance}

The substitution layer was optimized by using different binary formulas to
compute the S-box. The original formulas used 22 single-cycle instructions while
the improved formulas use 17 single-cycle instructions. We have made an attempt
to prove this is the smallest number possible, but the method we used was not
sufficiently powerful.

Because we can only operate on machine words and the Ascon state words are twice
the size of a machine word, we expect the substitution layer to take 34 cycles.

As described in the section on optimization, the linear diffusion layer must
take at least 80 cycles. We give an implementation that reaches this minimum,
while also moving the state words back into their original registers, to reverse
the shuffling from the substitution layer. This is possible thanks to RISC-V's
abundant number of registers.

Finally, the addition of the round constant takes a single cycle, and its
computation from the previous round constant takes one more. This round constant
is also used to terminate the loop, by comparing it to the final round constant.
As long as this constant is loaded in a register, comparing to it and jumping
back should take on cycle. Of course, due to branch misprediction, this may take
more cycles in the first few rounds and at the end of the last round. On all
other rounds, the addition of the round constant and the loop together should
take 3 cycles. Summing these, one round of the permutation is expected to take
117 cycles.

\subsection{Method}

We take several measures in order to get highly reliable benchmarks. First of
all, we run the processor at a low frequency, approximately 18 MHz, much lower
than this processor's maximum frequency of around 300 MHz. This assures other
components, like memory and instruction cache can keep up. Second, we run each
benchmark 32 times. We report the median of these results.

Next, in order to test only the speed of rounds after the branch predictor and
other factors have stabilized, we report the difference between the timings of
running the target number of rounds and twice that number of rounds in a single
permutation.

Finally, we choose the number of permutations to be the product of several small
factors that could be the periods of various kinds of fluctuations. We use 120
rounds, which is well above realistic usage of Ascon, but the computation
remains the same, so the speed should not be affected. After taking all these
measures, we find the results are completely consistent between runs.

\subsection{Results}

Figure~\ref{roundbench} shows the measured cycles for a single round of the Ascon
permutation. While a single round was expected to take 117 cycles, in practice
it takes 118 cycles. It is unclear what causes this. The most likely explanation
is that the compare-and-jump-if-not-equal instruction used at the end of the
loop takes two cycles rather than one. Unfortunately, the documentation from
SiFive does not specify clear cycle counts for each instruction.

Figure~\ref{asconbench} shows the resulting increase in speed for encryption and
decryption across the different implementations. To encrypt 4096 bytes using
Ascon128, it is first padded to 4104 bytes and then split into 513 blocks. To
separate these blocks, 6 rounds of the permutation are run between them, 512
times in total. Initialization and finalization cost another 12 rounds each, for
a total of 3096 rounds. These rounds cost 365328 cycles, before branch
mispredictions. This means encryption spends 66\% of its time on the Ascon
permutation and decryption spends 76\% of its time on the permutation. This
difference is due to the time spent copying and padding the plaintext during
encryption, which is not needed during decryption.

\begin{figure}
\caption{Number of cycles each phase of an Ascon permutation round takes after
the branch predictor has stabilized.}
\begin{center}
\begin{tabular}{l c c}
Phase & Expected & Measured \\ \hline
Substitution layer & 34 & 34 \\
Linear diffusion layer & 80 & 80 \\
Addition of constant and loop & 3 & 4 \\ \hline
Total & 117 & 118
\end{tabular}
\end{center}
\label{roundbench}
\end{figure}

\begin{figure}
\begin{center}
\begin{tabular}{l c c c}
    Implementation & Encryption & Decryption & Relative speed \\ \hline
    Reference implementation & 701340 & 627863 & 100\% \\
    Big endian state & 612095 & 538205 & 116\% \\
    Inner loop in assembly & 552076 & 478262 & 129\% \\
\end{tabular}
\end{center}
\caption{Number of cycles for encrypting and decrypting 4096 bytes for different
implementations.}
\label{asconbench}
\end{figure}

\section{Discussion}

We have optimized the Ascon internal loop for the RISC-V RV32IMAC architecture.
While doing this, we have noted that both Ascon and RISC-V are very simple and
have little room for optimization. Nonetheless, we have found improvements in
the boolean formulas used to compute the substitution layer and an oppertunity
to combine the round constant with the loop counter. We have also switched the
representation of the state to big endian in order to reduce the number of
endian conversions necessary. Our improvements have provided an increase in
speed of 29\% relative to the reference implementation.

There is still room for more improvement however. We have omitted to optimize
the Ascon outer loop. Because RISC-V offers plenty of registers, writing the
outer loop in assembly makes it possible to avoid loading and storing the state
to memory altogether. It is also possible that the boolean formulas of the
substitution layer are not yet optimal. Finding and proving the optimal formulas
will require some careful analysis however, as the search-space is quite large.

Finally, it is important to note that our optimizations are quite general and
should be portable to other architectures. Our new boolean formulas do not only
improve speed from 22 to 17 cycles, but also register usage from 5 to 3
temporary registers. This makes it plausible to implement the substitution layer
without needing the stack on 32-bit ARM architectures which often only have 14
registers available for implementation. It is also of interest to improve our
implementation using the RISC-V bit manipulation extension, once it is frozen
and more widely available.


\bibliographystyle{ieeetr}
\bibliography{references}

\begin{appendices}

\section{Reproducability}

\subsection{Reproducability using Nix}

Nix~\cite{nix} is a functional package manager designed to make its packages
highly reproducable. As a package manager, it cannot eliminate entropy from
concurrency, and some other sources, so multiple builds of the same package
are not necessarily byte-for-byte identical. Instead, it tries to eliminate
entropy by specifying almost all direct and less direct dependencies with high
precision.

Packages are specified by derivations, which specify a shell script that
builds the package, any relevant environment variables, and a list of
sources and package dependencies the build needs. These derivations are
cryptographically hashed, and the resulting hash uniquely identifies the
package. Whenever the version or configuration of a package or one of
its dependencies changes, the resulting hash will be different, which
means it will be a different package.

In order to make our results reproducable, we use Nix to specify
precisely what compiler was used to compile the binaries for the board.
We do this by creating a pseudo-package, which specifies dependencies,
but no build instructions. The \texttt{nix-shell} command was designed
to use this: It takes a package and starts a bash session with all
dependencies injected into the PATH, in the same way it would when
building the package.


\section{Code listing}

\subsection{\texttt{permutation.s}}

\lstinputlisting[style=customasm]{../riscv/permutation.s}

\end{appendices}

\end{document}
