\documentclass{article}

\usepackage{amsmath}
\usepackage{amssymb}
\usepackage[page]{appendix}
\usepackage{caption}
\usepackage[scaled=0.8]{DejaVuSansMono}
\usepackage{listings}
\usepackage{textcomp}
\usepackage{tikz}
\usepackage{url}
\usepackage{xcolor}

\newcommand{\hex}[1]{$\mathtt{#1}_h$}
\newcommand{\xor}[0]{\oplus}
\newcommand{\cat}[0]{\parallel}
\newcommand{\ror}[1]{\ggg_{#1}}

\definecolor{comment}{gray}{0.4}

\lstdefinestyle{customasm}{,
    language={[x86masm]Assembler},
    belowcaptionskip=1\baselineskip,
    basicstyle=\ttfamily,
    columns=fullflexible,
    keepspaces=true,
    upquote=true,
    numbers=left,
    numberstyle=\ttfamily,
    keywords={.endm,.macro,.text,.globl},
    keywordstyle=\bfseries,
    morecomment=[f][\color{comment}]{\#},
    commentstyle=\itshape,
}

%\renewcommand{\ttdefault}{dejavumono}

\title{Optimizing Ascon on RISC-V}
\author{Lars Jellema \\\\
    \textbf{\small Supervisors} \\
    \small Peter Schwabe \\
    \small Christoph Dobraunig}

\date{\today}

\begin{document}

\maketitle

\begin{abstract}
Lorem ipsum

\dots while keeping register usage to a minimum, so our techniques can be ported
easily to microarchitectures with fewer registers.

\end{abstract}

\clearpage

\section{Introduction}

In recent years, many critical vulnerabilities have been found in various kinds
of software and hardware. The software industry continues to grow and the value
of hacking grows with it. Making software secure is hard and many companies
consider it a waste of money. This has lead to botnets, among other things.
Botnets usually consist of a large number of cheap internet-connected consumer
devices that are easily hacked. Because the profit margins are smaller for cheap
devices, there is often less money available for implementing good security.

One way to make good security cheaper is by providing security primitives with
simple interfaces that nonetheless ensure all desirable properties. An example
of this is \emph{authenticated encryption} systems, which provide all of the
most important security properties for symmetric encryption. By using
authenticated encryption, a developer does not need to introduce authentication
to an encryption system manually, leading to fewer possibilities to make
mistakes.

Authenticated encryption has traditionally been implemented by combining
privacy-only encryption with message authentication codes~\cite{aeadorder}. This
combination turned out to be hard to make without introducing new
vulnerabilities. In addition, performance suffers because two passes need to be
made over the plaintext, one for encryption and one for authentication. To
resolve this, cryptographers have called for new primitives that integrate all
desirable properties.

This call has taken the shape of several competitions, where a diverse set of
teams each publish a family of new primitives, followed by public analysis of
these new primitives by many other researchers. New primitives are then selected
based on how well they withstood public analysis.

Ascon is one of these new primitives, aimed at lightweight applications. It was
first submitted to the CAESAR competition~\cite{caesar} and has been selected as
the first choice for lightweight applications in the final portfolio. It was
also recently submitted to be considered for standardization in NIST Lightweight
Cryptography~\cite{nistlc} and has been selected as a round~1 candidate.

While secure software is important, the hardware it runs on must also be secure.
The integrated circuit industry has tradionally been a closed ecosystem. Chip
manufacturers keep their implementations secret in order to monopolize the
market. This means that when a vulnerability is discovered in a chip, consumers
are entirely dependent on the manufacturer to fix them. It also makes it hard
for third parties to verify a chips behavior.

RISC-V is a hardware instruction set architecture that intends to solve this
issue by being completely open-source. This means anyone is free to use,
implement, extend and adapt it. As a result, RISC-V has been gaining popularity
with researchers as well as companies. RISC-V is an architecture based on RISC
principles, which makes it a good fit for low-power applications.

So far, there has been little work towards optimizing crypto algorithms for
RISC-V in software. Most efforts have been focused on creating hardware
extensions for RISC-V as this was not possible with closed-ecosystem
architectures. Despite the fact that RISC-V extensions do not have licensing
costs and have reduced development costs, chip manufacturing remains expensive,
so fast software implementations remain important. We optimize the Ascon
authenticated encryption system for the RISC-V instruction set architecture.

\section{Preliminaries}

\subsection{Symmetric encryption}

In communication, it is often desirable to keep one's messages hidden from third
parties. This property is called \emph{confidentiality}. In more precise terms,
confidentiality means that a message is transformed in such a way that
authorized parties can recover the original message from it, while unauthorized
parties can not.

In symmetric encryption, the authorized parties are defined as
the parties that know some secret information, called the \emph{key}. The
original message is called the \emph{plaintext} and the transformed message is
called the \emph{ciphertext}. The transformation from plaintext to ciphertext is
called \emph{encryption}, while the transformation back is called
\emph{decryption}. When the same secret key is needed for both encryption and
decryption, it is called \emph{symmetric encryption}. Because the key is needed
for decryption, unauthorized parties are prevented from recovering the
plaintext.

\subsubsection{Nonces}

While unauthorized parties are unable to decrypt ciphertexts, similar plaintext
may result in similar ciphertexts, and unauthorized parties may therefor detect
when similar messages are sent. In order to prevent this, during every
encryption, a different number is used to modify the resulting ciphertext. It is
needed again during decryption in order to revert that modification. Because a
different number is used every time, the resulting ciphertexts will also differ.

This number is called the \emph{nonce}, short for number used once. If it is
used more than once, unauthorized parties may be able to infer something about
the difference or similarity of the messages it was used for. Usually, this
number is made from a sufficiently large amount of random bits that the chance
of it occuring twice is negligable. When it is chosen randomly, it must be
attached to the ciphertext unencrypted, as decryption will fail without it. If
there is no space to attach the nonce to the ciphertext, it is not possible to
use a random nonce. In this case, information that is available during both the
encryption and decryption should be used to construct a unique nonce.

\subsubsection{Authentication}

Although unauthorized parties can't recover the plaintext from the ciphertext,
they may be able to modify the ciphertext, resulting in a modified plaintext
after decryption. It is not always possible to prevent such modifications, but
it is possible to detect them. The property that it is detectable whether a
message has been changed, is called \emph{integrity}.

Unauthorized parties may also attempt to construct messages from scratch.
Because the key is needed for encryption, they will be unable to encrypt a
specific plaintext, however, they will be able to send specific ciphertexts,
even if they don't know what plaintexts corresponds to them. In order to prevent
this, it is desirable for the receiving party to be able to verify that a
message comes from an authorized party. Together with integrity, this property
is called \emph{authenticity}. Encryption that provides both confidentiality and
authenticity to the plaintext is called \emph{authenticated encryption}, or
\emph{AE}.

\subsubsection{Associated data}

Even when confidentiality and authenticity are assured, unauthorized parties may
still repeat a message they've seen before in a different context. To prevent
this, some data about the context in which a message is allowed to appear can be
associated with it. This is one of the uses of \emph{associated data}, which is
defined as information that requires authentication but not confidentiality.
Just like nonces, this \emph{associated data} is used during encryption to
modify the resulting ciphertext and is needed during decryption in order to
revert that modification. It does not need to be stored or sent with the
ciphertext if it can be inferred during both encryption and decryption.

Associated data can prevent a ciphertext from being decrypted in unintended
contexts. An authenticated encryption scheme that supports this is called
\emph{authenticated encryption with associated data}, or \emph{AEAD}.

\subsubsection{Formal definition}

An authenticated encryption with associated data scheme is defined as a tuple of
functions $(E, D)$ with the following properties:

$E$ is a function that takes a nonce, a key, some associated data and a message
and produces a ciphertext and an authentication tag.
\begin{equation}
    E \colon N \times K \times A \times M \to C \times T
\end{equation}

$D$ is a function that takes a nonce, a key, some associated data, a ciphertext
and an authentication tag and produces either a failure or a message.
\begin{equation}
    D \colon N \times K \times A \times C \times T \to M \cup \{ \bot \}
\end{equation}

Let $n$ be a nonce, $k$ be a key, $a$ be any associated data and $m$ be a
message. The result of decryption after encryption with the same nonce, key and
associated data is the original message.
\begin{equation}
    D(n, k, a, E(n, k, a, m)) = m
\end{equation}

% TODO: No algorithm constructs valid decryption inputs different from ones seen
% before without the key.
% TODO: No algorithm decrypts without the correct key, nonce, tag, associated
% data and ciphertext.
% TODO: Different nonces imply unrelated ciphertexts
% TODO: Incorrect tag, key or associated data implies decryption failure

Let $n$ be a nonce, $k$ be a key, $a$ be any associated data and $m$ be a
message and let $c, t = E(n, k, a, m)$. Authenticity means the probability is
negligable that a decryption succeeds for a different ciphertext and the same
nonce, key, associated data and tag.

\begin{equation}
    P(D(n, k, a, c', t) \neq \bot \mid c \neq c') < \frac{1}{2^S}
\end{equation}

\subsection{Ascon internals}

\subsubsection{Mode}

% TODO: Describe algorithm independent of mixing function.

\subsubsection{Permutation}

Ascon's main component is a permutation consisting of three phases, which are
applied in several rounds. Ascon-128 uses 6 rounds to process blocks of 64 bits
at a time. Ascon-128a uses 8 rounds to process blocks of 128 bits at a time.
Both use keys of 128 bits and 12 rounds to initialize and finalize the state. A
round consists of the following three phases: The addition of the round
constant, the substitution layer and the linear diffusion layer. Each of these
phases modify the internal state in a different way. The internal state consists
of 320 bits, logically split into five 64-bit words called $x_0$, $x_1$, $x_2$,
$x_3$ and $x_4$.

The round constant is a single byte that changes from round to round and is
added bitwise to the least significant 8 bits of $x_2$. It ensures rounds are
not all identical. The final round constant is always \hex{4b}. The round
constant changes linearly, decreasing by \hex{f} every round. This means
the round constant can be computed easily based on the number of rounds that are
left.

The substitution layer applies a 5-bit lookup function in parallel to the five
state words. It provides non-linear mixing and mixing between the five state
words. It is usually implemented as a binary formula applied bitwise to five
machine words at a time.

The linear diffusion layer provides linear mixing between the bits within each
state word. It consists of a bitwise addition of the original state word with
the same state word rotated by two different amounts. Each of the five state
words uses different rotation amounts.

\subsection{RISC-V}
RISC-V\cite{riscv} is an open-source hardware instruction set architecture.

% TODO: introduce add means bitwise exclusive or
% TODO: introduce machine word vs state word
% TODO: List all used instructions
% TODO: Describe simplicity / RISCness which makes formal proofs easier

\section{Optimization}

\subsection{Permutation}

Ascon keeps an internal state of 320 bits. On 64-bit platforms, this is usually
kept in 5 registers, on 32-bit platforms, it needs to be kept in 10.

The core of Ascon consists of a transformation applied in several rounds. Each
round consists of three phases: The addition of the round constant, the
substitution layer and the linear diffusion layer.

\subsection{Substitution layer}

The substitution layer in Ascon is a 5-bit S-box which is applied in parallel to
64 sets of 5 bits. The paper describes an instruction sequence of bitwise
operations that will compute this S-box on an entire machine word in parallel.
It takes 5 input words, one for each bit in the S-box, and produces 5 output
words. The instruction sequence translates to 22 single-cycle instructions.
Because these can only be applied to 32 bits at a time on 32-bit platforms, they
need to be run twice. Therefor, a straightforward implementation takes 44
cycles. The instruction sequence is as follows:

\begin{samepage}
\begin{verbatim}
x0 ^=  x4;  x4 ^=  x3;  x2 ^=  x1;
t0  = ~x0;  t1  = ~x1;  t2  = ~x2;  t3  = ~x3;  t4  = ~x4;
t0 &=  x1;  t1 &=  x2;  t2 &=  x3;  t3 &=  x4;  t4 &=  x0;
x0 ^=  t1;  x1 ^=  t2;  x2 ^=  t3;  x3 ^=  t4;  x4 ^=  t0;
x1 ^=  x0;  x0 ^=  x4;  x3 ^=  x2;  x2  = ~x2;
\end{verbatim}
\end{samepage}

The substitution layer can be optimized by computing the same S-box with a
different formula. I derived shorter binary formulas by first writing down the
the bit sequences that occur for each of the 5 output bits for all 32 inputs.
This allowed us to recognize patterns in the bit sequences. I first eliminated
input bits that did not affect the output, or only affected the output through a
single exclusive or operation. Then, I looked at the remaining bit patterns and
found short and overlapping binary formulas for them. Here, $o_n$ indicates
output bit $n$ and $i_n$ indicates input bit $n$.

\begin{samepage}
\begin{align*}
   o_0 & = i_3 \oplus i_4 \oplus (i_1 \vee (i_0 \oplus i_2 \oplus i_4))
\\ o_1 & = i_0 \oplus i_4 \oplus ((i_2 \oplus i_1) \vee (i_3 \oplus i_2))
\\ o_2 & = i_1 \oplus i_2 \oplus (i_3 \vee \neg i_4)
\\ o_3 & = i_1 \oplus i_2 \oplus (i_0 \vee (i_4 \oplus i_3))
\\ o_4 & = i_3 \oplus i_4 \oplus (i_1 \wedge \neg (i_4 \oplus i_0))
\end{align*}
\end{samepage}

From these, an instruction sequence can be produced like the following:

\begin{samepage}
\begin{verbatim}
t12  =  x1 ^ x2; t04  = x0 ^ x4; t34  = x3 ^ x4;
x4   = ~x4;      x4  |= x3;      x4  ^= t12;
x3  ^=  x1;      x3  |= t12;     x3  ^= t04;
x2  ^=  t04;     x2  |= x1;      x2  ^= t34;
t04  = ~t04;     x1  &= t04;     x1  ^= t34;
                 x0  |= t34;     x0  ^= t12;
\end{verbatim}
\end{samepage}

This instruction sequence computes all five of the above formulas, but the
results end up in different registers. It is possible to compensate for this in
the linear diffusion layer.

\subsubsection{Superoptimization}

As these formulas were constructed by hand, it may be possible to do better.
There are many possible formulas, so it is infeasable to find the best option by
hand. One option is to use something like the GNU Superoptimizer which tries all
possible instruction sequence of a certain length in order to see if any of them
computes a specific formula. Unfortunately, according to its README, the longest
instruction sequence it was able to find for anything was seven instructions
long. This is not enough, since it's expected that at least five XOR operations
and at least five AND operations need to be computed, leading to a minimum of 10
instructions.

Ko Stoffelen\cite{sat} attempted to optimize the binary formulas using a SAT
solver. He found ways to compile an S-box to a satisfiability problem
determining whether it can be computed in a given number of instructions. I made
use of his project to prove that it is not possible to compute the Ascon S-box
in 10 instructions. Unfortunately, the project did not finish within reasonable
time for instruction counts larger than 10, so I was unable to verify whether 17
instructions is the best number possible.

\section{Benchmarks}

In this section, we will look at the expected and measured performance of the
permutation loop in detail. In principle, because this is the innermost loop,
the optimizations done here should have the most effect. We focus on the inner
loop and leave the outer loop for future work as it is relatively easy to
optimize, and therefor not as interesting.

\subsection{Register usage}

Before looking at our implementations speed, it is interesting to look at the
number of registers our implementation uses. Although variations exist, most
RISC-V microarchitectures offer 31 general-purpose registers. One of these is
always needed to point to the stack, leaving us with 30 registers for our
implementation. We will still try to keep register usage to a minimum however,
so that the techniques used in this implementation may be easily ported to other
microarchitectures.

The 320-bit Ascon state needs ten 32-bit registers at all times in order to
avoid loading and storing it from memory. Our implementation of the substitution
layer uses three temporary registers during its computation. Because these are
temporary registers, they are available again after the substitution layer
finishes.

Our linear diffusion layer uses the same three temporary registers for its own
computation, however, because the substitution layer moved the state words
around, two more registers are needed to move them back into place again.

The current round constant uses another register and the final round constant
uses one more. Because the final round constant is small and constant, it can be
traded for a single-cycle instruction to generate it, but on RISC-V, there is no
need.

In total, we use 17 registers out of 30, which leaves 13 registers to implement
the outer loop. This should allow the implementation of the outer loop to avoid
loading or storing anything but the inputs and outputs of the algorithm.
Figure~\ref{registers} gives an overview of these allocations.

\begin{figure}
\begin{center}
\begin{tabular}{l c}
Purpose & Registers used \\ \hline
Ascon state & 10 \\
Temporary & 3 \\
Shuffle into place & 2 \\
Current round constant & 1 \\
Final round constant & 1 \\ \hline
Total & 17
\end{tabular}
\end{center}
\caption{Register allocation during the }
\label{registers}
\end{figure}

\subsection{Expected performance}

The substitution layer was optimized by using different binary formulas to
compute the S-box. The original formulas used 22 single-cycle instructions while
the improved formulas use 17 single-cycle instructions. We have made an attempt
to prove this is the smallest number possible, but the method we used was not
sufficiently powerful.

Because we can only operate on machine words and the Ascon state words are twice
the size of a machine word, so we expect the substitution layer to take 34
cycles.

As described in the section on optimization, the linear diffusion layer must
take at least 80 cycles. We give an implementation that reaches this minimum,
while also moving the state words back into their original registers, to reverse
the shuffling from the substitution layer. This is possible thanks to RISC-V's
abundant number of registers.

Finally, the addition of the round constant takes a single cycle, and its
computation from the previous round constant takes one more. This round constant
is also used to terminate the loop, by comparing it to the final round constant.
As long as this constant is loaded in a register, comparing to it and jumping
back should take on cycle. Of course, due to branch misprediction, this may take
more cycles in the first few rounds and at the end of the last round. On all
other rounds, the addition of the round constant and the loop together should
take 3 cycles. Summing these, one round of the permutation is expected to take
117 cycles.

\subsection{Method}

We take several measures in order to get highly reliable benchmarks. First of
all, we run the processor at a low frequency, approximately 18 MHz, much lower
than this processor's maximum frequency of around 300 MHz. This assures other
components, like memory and instruction cache can keep up. Second, we run each
benchmark 32 times. We report the median of these results.

Next, in order to test only the speed of rounds after the branch predictor and
other factors have stabilized, we report the difference between the timings of
running the target number of rounds and twice that number of rounds in a single
permutation.

Finally, we choose the number of permutations to be the product of several small
factors that could be the periods of various kinds of fluctuations. We use 120
rounds, which is well above realistic usage of Ascon, but the computation
remains the same, so the speed should not be affected. After taking all these
measures, we find the results are completely consistent between runs.

\subsection{Results}

Figure~\ref{roundbench} shows the measured cycles for a single round of the Ascon
permutation. While a single round was expected to take 117 cycles, in practice
it takes 118 cycles. It is unclear what causes this. The most likely explanation
is that the compare-and-jump-if-not-equal instruction used at the end of the
loop takes two cycles rather than one. Unfortunately, the documentation from
SiFive does not specify clear cycle counts for each instruction.

Figure~\ref{asconbench} shows the resulting increase in speed for encryption and
decryption across the different implementations. To encrypt 4096 bytes using
Ascon128, it is first padded to 5004 bytes and then split into 513 blocks. To
separate these blocks, 6 rounds of the permutation are run between them, 512
times in total. Initialization and finalization cost another 12 rounds each, for
a total of 3096 rounds. These rounds cost 365328 cycles, before branch
mispredictions. This means encryption spends 66\% of its time on the Ascon
permutation and decryption spends 76\% of its time on the permutation.

\begin{figure}
\begin{center}
\begin{tabular}{l c c}
Phase & Expected & Measured \\ \hline
Substitution layer & 34 & 34 \\
Linear diffusion layer & 80 & 80 \\
Addition of constant and loop & 3 & 4 \\ \hline
Total & 117 & 118
\end{tabular}
\end{center}
\caption{Number of cycles each phase of an Ascon permutation round takes after
the branch predictor has stabilized.}
\label{roundbench}
\end{figure}

\begin{figure}
\begin{center}
\begin{tabular}{l c c c}
    Implementation & Encryption & Decryption & Relative speed \\ \hline
    Reference implementation & 701340 & 627863 & 100\% \\
    Big endian state & 612095 & 538205 & 116\% \\
    Inner loop in assembly & 552076 & 478262 & 129\% \\
\end{tabular}
\end{center}
\caption{Number of cycles for encrypting and decrypting 4096 bytes for different
implementations.}
\label{asconbench}
\end{figure}

\section{Discussion}

% TODO: 129% improvement over C due to simple algo and arch

\subsection{Future work}

% TODO: Optimize outer loop
% TODO: Optimize for RISC-V with bit manipulation extension
% TODO: Prove the substitution formulas are optimal or find better ones.
% TODO: Port techniques to other microarchitectures.


\bibliographystyle{ieeetr}
\bibliography{references}

\begin{appendices}

\section{Reproducability}

\subsection{Reproducability using Nix}

Nix~\cite{nix} is a functional package manager designed to make its packages
highly reproducable. As a package manager, it cannot eliminate entropy from
concurrency, and some other sources, so multiple builds of the same package
are not necessarily byte-for-byte identical. Instead, it tries to eliminate
entropy by specifying almost all direct and less direct dependencies with high
precision.

Packages are specified by derivations, which specify a shell script that
builds the package, any relevant environment variables, and a list of
sources and package dependencies the build needs. These derivations are
cryptographically hashed, and the resulting hash uniquely identifies the
package. Whenever the version or configuration of a package or one of
its dependencies changes, the resulting hash will be different, which
means it will be a different package.

In order to make our results reproducable, we use Nix to specify
precisely what compiler was used to compile the binaries for the board.
We do this by creating a pseudo-package, which specifies dependencies,
but no build instructions. The \texttt{nix-shell} command was designed
to use this: It takes a package and starts a bash session with all
dependencies injected into the PATH, in the same way it would when
building the package.


\section{Code listing}

\subsection{\texttt{permutation.s}}

\lstinputlisting[style=customasm]{../riscv/permutation.s}

\end{appendices}

\end{document}
