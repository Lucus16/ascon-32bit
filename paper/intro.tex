\section{Introduction}

\subsection{Symmetric Encryption}

In communication, it is often desirable to keep one's messages hidden from third
parties. This property is called \emph{confidentiality}. In more precise terms,
confidentiality means that a message is transformed in such a way that
authorized parties can recover the original message from it, while unauthorized
parties can not. In symmetric encryption, the authorized parties are defined as
the parties that know a secret number, called the \emph{key}. The original
message is called the \emph{plaintext} and the transformed message is called the
\emph{ciphertext}. The transformation from plaintext to ciphertext is called
\emph{encryption}, while the transformation back is called \emph{decryption}.
For symmetric encryption, both of these transformations need the key, which is
what prevents unauthorized parties from recovering the plaintext.

Although unauthorized parties can't recover the plaintext from the ciphertext,
they may be able to modify the ciphertext, resulting in a modified plaintext
after decryption. It is not always possible to prevent such modifications, but
it is possible to detect them. The property that it is detectable whether a
message has been changed, is called \emph{integrity}.

\subsection{Ascon}

\subsection{Related Work}
