\section{Introduction}

\subsection{Symmetric encryption}

In communication, it is often desirable to keep one's messages hidden from third
parties. This property is called \emph{confidentiality}. In more precise terms,
confidentiality means that a message is transformed in such a way that
authorized parties can recover the original message from it, while unauthorized
parties can not.

In symmetric encryption, the authorized parties are defined as
the parties that know a secret number, called the \emph{key}. The original
message is called the \emph{plaintext} and the transformed message is called the
\emph{ciphertext}. The transformation from plaintext to ciphertext is called
\emph{encryption}, while the transformation back is called \emph{decryption}.
When the same secret key is needed for both encryption and decryption, it is
called \emph{symmetric encryption}. Because the key is needed for decryption,
unauthorized parties are prevented from recovering the plaintext.

\subsubsection{Nonces}

While unauthorized parties are unable to decrypt ciphertexts, similar plaintext
may result in similar ciphertexts, and unauthorized parties may therefor detect
when similar messages are sent. In order to prevent this, during every
encryption, a different number is used to modify the resulting ciphertext. It is
needed again during decryption in order to revert that modification. Because a
different number is used every time, the resulting ciphertexts will also differ.

This number is called the \emph{nonce}, short for number used once. If it is
used more than once, unauthorized parties may be able to infer something about
the difference or similarity of the messages it was used for. Usually, this
number is made from a sufficiently large amount of random bits that the chance
of it occuring twice is negligable. When it is chosen randomly, it must be
attached to the ciphertext unencrypted, as decryption will fail without it.

\subsubsection{Authentication}

Although unauthorized parties can't recover the plaintext from the ciphertext,
they may be able to modify the ciphertext, resulting in a modified plaintext
after decryption. It is not always possible to prevent such modifications, but
it is possible to detect them. The property that it is detectable whether a
message has been changed, is called \emph{integrity}.

Unauthorized parties may also attempt to construct messages from scratch.
Because the key is needed for encryption, they will be unable to encrypt a
specific plaintext, however, they will be able to send specific ciphertexts,
even if they don't know what plaintexts corresponds to them. In order to prevent
this, it is desirable for the receiving party to be able to verify that a
message comes from an authorized party. Together with integrity, this property
is called \emph{authenticity}. Encryption that provides both confidentiality and
authenticity to the plaintext is called \emph{authenticated encryption}, or AE.

\subsubsection{Associated data}

Even when confidentiality and authenticity are assured, unauthorized parties may
still repeat a message they've seen before in a different context. To prevent
this, some data about the context in which a message is allowed to appear is
associated with it. Just like nonces, this \emph{associated data} is used during
encryption to modify the resulting ciphertext and is needed during decryption in
order to revert that modification. It does not need to be stored or sent with the
ciphertext if it can be inferred during both encryption and decryption.

Associated data prevents a ciphertext from being decrypted in unintended
contexts. An authenticated encryption scheme that supports this is called
\emph{authenticated encryption with associated data}, or AEAD.

\subsubsection{Formal definition}

In summary, an authenticated encryption with associated data scheme is defined
as a tuple of functions $(G, E, D)$, with the following properties:

\begin{enumerate}
    \item $G \colon R \to K$ is a function that takes a stream of random data and
        produces a key.
    \item $E \colon R \times K \times A \times P \to C$ is a function that takes
        a stream of random data, a key, some associated data and a plaintext and
        produces a ciphertext.
    \item $D \colon K \times A \times C \to P \cup \{ \bot \}$ is a function
        that takes a key, some associated data and a ciphertext and produces a
        plaintext or a failure.
    \item $\forall r_k, r_e \in R\ldotp \forall a \in A\ldotp
        \forall p \in P\ldotp D(G(r_k), a, E(r_e, G(r_k), a, p)) = p$, that is,
        if a key is generated and used to encrypt a message with some associated
        data and the ciphertext is passed with the same key and associated data
        to the decryption function, it should yield the same plaintext.
    \item never the same twice \ldots
    \item integrity \ldots
    \item authenticity \ldots
    \item confidentiality \ldots
    \item wrong context \ldots
\end{enumerate}

\subsection{Ascon}

Ascon\cite{ascon} is an authenticated encryption cipher designed for use in
resource constrained environments, like embedded devices. It has an internal
state of just 320 bits, which can be kept in registers on most
microarchitectures. This ensures moving data between registers and memory is
kept to a minimum, which is important, as embedded devices usually don't have
the same level of caching available as larger systems.

\subsection{Related Work}
