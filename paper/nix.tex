\section{Reproducability}

\subsection{Reproducability using Nix}

Nix~\cite{nix} is a functional package manager designed to make its packages
highly reproducable. As a package manager, it cannot eliminate entropy from
concurrency, and some other sources, so multiple builds of the same package
are not necessarily byte-for-byte identical. Instead, it tries to eliminate
entropy by specifying almost all direct and less direct dependencies with high
precision.

Packages are specified by derivations, which specify a shell script that
builds the package, any relevant environment variables, and a list of
sources and package dependencies the build needs. These derivations are
cryptographically hashed, and the resulting hash uniquely identifies the
package. Whenever the version or configuration of a package or one of
its dependencies changes, the resulting hash will be different, which
means it will be a different package.

In order to make our results reproducable, we use Nix to specify
precisely what compiler was used to compile the binaries for the board.
We do this by creating a pseudo-package, which specifies dependencies,
but no build instructions. The \texttt{nix-shell} command was designed
to use this: It takes a package and starts a bash session with all
dependencies injected into the PATH, in the same way it would when
building the package.
