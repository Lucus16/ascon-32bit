\section{Preliminaries}

\subsection{Symmetric encryption}

In communication, it is often desirable to keep one's messages hidden from third
parties. This property is called \emph{confidentiality}. In more precise terms,
confidentiality means that a message is transformed in such a way that
authorized parties can recover the original message from it, while unauthorized
parties can not.

In symmetric encryption, the authorized parties are defined as
the parties that know some secret information, called the \emph{key}. The
original message is called the \emph{plaintext} and the transformed message is
called the \emph{ciphertext}. The transformation from plaintext to ciphertext is
called \emph{encryption}, while the transformation back is called
\emph{decryption}. When the same secret key is needed for both encryption and
decryption, it is called \emph{symmetric encryption}. Because the key is needed
for decryption, unauthorized parties are prevented from recovering the
plaintext.

\subsubsection{Nonces}

While unauthorized parties are unable to decrypt ciphertexts, similar plaintext
may result in similar ciphertexts, and unauthorized parties may therefor detect
when similar messages are sent. In order to prevent this, during every
encryption, a different number is used to modify the resulting ciphertext. It is
needed again during decryption in order to revert that modification. Because a
different number is used every time, the resulting ciphertexts will also differ.

This number is called the \emph{nonce}, short for number used once. If it is
used more than once, unauthorized parties may be able to infer something about
the difference or similarity of the messages it was used for. Usually, this
number is made from a sufficiently large amount of random bits that the chance
of it occuring twice is negligable. When it is chosen randomly, it must be
attached to the ciphertext unencrypted, as decryption will fail without it. If
there is no space to attach the nonce to the ciphertext, it is not possible to
use a random nonce. In this case, information that is available during both the
encryption and decryption should be used to construct a unique nonce.

\subsubsection{Authentication}

Although unauthorized parties can't recover the plaintext from the ciphertext,
they may be able to modify the ciphertext, resulting in a modified plaintext
after decryption. It is not always possible to prevent such modifications, but
it is possible to detect them. The property that it is detectable whether a
message has been changed, is called \emph{integrity}.

Unauthorized parties may also attempt to construct messages from scratch.
Because the key is needed for encryption, they will be unable to encrypt a
specific plaintext, however, they will be able to send specific ciphertexts,
even if they don't know what plaintexts corresponds to them. In order to prevent
this, it is desirable for the receiving party to be able to verify that a
message comes from an authorized party. Together with integrity, this property
is called \emph{authenticity}. Encryption that provides both confidentiality and
authenticity to the plaintext is called \emph{authenticated encryption}, or
\emph{AE}.

\subsubsection{Associated data}

Even when confidentiality and authenticity are assured, unauthorized parties may
still repeat a message they've seen before in a different context. To prevent
this, some data about the context in which a message is allowed to appear can be
associated with it. This is one of the uses of \emph{associated data}, which is
defined as information that requires authentication but not confidentiality.
Just like nonces, this \emph{associated data} is used during encryption to
modify the resulting ciphertext and is needed during decryption in order to
revert that modification. It does not need to be stored or sent with the
ciphertext if it can be inferred during both encryption and decryption.

Associated data can prevent a ciphertext from being decrypted in unintended
contexts. An authenticated encryption scheme that supports this is called
\emph{authenticated encryption with associated data}, or \emph{AEAD}.

\subsubsection{Formal definition}

An authenticated encryption with associated data scheme is defined as a tuple of
functions $(E, D)$ with the following properties:

$E$ is a function that takes a nonce, a key, some associated data and a message
and produces a ciphertext and an authentication tag.
\begin{equation}
    E \colon N \times K \times A \times M \to C \times T
\end{equation}

$D$ is a function that takes a nonce, a key, some associated data, a ciphertext
and an authentication tag and produces either a failure or a message.
\begin{equation}
    D \colon N \times K \times A \times C \times T \to M \cup \{ \bot \}
\end{equation}

Let $n$ be a nonce, $k$ be a key, $a$ be any associated data and $m$ be a
message. The result of decryption after encryption with the same nonce, key and
associated data is the original message.
\begin{equation}
    D(n, k, a, E(n, k, a, m)) = m
\end{equation}

% TODO: No algorithm constructs valid decryption inputs different from ones seen
% before without the key.
% TODO: No algorithm decrypts without the correct key, nonce, tag, associated
% data and ciphertext.
% TODO: Different nonces imply unrelated ciphertexts
% TODO: Incorrect tag, key or associated data implies decryption failure

Let $n$ be a nonce, $k$ be a key, $a$ be any associated data and $m$ be a
message and let $c, t = E(n, k, a, m)$. Authenticity means the probability is
negligable that a decryption succeeds for a different ciphertext and the same
nonce, key, associated data and tag.

\begin{equation}
    P(D(n, k, a, c', t) \neq \bot \mid c \neq c') < \frac{1}{2^S}
\end{equation}

\subsection{Ascon internals}

\subsubsection{Mode}

% TODO: Describe algorithm independent of mixing function.

\subsubsection{Permutation}

Ascon's main component is a permutation consisting of three phases, which are
applied in several rounds. Ascon-128 uses 6 rounds to process blocks of 64 bits
at a time. Ascon-128a uses 8 rounds to process blocks of 128 bits at a time.
Both use keys of 128 bits and 12 rounds to initialize and finalize the state. A
round consists of the following three phases: The addition of the round
constant, the substitution layer and the linear diffusion layer. Each of these
phases modify the internal state in a different way. The internal state consists
of 320 bits, logically split into five 64-bit words called $x_0$, $x_1$, $x_2$,
$x_3$ and $x_4$.

The round constant is a single byte that changes from round to round and is
added bitwise to the least significant 8 bits of $x_2$. It ensures rounds are
not all identical. The final round constant is always \hex{4b}. The round
constant changes linearly, decreasing by \hex{f} every round. This means
the round constant can be computed easily based on the number of rounds that are
left.

The substitution layer applies a 5-bit lookup function in parallel to the five
state words. It provides non-linear mixing and mixing between the five state
words. It is usually implemented as a binary formula applied bitwise to five
machine words at a time.

The linear diffusion layer provides linear mixing between the bits within each
state word. It consists of a bitwise addition of the original state word with
the same state word rotated by two different amounts. Each of the five state
words uses different rotation amounts.

\subsection{RISC-V}
RISC-V\cite{riscv} is an open-source hardware instruction set architecture.

% TODO: introduce add means bitwise exclusive or
% TODO: introduce machine word vs state word
% TODO: List all used instructions
% TODO: Describe simplicity / RISCness which makes formal proofs easier
