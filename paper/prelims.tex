\section{Preliminaries}

\subsection{Symmetric encryption}

In communication, it is often desirable to keep one's messages hidden from third
parties. This property is called \emph{confidentiality}. In more precise terms,
confidentiality means that a message is transformed in such a way that
authorized parties can recover the original message from it, while unauthorized
parties can not.

In symmetric encryption, the authorized parties are defined as
the parties that know some secret information, called the \emph{key}. The
original message is called the \emph{plaintext} and the transformed message is
called the \emph{ciphertext}. The transformation from plaintext to ciphertext is
called \emph{encryption}, while the transformation back is called
\emph{decryption}. When the same secret key is needed for both encryption and
decryption, it is called \emph{symmetric encryption}. Because the key is needed
for decryption, unauthorized parties are prevented from recovering the
plaintext.

\subsubsection{Nonces}

While unauthorized parties are unable to decrypt ciphertexts, similar plaintext
may result in similar ciphertexts, and unauthorized parties may therefore detect
when similar messages are sent. In order to prevent this, during every
encryption, a different number is used to modify the resulting ciphertext. It is
needed again during decryption in order to revert that modification. Because a
different number is used every time, the resulting ciphertexts will also differ.

This number is called the \emph{nonce}, short for number used once. If it is
used more than once, unauthorized parties may be able to infer information about
the difference or similarity of the messages it was used for. One option to
ensure uniqueness in practice is to use random numbers that are sufficiently
large to make the chance of duplicates negligable. When it is chosen randomly,
it must be attached to the ciphertext unencrypted, as decryption will fail
without it.

A counter can also be used to generate unique nonces, if it is sufficiently
large that it will never overflow within its lifetime. This has the advantage
that a good source of randomness is not needed, and the disadvantage that the
current count must be kept track of and synchronized between uses. If all
messages are sent synchronously and messages cannot be lost, this can be as
simple as attaching the current counter to each message.

If there is no space to attach the nonce to the ciphertext, it is not possible
to use a random nonce. In this case, information that is available during both
the encryption and decryption should be used to construct a unique nonce. For
example, when adding encryption support to existing filesystems, there may be no
space for nonces in the existing data structures and these probably need to stay
compatible with earlier versions. In this case, a nonce for a block can be
constructed by combining the inode and offset within the file into a nonce. With
this construction, uniqueness cannot be guaranteed, so security is reduced, but
if encryption is otherwise not possible, it is still an improvement over
unencrypted data or encryption without nonces.

\subsubsection{Authentication}

Although unauthorized parties cannot recover the plaintext from the ciphertext,
they may be able to modify the ciphertext, resulting in a modified plaintext
after decryption. It is not always possible to prevent such modifications, but
it is possible to detect them. The property that it is detectable whether a
message has been changed, is called \emph{integrity}.

Unauthorized parties may also attempt to construct messages from scratch.
Because the key is needed for encryption, they will be unable to encrypt a
specific plaintext, however, they will be able to send specific ciphertexts,
even if they do not know what plaintexts corresponds to them. In order to
prevent this, it is desirable for the receiving party to be able to verify that
a message comes from an authorized party. Together with integrity, this property
is called \emph{authenticity}. Encryption that provides both confidentiality and
authenticity to the plaintext is called \emph{authenticated encryption} or
\emph{AE}.

Authenticated encryption is usually implemented by generating an
\emph{authentication tag}. Just like the ciphertext, this tag is based on the
key, the nonce and the plaintext. This tag is needed again during decryption.
The decryption algorithm checks if the tag is correct and only returns a
plaintext if it is. If the ciphertext is modified, it will decrypt to a
different plaintext from the one used to generate the tag and cause the
decryption to fail. Because the tag also depends on the key, valid tags cannot
be created without it.

\subsubsection{Associated data}

Even when confidentiality and authenticity are assured, unauthorized parties may
still repeat a message they have seen before in a different context. To prevent
this, some data about the context in which a message is allowed to appear can be
associated with it. This is one of the uses of \emph{associated data}, which is
defined as information that requires authentication but not confidentiality.
An authenticated encryption scheme that supports this is called
\emph{authenticated encryption with associated data} or \emph{AEAD}.

In some encryption schemes, associated data is used during encryption to modify
the resulting ciphertext and is needed during decryption in order to revert that
modification. In these cases, because the correct associated data is needed
during decryption, it prevents messages from being decrypted using the wrong
associated data. In other schemes, associated data only affects the
authentication tag and it is the responsibility of the implementation to reject
inputs with incorrect authentication tags.

The associated data can be sent or stored together with the ciphertext, but in
some cases it can also be inferred from other data during encryption and
decryption. In those cases, sending or storing it is not necessary. If the
associated data is inferred from the context, it can prevent encrypted data from
being decrypted in the wrong context. For example, this prevents two encrypted
records containing positive and negative information respectively from being
swapped between contexts specifying different persons.

Note that some encryption schemes do not need the associated data for
decryption. Correct implementations of these will still check the associated
data and return failure if it does not match. The behavior is the same, but such
schemes allow incorrect implementations to go undetected more easily.

\begin{figure}
\begin{center}
\begin{tabular}{c l}
    \hex{1337} & hexadecimal number
    \\ $\bot$ & verification failure
    \\ $x \cat y$ & concatenation of bitstrings $x$ and $y$
    \\ $x \xor y$ & bitwise addition of bitstrings $x$ and $y$
    \\ $x \ror n$ & 64-bit word $x$ rotated right by $n$ bits
    \\ $\mathbf{2}^l$ & set of messages of $l$ bits
    \\ $\mathcal{P}(S)$ & Set of all subsets of set $S$.
\end{tabular}
\end{center}
\caption{The following table specifies the symbols and notation used in this
document.}
\label{notation}
\end{figure}

\subsubsection{Formal definition}

To formalize AEAD schemes, we begin by defining their interface: An AEAD scheme
is defined by a tuple of functions $(\mathcal{E}, \mathcal{D})$. $\mathcal{E}$
is a function that takes a key, a nonce, some associated data and a message and
produces a ciphertext and an authentication tag.

\begin{equation}
    \mathcal{E} : K \times N \times A \times M \to C \times T
\end{equation}

$\mathcal{D}$ is a function that takes a key, a nonce, some associated data, a
ciphertext and an authentication tag and produces either a failure or a message.

\begin{equation}
    \mathcal{D} : K \times N \times A \times C \times T \to M \cup \{ \bot \}
\end{equation}

Let $k$ be a key, $n$ be a nonce, $a$ be any associated data and $m$ be a
message. The result of decryption after encryption with the same nonce, key and
associated data is the original message.

\begin{equation}
    \forall k, n, a, m.\quad \mathcal{D}(k, n, a, \mathcal{E}(k, n, a, m)) = m
\end{equation}

In particular, this means decryption never returns $\bot$ when supplied with
valid encryption results. This property ensures the message can be recovered
from the ciphertext and related data. If this property were not required, an
AEAD scheme could be constructed simply by always returning the empty ciphertext
for encryption and failure for decryption.

The remaining properties of AEAD systems, authenticity and confidentiality, must
be defined with respect to a \emph{security parameter}. The security parameter
specifies how much effort is required to break the security properties of the
system. The security parameter is expressed in bits and is equivelant to the
2-log of the number of operations needed to break the security claim.

A security property holds when no algorithm exists that can break it in fewer
operation than specified by its security parameter. We say that no
\emph{efficient} algorithm exists. These algorithms implicitly have access to
all public knowledge.

We can now define the property of authenticity. First, we define $V_k$ to be the
set of valid decryption inputs for key $k$.

\begin{equation}
    \forall k \in K.\quad
    V_k = \{ (n, a, c, t) \mid \mathcal{D}(k, n, a, c, t) \neq \bot \}
\end{equation}

Authenticity is violated when an efficient algorithm exists that constructs
valid and new decryption inputs without the correct key. An encryption system
provides authenticity if no such algorithm exists.

\begin{equation}
    \nexists \mathcal{A}.\quad \forall k \in K, s \in \mathcal{P}(V_k).\quad
    \mathcal{A}(s) \notin s \wedge \mathcal{A}(s) \in V_k
\end{equation}

To define confidentiality, we must define what it means to know something about
a message. In order to keep this definition simple, we will assume the length of
a message is always known, as is often the case with AEAD schemes.

We will define knowledge about a message $m$ of length $l$ as a probability
distribution over all messages of that length, $\mathbf{2}^l$. We define $u_l$
to be the uniform probability distribution, which has no information about any
message.

\begin{equation}
    \forall l \in \mathbb{N}, m \in \mathbf{2}^l.\quad u_l(m) = \frac{1}{2^l}
\end{equation}

When the probability that a distribution $d$ picks $m$ is greater then the
probability $u$ picks $m$ by a factor of $2^b$, we say $d$ knows $b$ bits about
$m$. $I(d, m)$ represents the information in bits that $d$ has about $m$.

\begin{equation}
    \forall l \in \mathbb{N}, m \in \mathbf{2}^l, d \in D(\mathbf{2}^l).\quad
    I(d, m) = \textrm{log}_2 \frac{d(m)}{u_l(m)}
\end{equation}

An encryption scheme provides confidentiality if no efficient algorithm exists
that determines knowledge about a message given its decryption inputs and all
prior public knowledge. This public knowledge can include many other valid
decryption inputs, as well as decryption inputs with their corresponding
plaintexts, but with different nonces.

\begin{equation}
    \nexists \mathcal{A}.\quad \forall k, n, a, m.\quad
    I(\mathcal{A}(k, n, a, \mathcal{E}(k, n, a, m)), m) > 0
\end{equation}

It should be noted that the formal definitions given above are not completely
accurate. For example, it is hard to formalize the tradeoffs between
computational power and odds of succes for an attack. It is also hard to
formalize all the public data that attacking algorithms can be based on. There
are entire books dedicated to formalizing these properties so the formalizations
above should be taken with a grain of salt.

\subsection{Ascon internals}

Ascon~\cite{ascon} is an authenticated encryption cipher designed for use in
resource-constrained environments, like embedded devices. It has an internal
state of just 320 bits, which can be kept in registers on most architectures.
This ensures moving data between registers and memory is kept to a minimum,
which is important, as embedded devices usually do not have the same amount of
cache available as larger systems.

\subsubsection{Mode of operation}

Ascon aims to provide 128 bits of security. To that end, its key, nonce, and
authentication tag are 128 bits in size each. The plaintext, ciphertext and
associated data can all be of any length and are processed in \emph{blocks}.
There are multiple variants of Ascon, we implement two of them: Ascon-128, which
processes 64-bit blocks and Ascon-128a, which processes 128-bit blocks.

Ascon uses a 320-bit state and a permutation that mixes the state in a way that
is hard to reverse. This permutation consists of a transformation that is
applied in multiple rounds, each with a different round constant. Both variants
of Ascon use permutation $p^a$, which consists of 12 rounds, during
initialization and finalization. After processing each block, Ascon uses
permutation $p^b$, which consists of 6 rounds in Ascon-128 and 8 rounds in
Ascon-128a. Figure~\ref{encdec} gives an overview of encryption and decryption.

%%%%%%%%%%%%%%%%%%%%%%%%%%%%%%%%%%%%%%%%%%%%%%%%%%%%%%%%%%%%%%%%%%%%%%%%%%%%%%%%%%
% The Ascon encryption mode
%
% public domain (CC0 1.0 https://creativecommons.org/publicdomain/zero/1.0/)
%%%%%%%%%%%%%%%%%%%%%%%%%%%%%%%%%%%%%%%%%%%%%%%%%%%%%%%%%%%%%%%%%%%%%%%%%%%%%%%%%%

\newif\ifsans
\newif\iftext
\newif\ifdetails

%%% CONFIGURATION %%%%%%%%%%%%%%%%%%%%%%%%%%%%%%%%%%%%%%%%%%%%%%%%%%%%%%%%%%%%%%%%
% preferably use build.py

%\sanstrue  % for sans-serif fonts (slides, web)
\sansfalse  % for serif fonts (article)

%\texttrue  % include phase description
\textfalse  % no phase description

\detailsfalse % simplify (eg, 'const' instead of 'kab0'
%%%%%%%%%%%%%%%%%%%%%%%%%%%%%%%%%%%%%%%%%%%%%%%%%%%%%%%%%%%%%%%%%%%%%%%%%%%%%%%%%%

\ifsans
\renewcommand*\familydefault{\sfdefault}
\usepackage{sfmath}
\fi

\tikzset{sparsam/.style={inner sep=1pt}}
\tikzset{bitwidth/.style={above=-1pt, font=\tiny}}
\tikzset{next/.style={->, >=latex}}

\begin{tikzpicture}[scale=0.75, every node/.style={scale=0.75}]
  \newcommand{\conc}{\ensuremath{\Vert}}
  \newcommand{\perm}[1]{node[rectangle, rounded corners=3pt, minimum width=.5cm, minimum height=1.8cm, draw, sparsam] {$p^{#1}$}}
  \ifsans
    \newcommand{\tikzxor}[1]{node[circle, inner sep=-1.3pt, name={#1}] {\tikz{\draw[] (0,0) circle (3.75pt) +(3.75pt,0) -- +(-3.75pt,0) +(0,3.75pt) -- +(0,-3.75pt);}}}
  \else
    \newcommand{\tikzxor}[1]{node[circle, inner sep=-1.3pt, name={#1}] {$\oplus$}}
  \fi
  \newcommand{\bitwidth}{\tikz{\draw[-] (-2pt,-2pt) -- (2pt, 2pt);}}
  \newcommand{\rate}{.5cm}
  \newcommand{\msg}{.6cm}
  \newcommand{\phase}{1.7cm}
  \newcommand{\minnext}{.4cm}

  \clip(-3,-2) rectangle (6,2);

  % --- init up to p^a ---
  \begin{scope}[xshift=0cm]
    \ifdetails
      \draw (0,\rate) node[left, sparsam] {$k \conc r \conc a \conc b \conc 0^*$};
      \draw (0,-\rate) node[left, sparsam] {$K \conc N$};

      \draw[next] (0,\rate) -- node {\bitwidth} node[bitwidth] {$r$} (.7,\rate);
      \draw[next] (0,-\rate) -- node {\bitwidth} node[bitwidth] {$c$} (.7,-\rate);
    \else
      \draw (0,0) node[left, sparsam] {$\mathrm{IV} \conc K \conc N$};
      \draw[next] (0,0) -- node {\bitwidth} node[bitwidth] {$320$} (.7,0);
    \fi

    \draw (.95,0) \perm{a};
  \end{scope}

  % --- init after p^a and auth A1 ---
  \begin{scope}[xshift=1.2cm]
    \draw (.4,-\rate) \tikzxor{Ki};
    \draw[next] (0,-\rate) -- (Ki);
    \draw[next] (Ki) +(0,-\msg) node[below] {$0^* \conc K$ \hspace*{.3cm}} -- (Ki);
    \draw[next] (Ki) -- node[pos=0.6] {\bitwidth} node [pos=0.6, bitwidth] {$c$} +(1.3,0);

    \draw[dashdotted] (.8,1.5) -- (.8,-1.5);

    \draw (1.3,\rate) \tikzxor{A1};
    \draw[next] (0,\rate) -- node[near start] {\bitwidth} node[near start, bitwidth] {$r$} (A1);
    \draw[next] (A1) +(0,\msg) node[above] {$A_1$} -- (A1);
    \draw[next] (A1) -- +(\minnext,0);

    \draw (1.95,0) \perm{b};
  \end{scope}

  % --- auth As ---
  \begin{scope}[xshift=3.4cm]
    \draw[dotted] (0,\rate) -- (\minnext,\rate)
                  (0,-\rate) -- (\minnext,-\rate);

    \draw (.9,\rate) \tikzxor{As};
    \draw[next] (\minnext,\rate) -- (As);
    \draw[next] (As) +(0,\msg) node[above] {$A_s$} -- (As);
    \draw[next] (As) -- +(\minnext,0);

    \draw[next] (\minnext,-\rate) -- node {\bitwidth} node[bitwidth] {$c$} (1.3,-\rate);

    \draw (1.55,0) \perm{b};
  \end{scope}

  % --- enc P1 ---
  \begin{scope}[xshift=5.2cm]
    \draw (.4,-\rate) \tikzxor{AuthPad};
    \draw[next] (0,-\rate) -- (AuthPad);
    \draw[next] (AuthPad) +(0,-\msg) node[below] {$0^* \conc 1$ \hspace*{.3cm}} -- (AuthPad);
    \draw[next] (AuthPad) -- node {\bitwidth} node [bitwidth] {$c$} +(1.7,0);

    \draw[dashdotted] (.8,1.5) -- (.8,-1.5);

    \draw (1.3,\rate) \tikzxor{P1};
    \draw[next] (0,\rate) -- node[near start] {\bitwidth} node[near start, bitwidth] {$r$} (P1);
    \draw[next] (P1) +(0,\msg) node[above] {$P_1$} -- (P1);
    \draw[next] (P1) -- +(2*\minnext,0);
    \draw[next] (P1) ++(.5,0) -- +(0,\msg) node[above] {$C_1$};

    \draw (2.35,0) \perm{b};
  \end{scope}

  %% --- enc Pt-1 ---
  %\begin{scope}[xshift=7.8cm]
  %  \draw[dotted] (0,\rate) -- (\minnext,\rate)
  %                (0,-\rate) -- (\minnext,-\rate);

  %  \draw[next] (\minnext,-\rate) -- node[pos=.4] {\bitwidth} node [pos=.4,bitwidth] {$c$} +(1.3,0);

  %  \draw (.9,\rate) \tikzxor{Pt1};
  %  \draw[next] (\minnext,\rate) -- (Pt1);
  %  \draw[next] (Pt1) +(0,\msg) node[above] {$P_{t\!-\!1}$ \hspace*{.15cm}} -- (Pt1);
  %  \draw[next] (Pt1) -- +(.8,0);
  %  \draw[next] (Pt1) ++(.5,0) -- +(0,\msg) node[above] {\hspace*{.2cm} $C_{t\!-\!1}$};

  %  \draw (1.95,0) \perm{b};
  %\end{scope}

  %% --- enc Pt and finalize ---
  %\begin{scope}[xshift=10.0cm]
  %  \draw (.5,\rate) \tikzxor{Pt};
  %  \draw[next] (0,\rate) -- (Pt);
  %  \draw[next] (Pt) +(0,\msg) node[above] {$P_t$} -- (Pt);
  %  \draw[next] (Pt) ++(.5,0) -- +(0,\msg) node[above] {$C_t$};
  %  \draw[next] (Pt) -- node[pos=.7] {\bitwidth} node[pos=.7,bitwidth] {$r$} +(1.6,0);

  %  \draw[dashdotted] (1.3,1.5) -- (1.3,-1.5);

  %  \draw (1.7,-\rate) \tikzxor{Kf};
  %  \draw[next] (Kf) +(0,-\msg) node[below] {\hspace*{.3cm} $K \conc 0^*$} -- (Kf);
  %  \draw[next] (0,-\rate) -- node[pos=.3] {\bitwidth} node[pos=.3,bitwidth] {$c$} (Kf);
  %  \draw[next] (Kf) -- (2.1,-\rate);

  %  \draw (2.35,0) \perm{a};

  %  \draw (3.3,-\rate) \tikzxor{Kt};
  %  \draw[next] (Kt) +(0,-\msg) node[below] {$K$} -- (Kt);
  %  \draw[next] (2.6,-\rate) -- node[pos=.4] {\bitwidth} node[pos=.4,bitwidth] {$k$} (Kt);
  %  \draw (4.0,-\rate) node[name=T,sparsam] {$T$};
  %  \draw[next] (Kt) -- (T);
  %\end{scope}

  % --- phase descriptions ---
  \iftext
    \draw (.5,-\rate-\phase) node {Initialization};
    \draw (4.0,-\rate-\phase) node {Associated Data};
    \draw (8.5,-\rate-\phase) node {Plaintext};
    \draw (12.8,-\rate-\phase) node {Finalization};
  \fi
\end{tikzpicture}

%\begin{tikzpicture}[scale=0.6, every node/.style={scale=0.6}]
%  \newcommand{\conc}{\ensuremath{\Vert}}
%  \newcommand{\perm}[1]{node[rectangle, rounded corners=3pt, minimum width=.5cm, minimum height=1.8cm, draw, sparsam] {$p^{#1}$}}
%  \ifsans
%    \newcommand{\tikzxor}[1]{node[circle, inner sep=-1.3pt, name={#1}] {\tikz{\draw[] (0,0) circle (3.75pt) +(3.75pt,0) -- +(-3.75pt,0) +(0,3.75pt) -- +(0,-3.75pt);}}}
%  \else
%    \newcommand{\tikzxor}[1]{node[circle, inner sep=-1.3pt, name={#1}] {$\oplus$}}
%  \fi
%  \newcommand{\bitwidth}{\tikz{\draw[-] (-2pt,-2pt) -- (2pt, 2pt);}}
%  \newcommand{\rate}{.5cm}
%  \newcommand{\msg}{.6cm}
%  \newcommand{\phase}{1.7cm}
%  \newcommand{\minnext}{.4cm}
%
%  % --- init up to p^a ---
%  \begin{scope}[xshift=0cm]
%    \ifdetails
%      \draw (0,\rate) node[left, sparsam] {$k \conc r \conc a \conc b \conc 0^*$};
%      \draw (0,-\rate) node[left, sparsam] {$K \conc N$};
%
%      \draw[next] (0,\rate) -- node {\bitwidth} node[bitwidth] {$r$} (.7,\rate);
%      \draw[next] (0,-\rate) -- node {\bitwidth} node[bitwidth] {$c$} (.7,-\rate);
%    \else
%      \draw (0,0) node[left, sparsam] {$\mathrm{IV} \conc K \conc N$};
%      \draw[next] (0,0) -- node {\bitwidth} node[bitwidth] {$320$} (.7,0);
%    \fi
%
%    \draw (.95,0) \perm{a};
%  \end{scope}
%
%  % --- init after p^a and auth A1 ---
%  \begin{scope}[xshift=1.2cm]
%    \draw (.4,-\rate) \tikzxor{Ki};
%    \draw[next] (0,-\rate) -- (Ki);
%    \draw[next] (Ki) +(0,-\msg) node[below] {$0^* \conc K$ \hspace*{.3cm}} -- (Ki);
%    \draw[next] (Ki) -- node[pos=0.6] {\bitwidth} node [pos=0.6, bitwidth] {$c$} +(1.3,0);
%
%    \draw[dashdotted] (.8,1.5) -- (.8,-1.5);
%
%    \draw (1.3,\rate) \tikzxor{A1};
%    \draw[next] (0,\rate) -- node[near start] {\bitwidth} node[near start, bitwidth] {$r$} (A1);
%    \draw[next] (A1) +(0,\msg) node[above] {$A_1$} -- (A1);
%    \draw[next] (A1) -- +(\minnext,0);
%
%    \draw (1.95,0) \perm{b};
%  \end{scope}
%
%  % --- auth As ---
%  \begin{scope}[xshift=3.4cm]
%    \draw[dotted] (0,\rate) -- (\minnext,\rate)
%                  (0,-\rate) -- (\minnext,-\rate);
%
%    \draw (.9,\rate) \tikzxor{As};
%    \draw[next] (\minnext,\rate) -- (As);
%    \draw[next] (As) +(0,\msg) node[above] {$A_s$} -- (As);
%    \draw[next] (As) -- +(\minnext,0);
%
%    \draw[next] (\minnext,-\rate) -- node {\bitwidth} node[bitwidth] {$c$} (1.3,-\rate);
%
%    \draw (1.55,0) \perm{b};
%  \end{scope}
%
%  % --- enc P1 ---
%  \begin{scope}[xshift=5.2cm]
%    \draw (.4,-\rate) \tikzxor{AuthPad};
%    \draw[next] (0,-\rate) -- (AuthPad);
%    \draw[next] (AuthPad) +(0,-\msg) node[below] {$0^* \conc 1$ \hspace*{.3cm}} -- (AuthPad);
%    \draw[next] (AuthPad) -- node {\bitwidth} node [bitwidth] {$c$} +(1.7,0);
%
%    \draw[dashdotted] (.8,1.5) -- (.8,-1.5);
%
%    \draw (1.3,\rate) \tikzxor{P1};
%    \draw[next] (0,\rate) -- node[near start] {\bitwidth} node[near start, bitwidth] {$r$} (P1);
%    \draw[next] (P1) -- +(0,\msg) node[above] {$P_1$};
%    \draw[<->,>=latex] (P1) -- +(2*\minnext,0);
%    \draw[next] (P1) ++(.5,\msg) node[above] {$C_1$} -- +(0,-\msg);
%
%    \draw (2.35,0) \perm{b};
%  \end{scope}
%
%  % --- enc Pt-1 ---
%  \begin{scope}[xshift=7.8cm]
%    \draw[dotted] (0,\rate) -- (\minnext,\rate)
%                  (0,-\rate) -- (\minnext,-\rate);
%
%    \draw[next] (\minnext,-\rate) -- node[pos=.4] {\bitwidth} node [pos=.4,bitwidth] {$c$} +(1.3,0);
%
%    \draw (.9,\rate) \tikzxor{Pt1};
%    \draw[next] (\minnext,\rate) -- (Pt1);
%    \draw[next] (Pt1) -- +(0,\msg) node[above] {$P_{t\!-\!1}$ \hspace*{.15cm}};
%    \draw[<->,>=latex] (Pt1) -- +(.8,0);
%    \draw[next] (Pt1) ++(.5,\msg) node[above] {\hspace*{.2cm} $C_{t\!-\!1}$} -- +(0,-\msg);
%
%    \draw (1.95,0) \perm{b};
%  \end{scope}
%
%  % --- enc Pt and finalize ---
%  \begin{scope}[xshift=10.0cm]
%    \draw (.5,\rate) \tikzxor{Pt};
%    \draw[next] (0,\rate) -- (Pt);
%    \draw[next] (Pt) -- +(0,\msg) node[above] {$P_t$};
%    \draw[next] (Pt) ++(.5,\msg) node[above] {$C_t$} -- +(0,-\msg);
%    \draw[<->,>=latex] (Pt) -- node[pos=.7] {\bitwidth} node[pos=.7,bitwidth] {$r$} +(1.6,0);
%
%    \draw[dashdotted] (1.3,1.5) -- (1.3,-1.5);
%
%    \draw (1.7,-\rate) \tikzxor{Kf};
%    \draw[next] (Kf) +(0,-\msg) node[below] {\hspace*{.3cm} $K \conc 0^*$} -- (Kf);
%    \draw[next] (0,-\rate) -- node[pos=.3] {\bitwidth} node[pos=.3,bitwidth] {$c$} (Kf);
%    \draw[next] (Kf) -- (2.1,-\rate);
%
%    \draw (2.35,0) \perm{a};
%
%    \draw (3.3,-\rate) \tikzxor{Kt};
%    \draw[next] (Kt) +(0,-\msg) node[below] {$K$} -- (Kt);
%    \draw[next] (2.6,-\rate) -- node[pos=.4] {\bitwidth} node[pos=.4,bitwidth] {$k$} (Kt);
%    \draw (4.0,-\rate) node[name=T,sparsam] {$T$};
%    \draw[next] (Kt) -- (T);
%  \end{scope}
%
%  % --- phase descriptions ---
%  \iftext
%    \draw (.5,-\rate-\phase) node {Initialization};
%    \draw (4.0,-\rate-\phase) node {Associated Data};
%    \draw (8.5,-\rate-\phase) node {Plaintext};
%    \draw (12.8,-\rate-\phase) node {Finalization};
%  \fi
%\end{tikzpicture}


Encryption starts with the initialization of the state. The state is initialized
to the concatenation of an initialization vector, the key and the nonce. The
initialization vector encodes the parameters of the variant of Ascon that it is
used for. This initial state is then passed through the first permutation,
$p^b$. Finally, the key is bitwise added to the least significant bits of the
state again.

After initialization, the associated data is mixed into the state. If there is
no associated data, the state is left unmodified. Otherwise, the associated data
is padded and split into blocks. Each of the blocks is bitwise added to the most
significant bits of the state, followed by applying the second permutation,
$p^b$. At the end, the least significant bit is inverted. This is done to
separate the associated data from the plaintext, which comes next.

The plaintext is padded and split into blocks in the same way as the associated
data. Each block of plaintext is bitwise added to the most significant bits of
the state. The result of this bitwise addition also forms the corresponding
block of the ciphertext. After adding each block, except for the last,
permutation $p^b$ is applied again.

Finalization begins with bitwise adding the key to the most significant bits of
the state that were not used to add the plaintext. After that, the permutation
$p^a$ is applied again and finally, the key is bitwise added to the least
significant bits of the state again. The result of this last operation is used
as the authentication tag.

Decryption is identical to encryption, except the ciphertext is processed in a
slightly different way from the plaintext: Each block of the ciphertext is again
bitwise added to the most significant bits of the state to form the plaintext,
but instead of using this result in the state, the most significant bits of the
state are replaced with the ciphertext before the bitwise addition.

\subsubsection{Permutation}

Ascon's main component is a permutation consisting of three phases, which are
applied in several rounds. Ascon-128 uses 6 rounds to process blocks of 64 bits
at a time. Ascon-128a uses 8 rounds to process blocks of 128 bits at a time.
Both use keys of 128 bits and 12 rounds to initialize and finalize the state. A
round consists of the following three phases: The addition of the round
constant, the substitution layer and the linear diffusion layer. Each of these
phases modify the internal state in a different way. The internal state consists
of 320 bits, logically split into five 64-bit state words called $x_0$, $x_1$,
$x_2$, $x_3$ and $x_4$.

The round constant is a single byte that changes from round to round and is
added bitwise to the least significant 8 bits of $x_2$. It ensures rounds are
not all identical. The final round constant is always \hex{4b}. The round
constant changes linearly, decreasing by \hex{f} every round. This means
the round constant can be computed easily based on the number of rounds that are
left.

The substitution layer applies a 5-bit lookup function (called an S-box) in
parallel to the five state words. It provides non-linear mixing between the five
state words. It is usually implemented as a sequence of bitwise operations which
compute the substitution on five machine words in parallel.

The linear diffusion layer provides linear mixing of the bits within each state
word. It consists of a bitwise addition of the original state word with the same
state word rotated by two different amounts. Each of the five state words uses
different rotation amounts.

% TODO: Expand Ascon permutation?

\subsection{RISC-V}
RISC-V~\cite{riscv} is a hardware instruction set architecture which aims to be
completely open: Anyone is free to create custom variations and implementations.
It is also designed to be modular: A basic RISC-V processor starts with just the
base integer instruction set \texttt{I}, using either 32-bit or 64-bit machine
words. Several extensions are available which offer different common processor
features, such as support for floating point operations. Manufacturers can
choose to implement any combination of these extensions and will usually pick
extensions based on the intended application.

In this thesis, we will use the HiFive1 development board created by SiFive. It
features a \texttt{RV32IMAC} processor. The \texttt{RV32I} stands for the RISC-V
32-bit base integer instruction set. The letters that follow stand for the
extensions that are implemented in this processor: \texttt{M} stands for the
multiplication extensions which offers integer multiplication and division
instructions, \texttt{A} stands for the atomic memory access extension and
\texttt{C} stands for the compressed instruction set extension which can be used
to reduce code size.

At this point it is important to note that our target processor uses 32-bit
\emph{machine words}, while Ascon is defined in terms of five 64-bit
\emph{state words}. These state words do not fit in a single register, so each
state word will be stored using two registers. We describe our implementation
mostly in terms of machine words.

For Ascon, we won't need multiplication instructions or atomic memory access.
The compressed instructions are used transparently by the assembler: Each
compressed instruction corresponds to a single full size instruction and
whenever a compressed instruction is available, it will be used instead of a
full size instruction. We would like to use the bit manipulation extension, but
it is not available on this board and may not be available on other embedded
RISC-V systems, so we optimize Ascon for RISC-V processors without it.

\emph{RISC} stands for Reduced Instruction Set Computer, as opposed to CISC,
which stands for Complex Instruction Set Computer. RISC architectures aim to
have a small set of simple and general instructions, while CISC architectures
aim to provide complex instructions which can do multiple things at once. An
example of this is that the CISC x86 architecture allows memory access as part
of almost every instruction, while RISC architectures usually have separate
memory access instructions.

RISC-V takes RISC to an extreme, it is much more reduced than ARM, which stands
for Advanced RISC Machine. For example, RISC-V does not save carries for
arithmetic operations and it only has about 40 instruction in the base integer
instruction set, all of which do only one thing. Figure~\ref{opcodes} shows the
eight instructions needed for the inner loop of Ascon.

% TODO: instructions specify destination first
% TODO: macro syntax
% TODO: Check for other missing RISC-V info

\begin{figure}
\begin{tabular}{l p{0.75\linewidth}}
       \texttt{and r, a, b} & Store the bitwise and of registers~\texttt{a} and
        \texttt{b} in register~\texttt{r}.
    \\ \texttt{or r, a, b} & Store the bitwise or of registers~\texttt{a} and
        \texttt{b} in register~\texttt{r}.
    \\ \texttt{xor r, a, b} & Store the bitwise addition of registers~\texttt{a}
        and \texttt{b} in register~\texttt{r}. This operation is also known as
        bitwise exclusive or.
    \\ \texttt{not r, a} & Store the bitwise inversion of register~\texttt{a} in
        register~\texttt{r}.
    \\ \texttt{addi r, a, b} & Store the sum of the value of register~\texttt{a}
        and constant~\texttt{b} in register~\texttt{r}.
    \\ \texttt{srli r, a, b} & Store the value of register~\texttt{a} shifted
        right by constant amount~\texttt{b} in register~\texttt{r}.
    \\ \texttt{ssli r, a, b} & Store the value of register~\texttt{a} shifted
        left by constant amount~\texttt{b} in register~\texttt{r}.
    \\ \texttt{bne a, b, l} & Jump to the label~\texttt{l} if the value of
        register~\texttt{a} is not the same as the value of register~\texttt{b}.
\end{tabular}

\caption{The RISC-V instructions used in the inner loop of Ascon.}

\label{opcodes}
\end{figure}
