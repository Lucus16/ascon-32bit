\section{Preliminaries}

\subsection{Ascon internals}

Ascon's main component is a permutation consisting of three phases, which are
applied in several rounds. Ascon-128 uses 6 rounds to process blocks of 64 bits
at a time. Ascon-128a uses 8 rounds to process blocks of 128 bits at a time.
Both use keys of 128 bits and 12 rounds to initialize and finalize the state. A
round consists of the following three phases: The addition of the round
constant, the substitution layer and the linear diffusion layer. Each of these
phases modify the internal state in a different way. The internal state consists
of 320 bits, logically split into 5 64-bit words called $x_0$, $x_1$, $x_2$,
$x_3$ and $x_4$.

\subsubsection{Round phases}

The round constant is a single byte that changes from round to round and is
added bitwise to the least significant 8 bits of $x_2$. It ensures rounds are
not all identical. The final round constant is always $\mathtt{4b}_h$. The round
constant changes linearly, decreasing by $\mathtt{f}_h$ every round. This means
the round constant can be computed easily based on the number of rounds that are
left.

The substitution layer applies a 5-bit lookup function in parallel to the five
state words. It provides non-linear mixing and mixing between the five state
words. It is usually implemented as a binary formula applied bitwise to five
machine words at a time.

The linear diffusion layer provides linear mixing between the bits within each
state word. It consists of a bitwise addition of the original state word with
the same state word rotated by two different amounts. Each of the five state
words uses different rotation amounts.

\subsubsection{Processing steps}



\subsection{RISC-V}
RISC-V\cite{riscv} is an open-source hardware instruction set architecture.

% TODO: introduce add means bitwise exclusive or
% TODO: introduce machine word vs state word

\subsection{Cortex-M4}

