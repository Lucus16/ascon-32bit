\section{Optimization}

\subsection{Permutation}

Ascon keeps an internal state of 320 bits. On 64-bit platforms, this is usually
kept in 5 registers, on 32-bit platforms, it needs to be kept in 10.

The core of Ascon consists of a transformation applied in several rounds. Each
round consists of three phases: The addition of the round constant, the
substitution layer and the linear diffusion layer.

\subsection{Substitution layer}

The substitution layer in Ascon is a 5-bit S-box which is applied in parallel to
64 sets of 5 bits. The paper describes an instruction sequence of bitwise
operations that will compute this S-box on an entire machine word in parallel.
It takes 5 input words, one for each bit in the S-box, and produces 5 output
words. The instruction sequence translates to 22 single-cycle instructions.
Because these can only be applied to 32 bits at a time on 32-bit platforms, they
need to be run twice. Therefor, a straightforward implementation takes 44
cycles. The instruction sequence is as follows:

\begin{samepage}
\begin{verbatim}
x0 ^=  x4;  x4 ^=  x3;  x2 ^=  x1;
t0  = ~x0;  t1  = ~x1;  t2  = ~x2;  t3  = ~x3;  t4  = ~x4;
t0 &=  x1;  t1 &=  x2;  t2 &=  x3;  t3 &=  x4;  t4 &=  x0;
x0 ^=  t1;  x1 ^=  t2;  x2 ^=  t3;  x3 ^=  t4;  x4 ^=  t0;
x1 ^=  x0;  x0 ^=  x4;  x3 ^=  x2;  x2  = ~x2;
\end{verbatim}
\end{samepage}

The substitution layer can be optimized by computing the same S-box with a
different formula. I derived shorter binary formulas by first writing down the
the bit sequences that occur for each of the 5 output bits for all 32 inputs.
This allowed us to recognize patterns in the bit sequences. I first eliminated
input bits that did not affect the output, or only affected the output through a
single exclusive or operation. Then, I looked at the remaining bit patterns and
found short and overlapping binary formulas for them. Here, $o_n$ indicates
output bit $n$ and $i_n$ indicates input bit $n$.

\begin{samepage}
\begin{align*}
   o_0 & = i_3 \oplus i_4 \oplus (i_1 \vee (i_0 \oplus i_2 \oplus i_4))
\\ o_1 & = i_0 \oplus i_4 \oplus ((i_2 \oplus i_1) \vee (i_3 \oplus i_2))
\\ o_2 & = i_1 \oplus i_2 \oplus (i_3 \vee \neg i_4)
\\ o_3 & = i_1 \oplus i_2 \oplus (i_0 \vee (i_4 \oplus i_3))
\\ o_4 & = i_3 \oplus i_4 \oplus (i_1 \wedge \neg (i_4 \oplus i_0))
\end{align*}
\end{samepage}

From these, an instruction sequence can be produced like the following:

\begin{samepage}
\begin{verbatim}
t12  =  x1 ^ x2; t04  = x0 ^ x4; t34  = x3 ^ x4;
x4   = ~x4;      x4  |= x3;      x4  ^= t12;
x3  ^=  x1;      x3  |= t12;     x3  ^= t04;
x2  ^=  t04;     x2  |= x1;      x2  ^= t34;
t04  = ~t04;     x1  &= t04;     x1  ^= t34;
                 x0  |= t34;     x0  ^= t12;
\end{verbatim}
\end{samepage}

This instruction sequence computes all five of the above formulas, but the
results end up in different registers. It is possible to compensate for this in
the linear diffusion layer.

\subsubsection{Superoptimization}

As these formulas were constructed by hand, it may be possible to do better.
There are many possible formulas, so it is infeasable to find the best option by
hand. One option is to use something like the GNU Superoptimizer which tries all
possible instruction sequence of a certain length in order to see if any of them
computes a specific formula. Unfortunately, according to its README, the longest
instruction sequence it was able to find for anything was seven instructions
long. This is not enough, since it's expected that at least five XOR operations
and at least five AND operations need to be computed, leading to a minimum of 10
instructions.

Ko Stoffelen\cite{sat} attempted to optimize the binary formulas using a SAT
solver. He found ways to compile an S-box to a satisfiability problem
determining whether it can be computed in a given number of instructions. I made
use of his project to prove that it is not possible to compute the Ascon S-box
in 10 instructions. Unfortunately, the project did not finish within reasonable
time for instruction counts larger than 10, so I was unable to verify whether 17
instructions is the best number possible.

\subsection{Linear diffusion layer}

The linear diffusion layer in Ascon mixes the bits with each of the five state
words. Each of them is rotated by two different amounts and added with the
results. For each of the five state words, two different rotation amounts are
used.

The state words are 64 bits in size, so 64-bit rotations are used. On most
64-bit architectures, this can be done with just one instruction, but 32-bit
architectures generally don't have 64-bit rotate instructions.

The most straightforward method of simulating a 64-bit rotate instruction is
using shifts. By combining a left shift and a right shift, all bits of a single
machine word can be placed at the correct offset. Because each shift throws
away some of the bits, it is not possible to simulate a rotate instruction with
one shift. Because a state word is two machine words in size, four shifts are
needed in total. After this, there are four intermediate results that need to be
combined to two resulting machine words. There are several options to combine
these intermediate results:

Since each bit will be zero in at least one of the two operands, the combining
instructions can be bitwise or, bitwise exclusive or, or integer addition
instructions. Since the results will be combined using bitwise exclusive or
operations, we will use these for combining intermediate shift results as well.
Because bitwise exclusive or operations are associative, it doesn't matter in
what order all intermediate results are combined, which gives us more freedom
during implementation. Bitwise exclusive or can also be seen as bitwise
addition, or addition without any carries between the bits. I will use
\emph{bitwise addition} to refer to this operation from here on.

Using shifts means throwing away bits, which is why four of them are needed. It
would be preferable to use rotate instructions, since they don't throw away
bits. Unfortunately, rotating both halves of a state word independently does not
have the same effect as rotating the word as a whole.

This is usually resolved using bit interleaving: By storing even-numbered bits
in one word and odd-numbered bits in another, a 64-bit rotation can be simulated
using two 32-bit rotations. For even rotation amounts, both words are rotated by
half the amount. For odd rotation amounts, the words are rotated by half the
amount, rounded up and down respectively, and then swapped. The disadvantage is
that some extra cycles are needed to convert between bit-interleaved
representation and normal representation.

Unfortunately, the microcontroller we're targeting does not have rotate
instructions at all. The SiFive HiFive1 does not include the bit manipulation
extension, which is where RISC-V introduces its rotate instructions. This means
we're stuck using the straightforward method using four shifts and two bitwise
additions. The result also needs to be bitwise added back into the original,
which takes another two instructions, one for each of the machine words. This
means that handling one rotation amount takes eight instructions in total.

There are five state words which each have two rotations applied to them. This
means a total of 80 instructions are needed for the linear diffusion layer. On
32-bit RISC-V, without the bit manipulation extension, it is not possible to do
better. This is easy to see:

The only method of moving bits to different offsets is the shift instruction,
and because it throws away bits, there are always two needed to handle 32 bits,
or one machine word. Therefor, to move all bits into two separate new positions
for five state words, or ten machine word, a total of 40 shifts are needed. Each
of these shifts produces an intermediate result and all those intermediate
results need to be merged back into the unrotated state words.

Each merging instruction takes two operands and produces one result, reducing
the number of intermediate results by one, so a total of 40 merging instructions
is needed to merge all intermediate results. On the HiFive1, there are no
instructions that can merge more than two machine words at a time. This means it
is not possible to do the linear diffusion layer in less than 80 instructions.
Figure~\ref{xorror} shows how the rotations for a single state word are
implemented in RISC-V assembly.

\begin{figure}
\begin{verbatim}
.macro xorror dl, dh, sl, sh, sl0, sh0, r0, sl1, sh1, r1, t0, t1, t2
        slli \t0, \sl0, (32 - \r0)
        srli \t2, \sh0, \r0
        xor \t0, \t0, \t2
        slli \t2, \sl1, (32 - \r1)
        xor \t0, \t0, \t2
        srli \t2, \sh1, \r1
        xor \t0, \t0, \t2
        slli \t1, \sh0, (32 - \r0)
        srli \t2, \sl0, \r0
        xor \t1, \t1, \t2
        slli \t2, \sh1, (32 - \r1)
        xor \t1, \t1, \t2
        srli \t2, \sl1, \r1
        xor \t1, \t1, \t2
        xor \dl, \sl, \t1
        xor \dh, \sh, \t0
.endm
\end{verbatim}

\caption{The \texttt{xorror} macro applies the rotates and bitwise additions of
the linear diffusion layer to a single state word. Because of the limitations of
assembly macros, it takes many arguments:
\texttt{dl} and \texttt{dh} are the registers for storing the low and high part
of the result. \texttt{sl}, \texttt{sh}, \texttt{sl0}, \texttt{sh0},
\texttt{sl1} and \texttt{sh1} are the source registers for the state word,
without rotation, with rotation \texttt{r0} and with rotation \texttt{r1}
respectively. Because the shift amounts can only be 32 at most, representing
rotate amounts above 64 is done by subtracting 32 and swapping the respective
source registers. This is why the source registers must be supplied thrice.
\texttt{t0}, \texttt{t1} and \texttt{t2} are three temporary registers used to
store intermediate results.}
\label{xorror}
\end{figure}
