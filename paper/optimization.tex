\section{Optimization}

\subsection{Permutation}

Ascon keeps an internal state of 320 bits. On 64-bit platforms, this is usually
kept in 5 registers, on 32-bit platforms, it needs to be kept in 10.

The core of Ascon consists of a transformation applied in several rounds. Each
round consists of three phases: The addition of the round constant, the
substitution layer and the linear diffusion layer.

\subsection{Substitution layer}

The substitution layer in Ascon is a 5-bit S-box which is applied in parallel to
64 sets of 5 bits. The paper describes an instruction sequence of bitwise
operations that will compute this S-box on an entire machine word in parallel.
It takes 5 input words, one for each bit in the S-box, and produces 5 output
words. The instruction sequence translates to 22 single-cycle instructions.
Because these can only be applied to 32 bits at a time on 32-bit platforms, they
need to be run twice. Therefor, a straightforward implementation takes 44
cycles. The instruction sequence is as follows:

\begin{samepage}
\begin{verbatim}
x0 ^=  x4;  x4 ^=  x3;  x2 ^=  x1;
t0  = ~x0;  t1  = ~x1;  t2  = ~x2;  t3  = ~x3;  t4  = ~x4;
t0 &=  x1;  t1 &=  x2;  t2 &=  x3;  t3 &=  x4;  t4 &=  x0;
x0 ^=  t1;  x1 ^=  t2;  x2 ^=  t3;  x3 ^=  t4;  x4 ^=  t0;
x1 ^=  x0;  x0 ^=  x4;  x3 ^=  x2;  x2  = ~x2;
\end{verbatim}
\end{samepage}

The substitution layer can be optimized by computing the same S-box with a
different formula. I derived shorter binary formulas by first writing down the
the bit sequences that occur for each of the 5 output bits for all 32 inputs.
This allowed us to recognize patterns in the bit sequences. I first eliminated
input bits that did not affect the output, or only affected the output through a
single exclusive or operation. Then, I looked at the remaining bit patterns and
found short and overlapping binary formulas for them. Here, $o_n$ indicates
output bit $n$ and $i_n$ indicates input bit $n$.

\begin{samepage}
\begin{align*}
   o_0 & = i_3 \oplus i_4 \oplus (i_1 \vee (i_0 \oplus i_2 \oplus i_4))
\\ o_1 & = i_0 \oplus i_4 \oplus ((i_2 \oplus i_1) \vee (i_3 \oplus i_2))
\\ o_2 & = i_1 \oplus i_2 \oplus (i_3 \vee \neg i_4)
\\ o_3 & = i_1 \oplus i_2 \oplus (i_0 \vee (i_4 \oplus i_3))
\\ o_4 & = i_3 \oplus i_4 \oplus (i_1 \wedge \neg (i_4 \oplus i_0))
\end{align*}
\end{samepage}

From these, an instruction sequence can be produced like the following:

\begin{samepage}
\begin{verbatim}
t12  =  x1 ^ x2; t04  = x0 ^ x4; t34  = x3 ^ x4;
x4   = ~x4;      x4  |= x3;      x4  ^= t12;
x3  ^=  x1;      x3  |= t12;     x3  ^= t04;
x2  ^=  t04;     x2  |= x1;      x2  ^= t34;
t04  = ~t04;     x1  &= t04;     x1  ^= t34;
                 x0  |= t34;     x0  ^= t12;
\end{verbatim}
\end{samepage}

This instruction sequence computes all five of the above formulas, but the
results end up in different registers. It is possible to compensate for this in
the linear diffusion layer.

\subsubsection{Superoptimization}

As these formulas were constructed by hand, it may be possible to do better.
There are many possible formulas, so it is infeasable to find the best option by
hand. One option is to use something like the GNU Superoptimizer which tries all
possible instruction sequence of a certain length in order to see if any of them
computes a specific formula. Unfortunately, according to its README, the longest
instruction sequence it was able to find for anything was seven instructions
long. This is not enough, since it's expected that at least five XOR operations
and at least five AND operations need to be computed, leading to a minimum of 10
instructions.

Ko Stoffelen\cite{sat} attempted to optimize the binary formulas using a SAT
solver. He found ways to compile an S-box to a satisfiability problem
determining whether it can be computed in a given number of instructions. I made
use of his project to prove that it is not possible to compute the Ascon S-box
in 10 instructions. Unfortunately, the project did not finish within reasonable
time for instruction counts larger than 10, so I was unable to verify whether 17
instructions is the best number possible.
