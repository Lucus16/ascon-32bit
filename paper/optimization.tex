\section{Optimization}

In this section, we will optimize Ascon in several parts. We begin by looking at
the state representation and we continue by considering the possibilities for
each phase of the Ascon permutation. We find efficient implementations for them
and consider whether further improvement is possible. In the end, we look at the
entire round and the entire permutation, which loops over several rounds.

\subsection{Endianness}

The first optimization we can make is due to endianness. Ascon uses big endian
state words, while most processors, including the HiFive1, support little endian
memory access only. This means the endianness of the plaintext and the
associated data needs to be reversed before it can be merged into the state.

The reference implementation switches endianness while loading and storing the
Ascon state to memory every time new data needs to be merged into it. It does
this by storing individual bytes, which means 40 load and store operations are
needed, while just 10 would be sufficient if the endianness had matched.

To improve on this number, it's possible to keep the endianness of the state the
same at all times and instead switch the endianness of the data that needs to be
merged into it. For Ascon128, this means only 64 bits of data needs their
endianness reversed instead of 320 bits, and for Ascon128a, only 128 bits of
data needs their endianness reversed. In addition, the data that needs to be
merged into the state only needs to be loaded, it doesn't need to be stored in
the same endianness again.

\subsection{Substitution layer}

The substitution layer in Ascon is a 5-bit S-box which is applied in parallel to
64 sets of 5 bits. This allows an efficient implementation using
\emph{bit slicing}. Bit slicing is a technique for computing many lookup tables
in parallel and in constant time by computing the lookup table using binary
operations. Each of these binary operations take two bits as input and produce
one bit as output. All common architectures have bitwise operation instructions
that run such binary operations in parallel on entire machine words.

Ascon was optimized for bit slicing: The 5 bits of the S-box are located in 5
different state words in the same bit position. The Ascon paper describes an
instruction sequence that uses bit slicing to compute the S-box efficiently.
The instruction sequence translates to 22 single-cycle instructions.
Because these can only be applied to 32 bits at a time on 32-bit platforms and
the five state words are 64 bits in size each, they need to be run twice.
Therefore, a straightforward implementation takes 44 cycles. The instruction
sequence is shown in Figure~\ref{originalsbox}.

\begin{figure}
\begin{verbatim}
    x0 ^=  x4;  x4 ^=  x3;  x2 ^=  x1;
    t0  = ~x0;  t1  = ~x1;  t2  = ~x2;  t3  = ~x3;  t4  = ~x4;
    t0 &=  x1;  t1 &=  x2;  t2 &=  x3;  t3 &=  x4;  t4 &=  x0;
    x0 ^=  t1;  x1 ^=  t2;  x2 ^=  t3;  x3 ^=  t4;  x4 ^=  t0;
    x1 ^=  x0;  x0 ^=  x4;  x3 ^=  x2;  x2  = ~x2;
\end{verbatim}

\caption{This instruction sequence is the parallel implementation the S-box of
Ascon was designed for. It consists of 22 operations.}

\label{originalsbox}
\end{figure}

The substitution layer can be optimized by computing the same S-box with a
different formula. I derived shorter binary formulas by first writing down the
the bit sequences that occur for each of the 5 output bits for all 32 inputs.
This allowed me to recognize patterns in the bit sequences. I first eliminated
input bits that did not affect the output, or only affected the output through a
single exclusive or operation. Then, I looked at the remaining bit patterns and
found short and overlapping binary formulas for them.

Figure~\ref{shortformulas} shows the result of this analysis. These formulas are
based on three bitwise additions which are used three times each. Three
operations are used to compute these bitwise additions and 14 more operations
are then used to compute all formulas. This results in a total of 17 operations,
five fewer than the reference implementation.

\begin{figure}
\begin{align*}
   o_0 & = i_3 \xor i_4 \xor (i_1 \vee (i_0 \xor i_2 \xor i_4))
\\ o_1 & = i_0 \xor i_4 \xor ((i_1 \xor i_2) \vee (i_2 \xor i_3))
\\ o_2 & = i_1 \xor i_2 \xor (i_3 \vee \neg i_4)
\\ o_3 & = i_1 \xor i_2 \xor (i_0 \vee (i_3 \xor i_4))
\\ o_4 & = i_3 \xor i_4 \xor (i_1 \wedge \neg (i_0 \xor i_4))
\end{align*}

\caption{These formulas compute the Ascon S-box in 17 operations (once duplicate
operations are taken out), while the reference implementation uses 22
operations. $o_n$ indicates output bit $n$ and $i_n$ indicates input bit $n$.}

\label{shortformulas}
\end{figure}

From these, an instruction sequence can be produced like the one in
Figure~\ref{substitution}. It computes all five of the above formulas with one
caveat: The results end up in different registers. We will compensate
for this in the linear diffusion layer.

\begin{figure}[p]
\lstinputlisting[style=customasm,firstline=9,lastline=27]{../riscv/permutation.s}

\caption{This macro computes boolean formulas of the substitution layer in
parallel on the registers \texttt{s0} through \texttt{s4}, using \texttt{t0},
\texttt{t1} and \texttt{t2} as temporary registers. This macro must be run
twice with different state registers, because each 64-bit state word is split in
two 32-bit machine words. Note that after this macro, the state words end up
in different registers. In order to make space to move the state words back into
position, the final result is stored in register \texttt{r0} instead.}

\label{substitution}
\end{figure}

\subsubsection{Superoptimization}

As these formulas were constructed by hand, it may be possible to do better.
There are many possible formulas, so it is infeasable to find the best option by
hand. One option is to use something like the GNU
Superoptimizer~\cite{superoptimizer} which tries all possible instruction
sequence of a certain length in order to see if any of them computes a specific
formula. Unfortunately, according to its README, the longest instruction
sequence it was able to find for anything was seven instructions long. This is
not enough, since it is expected that at least five bitwise addition operations
and at least five bitwise and operations need to be computed, leading to a
minimum of 10 instructions.

Stoffelen~\cite{sat} attempted to optimize binary formulas for S-boxes of
various cryptographic primitives using a SAT solver. He found ways to compile an
S-box to a satisfiability problem determining whether it can be computed in a
given number of instructions. I made use of his project to prove that it is not
possible to compute the Ascon S-box in 10 instructions. Unfortunately, the
project did not finish within reasonable time for instruction counts larger than
10, so I was unable to verify whether 17 instructions is the best number
possible.

\subsection{Linear diffusion layer}

The linear diffusion layer in Ascon mixes the bits with each of the five state
words. Each of them is rotated by two different amounts and added with the
results. For each of the five state words, two different rotation amounts are
used.

The state words are 64 bits in size, so 64-bit rotations are used. On most
64-bit architectures, this can be done with just one instruction, but 32-bit
architectures generally don't have 64-bit rotate instructions.

The most straightforward method of simulating a 64-bit rotate instruction is
using shifts. By combining a left shift and a right shift, all bits of a single
machine word can be placed at the correct offset. Because each shift loses some
of the bits when they are shifted out, it is not possible to simulate a rotate
instruction with one shift. Because a state word is two machine words in size,
four shifts are needed in total. After this, there are four intermediate results
that need to be combined to two resulting machine words. There are several
options to combine these intermediate results:

Since each bit will be zero in at least one of the two operands, the combining
instructions can be bitwise or, bitwise exclusive or, or integer addition
instructions. Since the results will be combined using bitwise exclusive or
operations, we will use these for combining intermediate shift results as well.
Because bitwise exclusive or operations are associative, it does not matter in
what order all intermediate results are combined, which gives us more freedom
during implementation. Bitwise exclusive or can also be seen as bitwise
addition, or addition without any carries between the bits. I will use
\emph{bitwise addition} to refer to this operation from here on.

Because two shifts are needed to simulate one rotate instruction, it would be
preferable to use the latter. Unfortunately, rotating both halves of a state
word independently does not have the same effect as rotating the word as a
whole.

This is usually resolved using bit interleaving: By storing even-numbered bits
in one word and odd-numbered bits in another, a 64-bit rotation can be simulated
using two 32-bit rotations. For even rotation amounts, both words are rotated by
half the amount. For odd rotation amounts, the words are rotated by half the
amount, rounded up and down respectively, and then swapped. The disadvantage is
that some extra cycles are needed to convert between bit-interleaved
representation and normal representation.

Unfortunately, the microcontroller we are targeting does not have rotate
instructions at all. The SiFive HiFive1 does not include the bit-manipulation
extension, which is where RISC-V introduces its rotate instructions. This means
that we are stuck using the straightforward method using four shifts and two
bitwise additions. The result also needs to be bitwise added back into the
original, which takes another two instructions, one for each of the machine
words. This means that handling one rotation takes eight instructions in total.

There are five state words which each have two rotations applied to them. This
means that a total of 80 instructions are needed for the linear diffusion layer. On
32-bit RISC-V, without the bit manipulation extension, it is not possible to do
better. This is easy to see:

The only method of moving bits to different offsets is the shift instruction,
and because some bits are lost when they are shifted out, there are always two
needed to handle 32 bits, or one machine word. Therefore, to move all bits into
two separate new positions for five state words, or ten machine word, a total of
40 shifts are needed. Each of these shifts produces an intermediate result and
all those intermediate results need to be merged back into the unrotated state
words.

Each merging instruction takes two operands and produces one result, reducing
the number of intermediate results by one, so a total of 40 merging instructions
is needed to merge all intermediate results. On the HiFive1, there are no
instructions that can merge more than two machine words at a time. This means
that it is not possible to do the linear diffusion layer in less than 80
instructions. Figure~\ref{xorror} shows how the rotations for a single state
word are implemented in RISC-V assembly.

\begin{figure}[p]
\lstinputlisting[style=customasm,firstline=39,lastline=56]{../riscv/permutation.s}

\caption{The \texttt{xorror} macro applies the rotates and bitwise additions of
the linear diffusion layer to a single state word. Because of the limitations of
assembly macros, it takes many arguments:
\texttt{dl} and \texttt{dh} are the registers for storing the low and high part
of the result. \texttt{sl}, \texttt{sh}, \texttt{sl0}, \texttt{sh0},
\texttt{sl1} and \texttt{sh1} are the source registers for the state word,
without rotation, with rotation \texttt{r0} and with rotation \texttt{r1}
respectively. Because 32-bit shift instructions can only shift by 31 at most,
representing rotate amounts above 32 is done by subtracting 32 and swapping the
respective source registers. This is why the source registers must be supplied
three times.}

\label{xorror}
\end{figure}

\subsection{Addition of round constant}

The addition of the round constant is a very short phase: A single byte gets
bitwise added to one of the state words. This takes one instruction. The correct
value to be added still needs to be computed though. We note that the round
constant for any given round is always \hex{f} lower than in the previous round.
This means we can compute it based on the previous round with one subtraction.

\subsection{Full round}

The round constant for the initial round is based only on the number of rounds
that will be run in that invocation of the permutation. Therefor, it can be a
parameter to the permutation which is constant in every location where it is
called. Because the constant is just one byte in size, the initial constant can
be created with just one move immediate instruction.

The initial constant has been chosen in such a way that the constant in the
final round is always \hex{4b}. This means the current round constant can be
used to check if we've reached the final round. This saves us the instructions
necessary to keep track of the round number.

As mentioned before, the substitution layer shuffles the state words into
different registers and this needs to be resolved during the linear diffusion
layer. The first step is to store one state word in temporary registers. That
frees up the two registers corresponding to another state word. We compute the
linear diffusion layer for the state word that should end up in those registers,
thereby freeing another pair of registers, the ones this new state word was
stored in.

We repeat this five times, and at the end, we use the state word that was stored
in the temporary registers. Figure~\ref{round} shows the inner loop for the
entire round. The \texttt{xorror} macro's first two arguments are the registers
where the result will be stored and the next two arguments are the registers
that are freed up.

\begin{figure}[p]
\lstinputlisting[style=customasm,firstline=104,lastline=123]{../riscv/permutation.s}

\caption{The main permutation loop that loops over all rounds in a permutation,
defined in terms of the \texttt{sbox} and \texttt{xorror} macros.
\texttt{t5} contains the round constant, \texttt{a0} through \texttt{a4}
contain the least significant halves of the five state words, \texttt{s0}
through \texttt{s4} contain the most significant halves of the five state words.
\texttt{t0} through \texttt{t4} are used as temporary registers and \texttt{a5}
contains the round constant where the loop should end, \hex{3c}.}

\label{round}
\end{figure}
