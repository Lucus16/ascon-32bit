\documentclass[17pt]{beamer}

\usepackage{amsmath}
\usepackage{amssymb}
\usepackage[page]{appendix}
\usepackage{caption}
\usepackage[scaled=0.8]{DejaVuSansMono}
\usepackage{listings}
\usepackage{sfmath}
\usepackage{textcomp}
\usepackage{tikz}
\usepackage{url}
\usepackage{xcolor}

\beamertemplatenavigationsymbolsempty
\setbeamertemplate{frametitle}[default][center]

\newcommand{\hex}[1]{$\mathtt{#1}_h$}
\newcommand{\xor}[0]{\oplus}
\newcommand{\cat}[0]{\parallel}
\newcommand{\ror}[0]{\ggg}

\newcommand{\red}[1]{\textcolor[rgb]{0.9,0.0,0.0}{#1}}
\newcommand{\blu}[1]{\textcolor[rgb]{0.0,0.0,0.9}{#1}}
\newcommand{\grn}[1]{\textcolor[rgb]{0.0,0.6,0.0}{#1}}

\definecolor{comment}{gray}{0.4}

\lstdefinestyle{customasm}{,
    language={[x86masm]Assembler},
    belowcaptionskip=1\baselineskip,
    basicstyle=\ttfamily,
    columns=fullflexible,
    keepspaces=true,
    upquote=true,
    numbers=none,
    numberstyle=\ttfamily\color{comment},
    keywords={.endm,.macro,.text,.globl},
    keywordstyle=\bfseries,
    morecomment=[f][\color{comment}]{\#},
}

\title{Optimizing Ascon on RISC-V}
\author{Lars Jellema}
\date{\today}

\begin{document}

\maketitle

\begin{frame}
    \frametitle{Ascon}
    \begin{itemize}
        \item Authenticated Encryption with Associated Data
        \item Small internal state
        \item Inner loop: permutation
    \end{itemize}
\end{frame}

\begin{frame}
    \frametitle{RISC-V}
    \begin{itemize}
        \item Instruction Set Architecture
        \item Open
        \item RV32IMAC
    \end{itemize}
\end{frame}

\begin{frame}
    \frametitle{Mode of Operation}
    Initialization and Associated Data
    \begin{center}
        \input{enca}
    \end{center}
\end{frame}

\begin{frame}
    \frametitle{Mode of Operation}
    Encryption and Finalization
    \begin{center}
        \input{encb}
    \end{center}
\end{frame}

\begin{frame}
    \frametitle{Mode of Operation}
    Decryption and Finalization
    \begin{center}
        \input{decb}
    \end{center}
\end{frame}

\begin{frame}
    \frametitle{Endianness}
    \begin{center}
    \begin{tabular}{l l}
        Ascon permutation & BE \\
        Plaintext loading & LE \\
    \end{tabular}
    \begin{tabular}{l l}
        \\\\
        Convert on permutation & 100\% \\
        Convert on load & 116\% \\
    \end{tabular}
    \end{center}
\end{frame}

\begin{frame}
    \frametitle{Questions?}
\end{frame}

\begin{frame}
    \frametitle{Permutation}
    \begin{itemize}
        \item Substitution layer
        \item Linear diffusion layer
        \item Addition of round constant
        \item Loop counter
    \end{itemize}
\end{frame}

\begin{frame}[t]
    \frametitle{Substitution Layer}
    22 operations
    \vfill
    \begin{center}
        \input{sbox}
    \end{center}
\end{frame}

\begin{frame}[t]
    \frametitle{Substitution Layer}
    23 operations
    \begin{align*}
        o_0 & = i_3 \xor i_4 \xor (i_1 \vee (i_0 \xor i_4 \xor i_2)) \\
        o_1 & = i_0 \xor i_4 \xor ((i_1 \xor i_2) \vee (i_2 \xor i_3)) \\
        o_2 & = i_1 \xor i_2 \xor (i_3 \vee \neg i_4) \\
        o_3 & = i_1 \xor i_2 \xor (i_0 \vee (i_3 \xor i_4)) \\
        o_4 & = i_3 \xor i_4 \xor (i_1 \wedge \neg (i_0 \xor i_4)) \\
    \end{align*}
\end{frame}

\begin{frame}[t]
    \frametitle{Substitution Layer}
    23 operations
    \begin{align*}
        o_0 & = \red{i_3 \xor i_4} \xor (i_1 \vee (\blu{i_0 \xor i_4} \xor i_2)) \\
        o_1 & = \blu{i_0 \xor i_4} \xor ((\grn{i_1 \xor i_2}) \vee (i_2 \xor i_3)) \\
        o_2 & = \grn{i_1 \xor i_2} \xor (i_3 \vee \neg i_4) \\
        o_3 & = \grn{i_1 \xor i_2} \xor (i_0 \vee (\red{i_3 \xor i_4})) \\
        o_4 & = \red{i_3 \xor i_4} \xor (i_1 \wedge \neg (\blu{i_0 \xor i_4})) \\
    \end{align*}
\end{frame}

\newcommand{\tde}[0]{\red{t_{34}}}
\newcommand{\tae}[0]{\blu{t_{04}}}
\newcommand{\tbc}[0]{\grn{t_{12}}}

\begin{frame}[t]
    \frametitle{Substitution Layer}
    17 operations
    \begin{align*}
        o_0 & = \tde \xor (i_1 \vee (\tae \xor i_2)) \\
        o_1 & = \tae \xor (\tbc \vee (i_2 \xor i_3)) \\
        o_2 & = \tbc \xor (i_3 \vee \neg i_4) \\
        o_3 & = \tbc \xor (i_0 \vee \tde) \\
        o_4 & = \tde \xor (i_1 \wedge \neg \tae) \\
        \tae & = i_0 \xor i_4 \quad
        \tbc = i_1 \xor i_2 \quad
        \tde = i_3 \xor i_4 \\
    \end{align*}
\end{frame}

\begin{frame}
    \frametitle{Bitslicing}
    \begin{itemize}
        \item Need to run 64 S-boxes
        \item Bitwise instructions run 32
        \item 34 operations total
    \end{itemize}
\end{frame}

\begin{frame}
    \frametitle{Questions?}
\end{frame}

\begin{frame}[t]
    \frametitle{Minimizing Register Usage}
    Operation order
    \begin{align*}
        r_4 & = o_2 = t_{12} \xor (r_3 \vee \neg r_4) \\
        r_3 & = o_1 = t_{04} \xor (t_{12} \vee (r_2 \xor r_3)) \\
        r_2 & = o_0 = t_{34} \xor (r_1 \vee (r_2 \xor t_{04})) \\
        r_1 & = o_4 = t_{34} \xor (r_1 \wedge \neg t_{04}) \\
        \red{t_0} & = o_3 = t_{12} \xor (r_0 \vee t_{34}) \\
    \end{align*}
\end{frame}

\begin{frame}
    \frametitle{Minimizing Register Usage}
    \begin{itemize}
        \item Used 13 registers instead of 15
        \item But: registers got shuffled
    \end{itemize}
\end{frame}

\begin{frame}
    \frametitle{Optimizing S-boxes}
    \begin{itemize}
        \item Find minimum operation count
        \item Use a SAT solver
    \end{itemize}
\end{frame}

\begin{frame}
    \frametitle{Optimizing S-boxes}
    \begin{center}
        \begin{tabular}{c c}
            Operations & Satisfiable \\ \hline
            10 & N \\
            11 & \\
            \vdots \\
            16 & \\
            17 & Y \\
        \end{tabular}
    \end{center}
\end{frame}

\begin{frame}
    \frametitle{Questions?}
\end{frame}

\begin{frame}[t]
    \frametitle{Linear Diffusion Layer}
    \begin{alignat*}{3}
        o_0 & = i_0 \xor (i_0 \ror \: & 19) & \xor (i_0 \ror \: & 28) \\
        o_1 & = i_1 \xor (i_1 \ror \: & 61) & \xor (i_1 \ror \: & 39) \\
        o_2 & = i_2 \xor (i_2 \ror \: &  1) & \xor (i_2 \ror \: &  6) \\
        o_3 & = i_3 \xor (i_3 \ror \: & 10) & \xor (i_3 \ror \: & 17) \\
        o_4 & = i_4 \xor (i_4 \ror \: &  7) & \xor (i_4 \ror \: & 41) \\
    \end{alignat*}
\end{frame}

\begin{frame}
    \frametitle{32-bit Rotate and Add}
    \begin{enumerate}
        \item Shift left
        \item Shift right
        \item Combine
        \item Add
    \end{enumerate}
\end{frame}

\begin{frame}
    \frametitle{Linear Diffusion Layer}
    \begin{itemize}
        \item 2 operations for $\ror$ and $\xor$
        \item $\times 2$ to emulate with shifts
        \item $\times 2$ for 64-bit
        \item $\times 2$ per word
        \item $\times 5$ for five words
        \item 80 operations total
    \end{itemize}
\end{frame}

\begin{frame}
    \frametitle{Loop and Round Constant}
    \begin{itemize}
        \item Final round constant is \hex{4b}
        \item Decreases by \hex{f} each round
        \item Round constant = loop counter
    \end{itemize}
\end{frame}

\begin{frame}
    \frametitle{Loop and Round Constant}
    \begin{enumerate}
        \item Add round constant
        \item Decrease
        \item Compare and loop if not done
    \end{enumerate}
    \vfill
    \begin{itemize}
        \item 3 operations total
    \end{itemize}
\end{frame}

\begin{frame}
    \frametitle{Questions?}
\end{frame}

\begin{frame}[t]
    \frametitle{Fixing Shuffled Registers}
    Operation order
    \begin{alignat*}{3}
        r_0 & = r_2 \xor (r_2 \ror \: & 19) & \xor (r_2 \ror \: & 28) \\
        r_2 & = r_4 \xor (r_4 \ror \: &  1) & \xor (r_4 \ror \: &  6) \\
        r_4 & = r_1 \xor (r_1 \ror \: &  7) & \xor (r_1 \ror \: & 41) \\
        r_1 & = r_3 \xor (r_3 \ror \: & 61) & \xor (r_3 \ror \: & 39) \\
        r_3 & = \red{t_0} \xor (\red{t_0} \ror \: & 10) & \xor (\red{t_0} \ror \: & 17) \\
    \end{alignat*}
\end{frame}

\begin{frame}
    \frametitle{Permutation Cycle Counts}
    \begin{center}
        \begin{tabular}{l c c}
            Substitution layer & 34 & 34 \\
            Linear diffusion layer & 80 & 80 \\
            Loop and round constant & 3 & 4 \\ \hline
            Total & 117 & 118
        \end{tabular}
    \end{center}
\end{frame}

\begin{frame}
    \frametitle{Register Usage}
    \begin{center}
        \begin{tabular}{l c}
            Ascon state & 10 \\
            Temporary & 3 \\
            Shuffle registers & 2 \\
            Current round constant & 1 \\
            Final round constant & 1 \\ \hline
            Total & 17
        \end{tabular}
    \end{center}
\end{frame}

\begin{frame}
    \frametitle{Relative Performance}
    \begin{center}
        \begin{tabular}{l c c c}
            Reference implementation & 100\% \\
            Big endian state & 116\% \\
            Inner loop in assembly & 129\% \\
        \end{tabular}
    \end{center}
\end{frame}

\begin{frame}
    \frametitle{Questions?}
\end{frame}

\end{document}
