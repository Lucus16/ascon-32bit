\documentclass[17pt]{beamer}

\usepackage{amsmath}
\usepackage{amssymb}
\usepackage[page]{appendix}
\usepackage{caption}
\usepackage[scaled=0.8]{DejaVuSansMono}
\usepackage{listings}
\usepackage{sfmath}
\usepackage{textcomp}
\usepackage{tikz}
\usepackage{url}
\usepackage{xcolor}

\beamertemplatenavigationsymbolsempty
\setbeamertemplate{frametitle}[default][center]

\newcommand{\hex}[1]{$\mathtt{#1}_h$}
\newcommand{\xor}[0]{\oplus}
\newcommand{\cat}[0]{\parallel}
\newcommand{\ror}[0]{\ggg}

\newcommand{\red}[1]{\textcolor[rgb]{0.9,0.0,0.0}{#1}}
\newcommand{\blu}[1]{\textcolor[rgb]{0.0,0.0,0.9}{#1}}
\newcommand{\grn}[1]{\textcolor[rgb]{0.0,0.6,0.0}{#1}}

\definecolor{comment}{gray}{0.4}

\lstdefinestyle{customasm}{,
    language={[x86masm]Assembler},
    belowcaptionskip=1\baselineskip,
    basicstyle=\ttfamily,
    columns=fullflexible,
    keepspaces=true,
    upquote=true,
    numbers=none,
    numberstyle=\ttfamily\color{comment},
    keywords={.endm,.macro,.text,.globl},
    keywordstyle=\bfseries,
    morecomment=[f][\color{comment}]{\#},
}

\title{Optimizing Ascon on RISC-V}
\author{Lars Jellema}
\date{\today}

\begin{document}

\maketitle

\begin{frame}
    \frametitle{Ascon}
    \begin{itemize}
        \item Authenticated Encryption with Associated Data
        \item Small internal state
        \item Inner loop: permutation
    \end{itemize}
\end{frame}

\begin{frame}
    \frametitle{RISC-V}
    \begin{itemize}
        \item Instruction Set Architecture
        \item Open
        \item RV32IMAC
    \end{itemize}
\end{frame}

\begin{frame}
    \frametitle{Mode of Operation}
    Initialization and Associated Data
    \begin{center}
        %%%%%%%%%%%%%%%%%%%%%%%%%%%%%%%%%%%%%%%%%%%%%%%%%%%%%%%%%%%%%%%%%%%%%%%%%%%%%%%%%%
% The Ascon encryption mode
%
% public domain (CC0 1.0 https://creativecommons.org/publicdomain/zero/1.0/)
%%%%%%%%%%%%%%%%%%%%%%%%%%%%%%%%%%%%%%%%%%%%%%%%%%%%%%%%%%%%%%%%%%%%%%%%%%%%%%%%%%

\newif\ifsans
\newif\iftext
\newif\ifdetails

%%% CONFIGURATION %%%%%%%%%%%%%%%%%%%%%%%%%%%%%%%%%%%%%%%%%%%%%%%%%%%%%%%%%%%%%%%%
% preferably use build.py

%\sanstrue  % for sans-serif fonts (slides, web)
\sansfalse  % for serif fonts (article)

%\texttrue  % include phase description
\textfalse  % no phase description

\detailsfalse % simplify (eg, 'const' instead of 'kab0'
%%%%%%%%%%%%%%%%%%%%%%%%%%%%%%%%%%%%%%%%%%%%%%%%%%%%%%%%%%%%%%%%%%%%%%%%%%%%%%%%%%

\ifsans
\renewcommand*\familydefault{\sfdefault}
\usepackage{sfmath}
\fi

\tikzset{sparsam/.style={inner sep=1pt}}
\tikzset{bitwidth/.style={above=-1pt, font=\small}}
\tikzset{next/.style={->, >=latex}}

\begin{tikzpicture}
  \newcommand{\conc}{\ensuremath{\Vert}}
  \newcommand{\perm}[1]{node[rectangle, rounded corners=3pt, minimum width=.5cm, minimum height=1.8cm, draw, sparsam] {$p^{#1}$}}
  \ifsans
    \newcommand{\tikzxor}[1]{node[circle, inner sep=-1.3pt, name={#1}] {\tikz{\draw[] (0,0) circle (3.75pt) +(3.75pt,0) -- +(-3.75pt,0) +(0,3.75pt) -- +(0,-3.75pt);}}}
  \else
    \newcommand{\tikzxor}[1]{node[circle, inner sep=-1.3pt, name={#1}] {$\oplus$}}
  \fi
  \newcommand{\bitwidth}{\tikz{\draw[-] (-2pt,-2pt) -- (2pt, 2pt);}}
  \newcommand{\rate}{.5cm}
  \newcommand{\msg}{.6cm}
  \newcommand{\phase}{1.7cm}
  \newcommand{\minnext}{.4cm}

  \clip(-3,-2) rectangle (6,2);

  % --- init up to p^a ---
  \begin{scope}[xshift=0cm]
    \ifdetails
      \draw (0,\rate) node[left, sparsam] {$k \conc r \conc a \conc b \conc 0^*$};
      \draw (0,-\rate) node[left, sparsam] {$K \conc N$};

      \draw[next] (0,\rate) -- node {\bitwidth} node[bitwidth] {$r$} (.7,\rate);
      \draw[next] (0,-\rate) -- node {\bitwidth} node[bitwidth] {$c$} (.7,-\rate);
    \else
      \draw (0,0) node[left, sparsam] {$\mathrm{IV} \conc K \conc N$};
      \draw[next] (0,0) -- node {\bitwidth} node[bitwidth] {$320$} (.7,0);
    \fi

    \draw (.95,0) \perm{a};
  \end{scope}

  % --- init after p^a and auth A1 ---
  \begin{scope}[xshift=1.2cm]
    \draw (.4,-\rate) \tikzxor{Ki};
    \draw[next] (0,-\rate) -- (Ki);
    \draw[next] (Ki) +(0,-\msg) node[below] {$0^* \conc K$ \hspace*{.3cm}} -- (Ki);
    \draw[next] (Ki) -- node[pos=0.6] {\bitwidth} node [pos=0.6, bitwidth] {$c$} +(1.3,0);

    \draw[dashdotted] (.8,1.5) -- (.8,-1.5);

    \draw (1.3,\rate) \tikzxor{A1};
    \draw[next] (0,\rate) -- node[near start] {\bitwidth} node[near start, bitwidth] {$r$} (A1);
    \draw[next] (A1) +(0,\msg) node[above] {$A_1$} -- (A1);
    \draw[next] (A1) -- +(\minnext,0);

    \draw (1.95,0) \perm{b};
  \end{scope}

  % --- auth As ---
  \begin{scope}[xshift=3.4cm]
    \draw[dotted] (0,\rate) -- (\minnext,\rate)
                  (0,-\rate) -- (\minnext,-\rate);

    \draw (.9,\rate) \tikzxor{As};
    \draw[next] (\minnext,\rate) -- (As);
    \draw[next] (As) +(0,\msg) node[above] {$A_s$} -- (As);
    \draw[next] (As) -- +(\minnext,0);

    \draw[next] (\minnext,-\rate) -- node {\bitwidth} node[bitwidth] {$c$} (1.3,-\rate);

    \draw (1.55,0) \perm{b};
  \end{scope}

  % --- enc P1 ---
  \begin{scope}[xshift=5.2cm]
    \draw (.4,-\rate) \tikzxor{AuthPad};
    \draw[next] (0,-\rate) -- (AuthPad);
    \draw[next] (AuthPad) +(0,-\msg) node[below] {$0^* \conc 1$ \hspace*{.3cm}} -- (AuthPad);
    \draw[next] (AuthPad) -- node {\bitwidth} node [bitwidth] {$c$} +(1.7,0);

    \draw[dashdotted] (.8,1.5) -- (.8,-1.5);

    \draw (1.3,\rate) \tikzxor{P1};
    \draw[next] (0,\rate) -- node[near start] {\bitwidth} node[near start, bitwidth] {$r$} (P1);
    \draw[next] (P1) +(0,\msg) node[above] {$P_1$} -- (P1);
    \draw[next] (P1) -- +(2*\minnext,0);
    \draw[next] (P1) ++(.5,0) -- +(0,\msg) node[above] {$C_1$};

    \draw (2.35,0) \perm{b};
  \end{scope}

  % --- enc Pt-1 ---
  \begin{scope}[xshift=7.8cm]
    \draw[dotted] (0,\rate) -- (\minnext,\rate)
                  (0,-\rate) -- (\minnext,-\rate);

    \draw[next] (\minnext,-\rate) -- node[pos=.4] {\bitwidth} node [pos=.4,bitwidth] {$c$} +(1.3,0);

    \draw (.9,\rate) \tikzxor{Pt1};
    \draw[next] (\minnext,\rate) -- (Pt1);
    \draw[next] (Pt1) +(0,\msg) node[above] {$P_{t\!-\!1}$ \hspace*{.15cm}} -- (Pt1);
    \draw[next] (Pt1) -- +(.8,0);
    \draw[next] (Pt1) ++(.5,0) -- +(0,\msg) node[above] {\hspace*{.2cm} $C_{t\!-\!1}$};

    \draw (1.95,0) \perm{b};
  \end{scope}

  % --- enc Pt and finalize ---
  \begin{scope}[xshift=10.0cm]
    \draw (.5,\rate) \tikzxor{Pt};
    \draw[next] (0,\rate) -- (Pt);
    \draw[next] (Pt) +(0,\msg) node[above] {$P_t$} -- (Pt);
    \draw[next] (Pt) ++(.5,0) -- +(0,\msg) node[above] {$C_t$};
    \draw[next] (Pt) -- node[pos=.7] {\bitwidth} node[pos=.7,bitwidth] {$r$} +(1.6,0);

    \draw[dashdotted] (1.3,1.5) -- (1.3,-1.5);

    \draw (1.7,-\rate) \tikzxor{Kf};
    \draw[next] (Kf) +(0,-\msg) node[below] {\hspace*{.3cm} $K \conc 0^*$} -- (Kf);
    \draw[next] (0,-\rate) -- node[pos=.3] {\bitwidth} node[pos=.3,bitwidth] {$c$} (Kf);
    \draw[next] (Kf) -- (2.1,-\rate);

    \draw (2.35,0) \perm{a};

    \draw (3.3,-\rate) \tikzxor{Kt};
    \draw[next] (Kt) +(0,-\msg) node[below] {$K$} -- (Kt);
    \draw[next] (2.6,-\rate) -- node[pos=.4] {\bitwidth} node[pos=.4,bitwidth] {$k$} (Kt);
    \draw (4.0,-\rate) node[name=T,sparsam] {$T$};
    \draw[next] (Kt) -- (T);
  \end{scope}

  % --- phase descriptions ---
  \iftext
    \draw (.5,-\rate-\phase) node {Initialization};
    \draw (4.0,-\rate-\phase) node {Associated Data};
    \draw (8.5,-\rate-\phase) node {Plaintext};
    \draw (12.8,-\rate-\phase) node {Finalization};
  \fi
\end{tikzpicture}

%\begin{tikzpicture}[scale=0.6, every node/.style={scale=0.6}]
%  \newcommand{\conc}{\ensuremath{\Vert}}
%  \newcommand{\perm}[1]{node[rectangle, rounded corners=3pt, minimum width=.5cm, minimum height=1.8cm, draw, sparsam] {$p^{#1}$}}
%  \ifsans
%    \newcommand{\tikzxor}[1]{node[circle, inner sep=-1.3pt, name={#1}] {\tikz{\draw[] (0,0) circle (3.75pt) +(3.75pt,0) -- +(-3.75pt,0) +(0,3.75pt) -- +(0,-3.75pt);}}}
%  \else
%    \newcommand{\tikzxor}[1]{node[circle, inner sep=-1.3pt, name={#1}] {$\oplus$}}
%  \fi
%  \newcommand{\bitwidth}{\tikz{\draw[-] (-2pt,-2pt) -- (2pt, 2pt);}}
%  \newcommand{\rate}{.5cm}
%  \newcommand{\msg}{.6cm}
%  \newcommand{\phase}{1.7cm}
%  \newcommand{\minnext}{.4cm}
%
%  % --- init up to p^a ---
%  \begin{scope}[xshift=0cm]
%    \ifdetails
%      \draw (0,\rate) node[left, sparsam] {$k \conc r \conc a \conc b \conc 0^*$};
%      \draw (0,-\rate) node[left, sparsam] {$K \conc N$};
%
%      \draw[next] (0,\rate) -- node {\bitwidth} node[bitwidth] {$r$} (.7,\rate);
%      \draw[next] (0,-\rate) -- node {\bitwidth} node[bitwidth] {$c$} (.7,-\rate);
%    \else
%      \draw (0,0) node[left, sparsam] {$\mathrm{IV} \conc K \conc N$};
%      \draw[next] (0,0) -- node {\bitwidth} node[bitwidth] {$320$} (.7,0);
%    \fi
%
%    \draw (.95,0) \perm{a};
%  \end{scope}
%
%  % --- init after p^a and auth A1 ---
%  \begin{scope}[xshift=1.2cm]
%    \draw (.4,-\rate) \tikzxor{Ki};
%    \draw[next] (0,-\rate) -- (Ki);
%    \draw[next] (Ki) +(0,-\msg) node[below] {$0^* \conc K$ \hspace*{.3cm}} -- (Ki);
%    \draw[next] (Ki) -- node[pos=0.6] {\bitwidth} node [pos=0.6, bitwidth] {$c$} +(1.3,0);
%
%    \draw[dashdotted] (.8,1.5) -- (.8,-1.5);
%
%    \draw (1.3,\rate) \tikzxor{A1};
%    \draw[next] (0,\rate) -- node[near start] {\bitwidth} node[near start, bitwidth] {$r$} (A1);
%    \draw[next] (A1) +(0,\msg) node[above] {$A_1$} -- (A1);
%    \draw[next] (A1) -- +(\minnext,0);
%
%    \draw (1.95,0) \perm{b};
%  \end{scope}
%
%  % --- auth As ---
%  \begin{scope}[xshift=3.4cm]
%    \draw[dotted] (0,\rate) -- (\minnext,\rate)
%                  (0,-\rate) -- (\minnext,-\rate);
%
%    \draw (.9,\rate) \tikzxor{As};
%    \draw[next] (\minnext,\rate) -- (As);
%    \draw[next] (As) +(0,\msg) node[above] {$A_s$} -- (As);
%    \draw[next] (As) -- +(\minnext,0);
%
%    \draw[next] (\minnext,-\rate) -- node {\bitwidth} node[bitwidth] {$c$} (1.3,-\rate);
%
%    \draw (1.55,0) \perm{b};
%  \end{scope}
%
%  % --- enc P1 ---
%  \begin{scope}[xshift=5.2cm]
%    \draw (.4,-\rate) \tikzxor{AuthPad};
%    \draw[next] (0,-\rate) -- (AuthPad);
%    \draw[next] (AuthPad) +(0,-\msg) node[below] {$0^* \conc 1$ \hspace*{.3cm}} -- (AuthPad);
%    \draw[next] (AuthPad) -- node {\bitwidth} node [bitwidth] {$c$} +(1.7,0);
%
%    \draw[dashdotted] (.8,1.5) -- (.8,-1.5);
%
%    \draw (1.3,\rate) \tikzxor{P1};
%    \draw[next] (0,\rate) -- node[near start] {\bitwidth} node[near start, bitwidth] {$r$} (P1);
%    \draw[next] (P1) -- +(0,\msg) node[above] {$P_1$};
%    \draw[<->,>=latex] (P1) -- +(2*\minnext,0);
%    \draw[next] (P1) ++(.5,\msg) node[above] {$C_1$} -- +(0,-\msg);
%
%    \draw (2.35,0) \perm{b};
%  \end{scope}
%
%  % --- enc Pt-1 ---
%  \begin{scope}[xshift=7.8cm]
%    \draw[dotted] (0,\rate) -- (\minnext,\rate)
%                  (0,-\rate) -- (\minnext,-\rate);
%
%    \draw[next] (\minnext,-\rate) -- node[pos=.4] {\bitwidth} node [pos=.4,bitwidth] {$c$} +(1.3,0);
%
%    \draw (.9,\rate) \tikzxor{Pt1};
%    \draw[next] (\minnext,\rate) -- (Pt1);
%    \draw[next] (Pt1) -- +(0,\msg) node[above] {$P_{t\!-\!1}$ \hspace*{.15cm}};
%    \draw[<->,>=latex] (Pt1) -- +(.8,0);
%    \draw[next] (Pt1) ++(.5,\msg) node[above] {\hspace*{.2cm} $C_{t\!-\!1}$} -- +(0,-\msg);
%
%    \draw (1.95,0) \perm{b};
%  \end{scope}
%
%  % --- enc Pt and finalize ---
%  \begin{scope}[xshift=10.0cm]
%    \draw (.5,\rate) \tikzxor{Pt};
%    \draw[next] (0,\rate) -- (Pt);
%    \draw[next] (Pt) -- +(0,\msg) node[above] {$P_t$};
%    \draw[next] (Pt) ++(.5,\msg) node[above] {$C_t$} -- +(0,-\msg);
%    \draw[<->,>=latex] (Pt) -- node[pos=.7] {\bitwidth} node[pos=.7,bitwidth] {$r$} +(1.6,0);
%
%    \draw[dashdotted] (1.3,1.5) -- (1.3,-1.5);
%
%    \draw (1.7,-\rate) \tikzxor{Kf};
%    \draw[next] (Kf) +(0,-\msg) node[below] {\hspace*{.3cm} $K \conc 0^*$} -- (Kf);
%    \draw[next] (0,-\rate) -- node[pos=.3] {\bitwidth} node[pos=.3,bitwidth] {$c$} (Kf);
%    \draw[next] (Kf) -- (2.1,-\rate);
%
%    \draw (2.35,0) \perm{a};
%
%    \draw (3.3,-\rate) \tikzxor{Kt};
%    \draw[next] (Kt) +(0,-\msg) node[below] {$K$} -- (Kt);
%    \draw[next] (2.6,-\rate) -- node[pos=.4] {\bitwidth} node[pos=.4,bitwidth] {$k$} (Kt);
%    \draw (4.0,-\rate) node[name=T,sparsam] {$T$};
%    \draw[next] (Kt) -- (T);
%  \end{scope}
%
%  % --- phase descriptions ---
%  \iftext
%    \draw (.5,-\rate-\phase) node {Initialization};
%    \draw (4.0,-\rate-\phase) node {Associated Data};
%    \draw (8.5,-\rate-\phase) node {Plaintext};
%    \draw (12.8,-\rate-\phase) node {Finalization};
%  \fi
%\end{tikzpicture}

    \end{center}
\end{frame}

\begin{frame}
    \frametitle{Mode of Operation}
    Encryption and Finalization
    \begin{center}
        %%%%%%%%%%%%%%%%%%%%%%%%%%%%%%%%%%%%%%%%%%%%%%%%%%%%%%%%%%%%%%%%%%%%%%%%%%%%%%%%%%
% The Ascon encryption mode
%
% public domain (CC0 1.0 https://creativecommons.org/publicdomain/zero/1.0/)
%%%%%%%%%%%%%%%%%%%%%%%%%%%%%%%%%%%%%%%%%%%%%%%%%%%%%%%%%%%%%%%%%%%%%%%%%%%%%%%%%%

\newif\ifsans
\newif\iftext
\newif\ifdetails

%%% CONFIGURATION %%%%%%%%%%%%%%%%%%%%%%%%%%%%%%%%%%%%%%%%%%%%%%%%%%%%%%%%%%%%%%%%
% preferably use build.py

%\sanstrue  % for sans-serif fonts (slides, web)
\sansfalse  % for serif fonts (article)

%\texttrue  % include phase description
\textfalse  % no phase description

\detailsfalse % simplify (eg, 'const' instead of 'kab0'
%%%%%%%%%%%%%%%%%%%%%%%%%%%%%%%%%%%%%%%%%%%%%%%%%%%%%%%%%%%%%%%%%%%%%%%%%%%%%%%%%%

\ifsans
\renewcommand*\familydefault{\sfdefault}
\usepackage{sfmath}
\fi

\tikzset{sparsam/.style={inner sep=1pt}}
\tikzset{bitwidth/.style={above=-1pt, font=\small}}
\tikzset{next/.style={->, >=latex}}

\begin{tikzpicture}
  \newcommand{\conc}{\ensuremath{\Vert}}
  \newcommand{\perm}[1]{node[rectangle, rounded corners=3pt, minimum width=.5cm, minimum height=1.8cm, draw, sparsam] {$p^{#1}$}}
  \ifsans
    \newcommand{\tikzxor}[1]{node[circle, inner sep=-1.3pt, name={#1}] {\tikz{\draw[] (0,0) circle (3.75pt) +(3.75pt,0) -- +(-3.75pt,0) +(0,3.75pt) -- +(0,-3.75pt);}}}
  \else
    \newcommand{\tikzxor}[1]{node[circle, inner sep=-1.3pt, name={#1}] {$\oplus$}}
  \fi
  \newcommand{\bitwidth}{\tikz{\draw[-] (-2pt,-2pt) -- (2pt, 2pt);}}
  \newcommand{\rate}{.5cm}
  \newcommand{\msg}{.6cm}
  \newcommand{\phase}{1.7cm}
  \newcommand{\minnext}{.4cm}

  \clip(6,-2) rectangle (15,2);

  % --- init up to p^a ---
  \begin{scope}[xshift=0cm]
    \ifdetails
      \draw (0,\rate) node[left, sparsam] {$k \conc r \conc a \conc b \conc 0^*$};
      \draw (0,-\rate) node[left, sparsam] {$K \conc N$};

      \draw[next] (0,\rate) -- node {\bitwidth} node[bitwidth] {$r$} (.7,\rate);
      \draw[next] (0,-\rate) -- node {\bitwidth} node[bitwidth] {$c$} (.7,-\rate);
    \else
      \draw (0,0) node[left, sparsam] {$\mathrm{IV} \conc K \conc N$};
      \draw[next] (0,0) -- node {\bitwidth} node[bitwidth] {$320$} (.7,0);
    \fi

    \draw (.95,0) \perm{a};
  \end{scope}

  % --- init after p^a and auth A1 ---
  \begin{scope}[xshift=1.2cm]
    \draw (.4,-\rate) \tikzxor{Ki};
    \draw[next] (0,-\rate) -- (Ki);
    \draw[next] (Ki) +(0,-\msg) node[below] {$0^* \conc K$ \hspace*{.3cm}} -- (Ki);
    \draw[next] (Ki) -- node[pos=0.6] {\bitwidth} node [pos=0.6, bitwidth] {$c$} +(1.3,0);

    \draw[dashdotted] (.8,1.5) -- (.8,-1.5);

    \draw (1.3,\rate) \tikzxor{A1};
    \draw[next] (0,\rate) -- node[near start] {\bitwidth} node[near start, bitwidth] {$r$} (A1);
    \draw[next] (A1) +(0,\msg) node[above] {$A_1$} -- (A1);
    \draw[next] (A1) -- +(\minnext,0);

    \draw (1.95,0) \perm{b};
  \end{scope}

  % --- auth As ---
  \begin{scope}[xshift=3.4cm]
    \draw[dotted] (0,\rate) -- (\minnext,\rate)
                  (0,-\rate) -- (\minnext,-\rate);

    \draw (.9,\rate) \tikzxor{As};
    \draw[next] (\minnext,\rate) -- (As);
    \draw[next] (As) +(0,\msg) node[above] {$A_s$} -- (As);
    \draw[next] (As) -- +(\minnext,0);

    \draw[next] (\minnext,-\rate) -- node {\bitwidth} node[bitwidth] {$c$} (1.3,-\rate);

    \draw (1.55,0) \perm{b};
  \end{scope}

  % --- enc P1 ---
  \begin{scope}[xshift=5.2cm]
    \draw (.4,-\rate) \tikzxor{AuthPad};
    \draw[next] (0,-\rate) -- (AuthPad);
    \draw[next] (AuthPad) +(0,-\msg) node[below] {$0^* \conc 1$ \hspace*{.3cm}} -- (AuthPad);
    \draw[next] (AuthPad) -- node {\bitwidth} node [bitwidth] {$c$} +(1.7,0);

    \draw[dashdotted] (.8,1.5) -- (.8,-1.5);

    \draw (1.3,\rate) \tikzxor{P1};
    \draw[next] (0,\rate) -- node[near start] {\bitwidth} node[near start, bitwidth] {$r$} (P1);
    \draw[next] (P1) +(0,\msg) node[above] {$P_1$} -- (P1);
    \draw[next] (P1) -- +(2*\minnext,0);
    \draw[next] (P1) ++(.5,0) -- +(0,\msg) node[above] {$C_1$};

    \draw (2.35,0) \perm{b};
  \end{scope}

  % --- enc Pt-1 ---
  \begin{scope}[xshift=7.8cm]
    \draw[dotted] (0,\rate) -- (\minnext,\rate)
                  (0,-\rate) -- (\minnext,-\rate);

    \draw[next] (\minnext,-\rate) -- node[pos=.4] {\bitwidth} node [pos=.4,bitwidth] {$c$} +(1.3,0);

    \draw (.9,\rate) \tikzxor{Pt1};
    \draw[next] (\minnext,\rate) -- (Pt1);
    \draw[next] (Pt1) +(0,\msg) node[above] {$P_{t\!-\!1}$ \hspace*{.15cm}} -- (Pt1);
    \draw[next] (Pt1) -- +(.8,0);
    \draw[next] (Pt1) ++(.5,0) -- +(0,\msg) node[above] {\hspace*{.2cm} $C_{t\!-\!1}$};

    \draw (1.95,0) \perm{b};
  \end{scope}

  % --- enc Pt and finalize ---
  \begin{scope}[xshift=10.0cm]
    \draw (.5,\rate) \tikzxor{Pt};
    \draw[next] (0,\rate) -- (Pt);
    \draw[next] (Pt) +(0,\msg) node[above] {$P_t$} -- (Pt);
    \draw[next] (Pt) ++(.5,0) -- +(0,\msg) node[above] {$C_t$};
    \draw[next] (Pt) -- node[pos=.7] {\bitwidth} node[pos=.7,bitwidth] {$r$} +(1.6,0);

    \draw[dashdotted] (1.3,1.5) -- (1.3,-1.5);

    \draw (1.7,-\rate) \tikzxor{Kf};
    \draw[next] (Kf) +(0,-\msg) node[below] {\hspace*{.3cm} $K \conc 0^*$} -- (Kf);
    \draw[next] (0,-\rate) -- node[pos=.3] {\bitwidth} node[pos=.3,bitwidth] {$c$} (Kf);
    \draw[next] (Kf) -- (2.1,-\rate);

    \draw (2.35,0) \perm{a};

    \draw (3.3,-\rate) \tikzxor{Kt};
    \draw[next] (Kt) +(0,-\msg) node[below] {$K$} -- (Kt);
    \draw[next] (2.6,-\rate) -- node[pos=.4] {\bitwidth} node[pos=.4,bitwidth] {$k$} (Kt);
    \draw (4.0,-\rate) node[name=T,sparsam] {$T$};
    \draw[next] (Kt) -- (T);
  \end{scope}

  % --- phase descriptions ---
  \iftext
    \draw (.5,-\rate-\phase) node {Initialization};
    \draw (4.0,-\rate-\phase) node {Associated Data};
    \draw (8.5,-\rate-\phase) node {Plaintext};
    \draw (12.8,-\rate-\phase) node {Finalization};
  \fi
\end{tikzpicture}

%\begin{tikzpicture}[scale=0.6, every node/.style={scale=0.6}]
%  \newcommand{\conc}{\ensuremath{\Vert}}
%  \newcommand{\perm}[1]{node[rectangle, rounded corners=3pt, minimum width=.5cm, minimum height=1.8cm, draw, sparsam] {$p^{#1}$}}
%  \ifsans
%    \newcommand{\tikzxor}[1]{node[circle, inner sep=-1.3pt, name={#1}] {\tikz{\draw[] (0,0) circle (3.75pt) +(3.75pt,0) -- +(-3.75pt,0) +(0,3.75pt) -- +(0,-3.75pt);}}}
%  \else
%    \newcommand{\tikzxor}[1]{node[circle, inner sep=-1.3pt, name={#1}] {$\oplus$}}
%  \fi
%  \newcommand{\bitwidth}{\tikz{\draw[-] (-2pt,-2pt) -- (2pt, 2pt);}}
%  \newcommand{\rate}{.5cm}
%  \newcommand{\msg}{.6cm}
%  \newcommand{\phase}{1.7cm}
%  \newcommand{\minnext}{.4cm}
%
%  % --- init up to p^a ---
%  \begin{scope}[xshift=0cm]
%    \ifdetails
%      \draw (0,\rate) node[left, sparsam] {$k \conc r \conc a \conc b \conc 0^*$};
%      \draw (0,-\rate) node[left, sparsam] {$K \conc N$};
%
%      \draw[next] (0,\rate) -- node {\bitwidth} node[bitwidth] {$r$} (.7,\rate);
%      \draw[next] (0,-\rate) -- node {\bitwidth} node[bitwidth] {$c$} (.7,-\rate);
%    \else
%      \draw (0,0) node[left, sparsam] {$\mathrm{IV} \conc K \conc N$};
%      \draw[next] (0,0) -- node {\bitwidth} node[bitwidth] {$320$} (.7,0);
%    \fi
%
%    \draw (.95,0) \perm{a};
%  \end{scope}
%
%  % --- init after p^a and auth A1 ---
%  \begin{scope}[xshift=1.2cm]
%    \draw (.4,-\rate) \tikzxor{Ki};
%    \draw[next] (0,-\rate) -- (Ki);
%    \draw[next] (Ki) +(0,-\msg) node[below] {$0^* \conc K$ \hspace*{.3cm}} -- (Ki);
%    \draw[next] (Ki) -- node[pos=0.6] {\bitwidth} node [pos=0.6, bitwidth] {$c$} +(1.3,0);
%
%    \draw[dashdotted] (.8,1.5) -- (.8,-1.5);
%
%    \draw (1.3,\rate) \tikzxor{A1};
%    \draw[next] (0,\rate) -- node[near start] {\bitwidth} node[near start, bitwidth] {$r$} (A1);
%    \draw[next] (A1) +(0,\msg) node[above] {$A_1$} -- (A1);
%    \draw[next] (A1) -- +(\minnext,0);
%
%    \draw (1.95,0) \perm{b};
%  \end{scope}
%
%  % --- auth As ---
%  \begin{scope}[xshift=3.4cm]
%    \draw[dotted] (0,\rate) -- (\minnext,\rate)
%                  (0,-\rate) -- (\minnext,-\rate);
%
%    \draw (.9,\rate) \tikzxor{As};
%    \draw[next] (\minnext,\rate) -- (As);
%    \draw[next] (As) +(0,\msg) node[above] {$A_s$} -- (As);
%    \draw[next] (As) -- +(\minnext,0);
%
%    \draw[next] (\minnext,-\rate) -- node {\bitwidth} node[bitwidth] {$c$} (1.3,-\rate);
%
%    \draw (1.55,0) \perm{b};
%  \end{scope}
%
%  % --- enc P1 ---
%  \begin{scope}[xshift=5.2cm]
%    \draw (.4,-\rate) \tikzxor{AuthPad};
%    \draw[next] (0,-\rate) -- (AuthPad);
%    \draw[next] (AuthPad) +(0,-\msg) node[below] {$0^* \conc 1$ \hspace*{.3cm}} -- (AuthPad);
%    \draw[next] (AuthPad) -- node {\bitwidth} node [bitwidth] {$c$} +(1.7,0);
%
%    \draw[dashdotted] (.8,1.5) -- (.8,-1.5);
%
%    \draw (1.3,\rate) \tikzxor{P1};
%    \draw[next] (0,\rate) -- node[near start] {\bitwidth} node[near start, bitwidth] {$r$} (P1);
%    \draw[next] (P1) -- +(0,\msg) node[above] {$P_1$};
%    \draw[<->,>=latex] (P1) -- +(2*\minnext,0);
%    \draw[next] (P1) ++(.5,\msg) node[above] {$C_1$} -- +(0,-\msg);
%
%    \draw (2.35,0) \perm{b};
%  \end{scope}
%
%  % --- enc Pt-1 ---
%  \begin{scope}[xshift=7.8cm]
%    \draw[dotted] (0,\rate) -- (\minnext,\rate)
%                  (0,-\rate) -- (\minnext,-\rate);
%
%    \draw[next] (\minnext,-\rate) -- node[pos=.4] {\bitwidth} node [pos=.4,bitwidth] {$c$} +(1.3,0);
%
%    \draw (.9,\rate) \tikzxor{Pt1};
%    \draw[next] (\minnext,\rate) -- (Pt1);
%    \draw[next] (Pt1) -- +(0,\msg) node[above] {$P_{t\!-\!1}$ \hspace*{.15cm}};
%    \draw[<->,>=latex] (Pt1) -- +(.8,0);
%    \draw[next] (Pt1) ++(.5,\msg) node[above] {\hspace*{.2cm} $C_{t\!-\!1}$} -- +(0,-\msg);
%
%    \draw (1.95,0) \perm{b};
%  \end{scope}
%
%  % --- enc Pt and finalize ---
%  \begin{scope}[xshift=10.0cm]
%    \draw (.5,\rate) \tikzxor{Pt};
%    \draw[next] (0,\rate) -- (Pt);
%    \draw[next] (Pt) -- +(0,\msg) node[above] {$P_t$};
%    \draw[next] (Pt) ++(.5,\msg) node[above] {$C_t$} -- +(0,-\msg);
%    \draw[<->,>=latex] (Pt) -- node[pos=.7] {\bitwidth} node[pos=.7,bitwidth] {$r$} +(1.6,0);
%
%    \draw[dashdotted] (1.3,1.5) -- (1.3,-1.5);
%
%    \draw (1.7,-\rate) \tikzxor{Kf};
%    \draw[next] (Kf) +(0,-\msg) node[below] {\hspace*{.3cm} $K \conc 0^*$} -- (Kf);
%    \draw[next] (0,-\rate) -- node[pos=.3] {\bitwidth} node[pos=.3,bitwidth] {$c$} (Kf);
%    \draw[next] (Kf) -- (2.1,-\rate);
%
%    \draw (2.35,0) \perm{a};
%
%    \draw (3.3,-\rate) \tikzxor{Kt};
%    \draw[next] (Kt) +(0,-\msg) node[below] {$K$} -- (Kt);
%    \draw[next] (2.6,-\rate) -- node[pos=.4] {\bitwidth} node[pos=.4,bitwidth] {$k$} (Kt);
%    \draw (4.0,-\rate) node[name=T,sparsam] {$T$};
%    \draw[next] (Kt) -- (T);
%  \end{scope}
%
%  % --- phase descriptions ---
%  \iftext
%    \draw (.5,-\rate-\phase) node {Initialization};
%    \draw (4.0,-\rate-\phase) node {Associated Data};
%    \draw (8.5,-\rate-\phase) node {Plaintext};
%    \draw (12.8,-\rate-\phase) node {Finalization};
%  \fi
%\end{tikzpicture}

    \end{center}
\end{frame}

\begin{frame}
    \frametitle{Mode of Operation}
    Decryption and Finalization
    \begin{center}
        %%%%%%%%%%%%%%%%%%%%%%%%%%%%%%%%%%%%%%%%%%%%%%%%%%%%%%%%%%%%%%%%%%%%%%%%%%%%%%%%%%
% The Ascon encryption mode
%
% public domain (CC0 1.0 https://creativecommons.org/publicdomain/zero/1.0/)
%%%%%%%%%%%%%%%%%%%%%%%%%%%%%%%%%%%%%%%%%%%%%%%%%%%%%%%%%%%%%%%%%%%%%%%%%%%%%%%%%%

\newif\ifsans
\newif\iftext
\newif\ifdetails

%%% CONFIGURATION %%%%%%%%%%%%%%%%%%%%%%%%%%%%%%%%%%%%%%%%%%%%%%%%%%%%%%%%%%%%%%%%
% preferably use build.py

%\sanstrue  % for sans-serif fonts (slides, web)
\sansfalse  % for serif fonts (article)

%\texttrue  % include phase description
\textfalse  % no phase description

\detailsfalse % simplify (eg, 'const' instead of 'kab0'
%%%%%%%%%%%%%%%%%%%%%%%%%%%%%%%%%%%%%%%%%%%%%%%%%%%%%%%%%%%%%%%%%%%%%%%%%%%%%%%%%%

\ifsans
\renewcommand*\familydefault{\sfdefault}
\usepackage{sfmath}
\fi

\tikzset{sparsam/.style={inner sep=1pt}}
\tikzset{bitwidth/.style={above=-1pt, font=\small}}
\tikzset{next/.style={->, >=latex}}

%\begin{tikzpicture}
%  \newcommand{\conc}{\ensuremath{\Vert}}
%  \newcommand{\perm}[1]{node[rectangle, rounded corners=3pt, minimum width=.5cm, minimum height=1.8cm, draw, sparsam] {$p^{#1}$}}
%  \ifsans
%    \newcommand{\tikzxor}[1]{node[circle, inner sep=-1.3pt, name={#1}] {\tikz{\draw[] (0,0) circle (3.75pt) +(3.75pt,0) -- +(-3.75pt,0) +(0,3.75pt) -- +(0,-3.75pt);}}}
%  \else
%    \newcommand{\tikzxor}[1]{node[circle, inner sep=-1.3pt, name={#1}] {$\oplus$}}
%  \fi
%  \newcommand{\bitwidth}{\tikz{\draw[-] (-2pt,-2pt) -- (2pt, 2pt);}}
%  \newcommand{\rate}{.5cm}
%  \newcommand{\msg}{.6cm}
%  \newcommand{\phase}{1.7cm}
%  \newcommand{\minnext}{.4cm}
%
%  \clip(6,-2) rectangle (15,2);
%
%  % --- init up to p^a ---
%  \begin{scope}[xshift=0cm]
%    \ifdetails
%      \draw (0,\rate) node[left, sparsam] {$k \conc r \conc a \conc b \conc 0^*$};
%      \draw (0,-\rate) node[left, sparsam] {$K \conc N$};
%
%      \draw[next] (0,\rate) -- node {\bitwidth} node[bitwidth] {$r$} (.7,\rate);
%      \draw[next] (0,-\rate) -- node {\bitwidth} node[bitwidth] {$c$} (.7,-\rate);
%    \else
%      \draw (0,0) node[left, sparsam] {$\mathrm{IV} \conc K \conc N$};
%      \draw[next] (0,0) -- node {\bitwidth} node[bitwidth] {$320$} (.7,0);
%    \fi
%
%    \draw (.95,0) \perm{a};
%  \end{scope}
%
%  % --- init after p^a and auth A1 ---
%  \begin{scope}[xshift=1.2cm]
%    \draw (.4,-\rate) \tikzxor{Ki};
%    \draw[next] (0,-\rate) -- (Ki);
%    \draw[next] (Ki) +(0,-\msg) node[below] {$0^* \conc K$ \hspace*{.3cm}} -- (Ki);
%    \draw[next] (Ki) -- node[pos=0.6] {\bitwidth} node [pos=0.6, bitwidth] {$c$} +(1.3,0);
%
%    \draw[dashdotted] (.8,1.5) -- (.8,-1.5);
%
%    \draw (1.3,\rate) \tikzxor{A1};
%    \draw[next] (0,\rate) -- node[near start] {\bitwidth} node[near start, bitwidth] {$r$} (A1);
%    \draw[next] (A1) +(0,\msg) node[above] {$A_1$} -- (A1);
%    \draw[next] (A1) -- +(\minnext,0);
%
%    \draw (1.95,0) \perm{b};
%  \end{scope}
%
%  % --- auth As ---
%  \begin{scope}[xshift=3.4cm]
%    \draw[dotted] (0,\rate) -- (\minnext,\rate)
%                  (0,-\rate) -- (\minnext,-\rate);
%
%    \draw (.9,\rate) \tikzxor{As};
%    \draw[next] (\minnext,\rate) -- (As);
%    \draw[next] (As) +(0,\msg) node[above] {$A_s$} -- (As);
%    \draw[next] (As) -- +(\minnext,0);
%
%    \draw[next] (\minnext,-\rate) -- node {\bitwidth} node[bitwidth] {$c$} (1.3,-\rate);
%
%    \draw (1.55,0) \perm{b};
%  \end{scope}
%
%  % --- enc P1 ---
%  \begin{scope}[xshift=5.2cm]
%    \draw (.4,-\rate) \tikzxor{AuthPad};
%    \draw[next] (0,-\rate) -- (AuthPad);
%    \draw[next] (AuthPad) +(0,-\msg) node[below] {$0^* \conc 1$ \hspace*{.3cm}} -- (AuthPad);
%    \draw[next] (AuthPad) -- node {\bitwidth} node [bitwidth] {$c$} +(1.7,0);
%
%    \draw[dashdotted] (.8,1.5) -- (.8,-1.5);
%
%    \draw (1.3,\rate) \tikzxor{P1};
%    \draw[next] (0,\rate) -- node[near start] {\bitwidth} node[near start, bitwidth] {$r$} (P1);
%    \draw[next] (P1) +(0,\msg) node[above] {$P_1$} -- (P1);
%    \draw[next] (P1) -- +(2*\minnext,0);
%    \draw[next] (P1) ++(.5,0) -- +(0,\msg) node[above] {$C_1$};
%
%    \draw (2.35,0) \perm{b};
%  \end{scope}
%
%  % --- enc Pt-1 ---
%  \begin{scope}[xshift=7.8cm]
%    \draw[dotted] (0,\rate) -- (\minnext,\rate)
%                  (0,-\rate) -- (\minnext,-\rate);
%
%    \draw[next] (\minnext,-\rate) -- node[pos=.4] {\bitwidth} node [pos=.4,bitwidth] {$c$} +(1.3,0);
%
%    \draw (.9,\rate) \tikzxor{Pt1};
%    \draw[next] (\minnext,\rate) -- (Pt1);
%    \draw[next] (Pt1) +(0,\msg) node[above] {$P_{t\!-\!1}$ \hspace*{.15cm}} -- (Pt1);
%    \draw[next] (Pt1) -- +(.8,0);
%    \draw[next] (Pt1) ++(.5,0) -- +(0,\msg) node[above] {\hspace*{.2cm} $C_{t\!-\!1}$};
%
%    \draw (1.95,0) \perm{b};
%  \end{scope}
%
%  % --- enc Pt and finalize ---
%  \begin{scope}[xshift=10.0cm]
%    \draw (.5,\rate) \tikzxor{Pt};
%    \draw[next] (0,\rate) -- (Pt);
%    \draw[next] (Pt) +(0,\msg) node[above] {$P_t$} -- (Pt);
%    \draw[next] (Pt) ++(.5,0) -- +(0,\msg) node[above] {$C_t$};
%    \draw[next] (Pt) -- node[pos=.7] {\bitwidth} node[pos=.7,bitwidth] {$r$} +(1.6,0);
%
%    \draw[dashdotted] (1.3,1.5) -- (1.3,-1.5);
%
%    \draw (1.7,-\rate) \tikzxor{Kf};
%    \draw[next] (Kf) +(0,-\msg) node[below] {\hspace*{.3cm} $K \conc 0^*$} -- (Kf);
%    \draw[next] (0,-\rate) -- node[pos=.3] {\bitwidth} node[pos=.3,bitwidth] {$c$} (Kf);
%    \draw[next] (Kf) -- (2.1,-\rate);
%
%    \draw (2.35,0) \perm{a};
%
%    \draw (3.3,-\rate) \tikzxor{Kt};
%    \draw[next] (Kt) +(0,-\msg) node[below] {$K$} -- (Kt);
%    \draw[next] (2.6,-\rate) -- node[pos=.4] {\bitwidth} node[pos=.4,bitwidth] {$k$} (Kt);
%    \draw (4.0,-\rate) node[name=T,sparsam] {$T$};
%    \draw[next] (Kt) -- (T);
%  \end{scope}
%
%  % --- phase descriptions ---
%  \iftext
%    \draw (.5,-\rate-\phase) node {Initialization};
%    \draw (4.0,-\rate-\phase) node {Associated Data};
%    \draw (8.5,-\rate-\phase) node {Plaintext};
%    \draw (12.8,-\rate-\phase) node {Finalization};
%  \fi
%\end{tikzpicture}

\begin{tikzpicture}
  \newcommand{\conc}{\ensuremath{\Vert}}
  \newcommand{\perm}[1]{node[rectangle, rounded corners=3pt, minimum width=.5cm, minimum height=1.8cm, draw, sparsam] {$p^{#1}$}}
  \ifsans
    \newcommand{\tikzxor}[1]{node[circle, inner sep=-1.3pt, name={#1}] {\tikz{\draw[] (0,0) circle (3.75pt) +(3.75pt,0) -- +(-3.75pt,0) +(0,3.75pt) -- +(0,-3.75pt);}}}
  \else
    \newcommand{\tikzxor}[1]{node[circle, inner sep=-1.3pt, name={#1}] {$\oplus$}}
  \fi
  \newcommand{\bitwidth}{\tikz{\draw[-] (-2pt,-2pt) -- (2pt, 2pt);}}
  \newcommand{\rate}{.5cm}
  \newcommand{\msg}{.6cm}
  \newcommand{\phase}{1.7cm}
  \newcommand{\minnext}{.4cm}

  \clip(6,-2) rectangle (15,2);

  % --- init up to p^a ---
  \begin{scope}[xshift=0cm]
    \ifdetails
      \draw (0,\rate) node[left, sparsam] {$k \conc r \conc a \conc b \conc 0^*$};
      \draw (0,-\rate) node[left, sparsam] {$K \conc N$};

      \draw[next] (0,\rate) -- node {\bitwidth} node[bitwidth] {$r$} (.7,\rate);
      \draw[next] (0,-\rate) -- node {\bitwidth} node[bitwidth] {$c$} (.7,-\rate);
    \else
      \draw (0,0) node[left, sparsam] {$\mathrm{IV} \conc K \conc N$};
      \draw[next] (0,0) -- node {\bitwidth} node[bitwidth] {$320$} (.7,0);
    \fi

    \draw (.95,0) \perm{a};
  \end{scope}

  % --- init after p^a and auth A1 ---
  \begin{scope}[xshift=1.2cm]
    \draw (.4,-\rate) \tikzxor{Ki};
    \draw[next] (0,-\rate) -- (Ki);
    \draw[next] (Ki) +(0,-\msg) node[below] {$0^* \conc K$ \hspace*{.3cm}} -- (Ki);
    \draw[next] (Ki) -- node[pos=0.6] {\bitwidth} node [pos=0.6, bitwidth] {$c$} +(1.3,0);

    \draw[dashdotted] (.8,1.5) -- (.8,-1.5);

    \draw (1.3,\rate) \tikzxor{A1};
    \draw[next] (0,\rate) -- node[near start] {\bitwidth} node[near start, bitwidth] {$r$} (A1);
    \draw[next] (A1) +(0,\msg) node[above] {$A_1$} -- (A1);
    \draw[next] (A1) -- +(\minnext,0);

    \draw (1.95,0) \perm{b};
  \end{scope}

  % --- auth As ---
  \begin{scope}[xshift=3.4cm]
    \draw[dotted] (0,\rate) -- (\minnext,\rate)
                  (0,-\rate) -- (\minnext,-\rate);

    \draw (.9,\rate) \tikzxor{As};
    \draw[next] (\minnext,\rate) -- (As);
    \draw[next] (As) +(0,\msg) node[above] {$A_s$} -- (As);
    \draw[next] (As) -- +(\minnext,0);

    \draw[next] (\minnext,-\rate) -- node {\bitwidth} node[bitwidth] {$c$} (1.3,-\rate);

    \draw (1.55,0) \perm{b};
  \end{scope}

  % --- enc P1 ---
  \begin{scope}[xshift=5.2cm]
    \draw (.4,-\rate) \tikzxor{AuthPad};
    \draw[next] (0,-\rate) -- (AuthPad);
    \draw[next] (AuthPad) +(0,-\msg) node[below] {$0^* \conc 1$ \hspace*{.3cm}} -- (AuthPad);
    \draw[next] (AuthPad) -- node {\bitwidth} node [bitwidth] {$c$} +(1.7,0);

    \draw[dashdotted] (.8,1.5) -- (.8,-1.5);

    \draw (1.3,\rate) \tikzxor{P1};
    \draw[next] (0,\rate) -- node[near start] {\bitwidth} node[near start, bitwidth] {$r$} (P1);
    \draw[next] (P1) -- +(0,\msg) node[above] {$P_1$};
    \draw[<->,>=latex] (P1) -- +(2*\minnext,0);
    \draw[next] (P1) ++(.5,\msg) node[above] {$C_1$} -- +(0,-\msg);

    \draw (2.35,0) \perm{b};
  \end{scope}

  % --- enc Pt-1 ---
  \begin{scope}[xshift=7.8cm]
    \draw[dotted] (0,\rate) -- (\minnext,\rate)
                  (0,-\rate) -- (\minnext,-\rate);

    \draw[next] (\minnext,-\rate) -- node[pos=.4] {\bitwidth} node [pos=.4,bitwidth] {$c$} +(1.3,0);

    \draw (.9,\rate) \tikzxor{Pt1};
    \draw[next] (\minnext,\rate) -- (Pt1);
    \draw[next] (Pt1) -- +(0,\msg) node[above] {$P_{t\!-\!1}$ \hspace*{.15cm}};
    \draw[<->,>=latex] (Pt1) -- +(.8,0);
    \draw[next] (Pt1) ++(.5,\msg) node[above] {\hspace*{.2cm} $C_{t\!-\!1}$} -- +(0,-\msg);

    \draw (1.95,0) \perm{b};
  \end{scope}

  % --- enc Pt and finalize ---
  \begin{scope}[xshift=10.0cm]
    \draw (.5,\rate) \tikzxor{Pt};
    \draw[next] (0,\rate) -- (Pt);
    \draw[next] (Pt) -- +(0,\msg) node[above] {$P_t$};
    \draw[next] (Pt) ++(.5,\msg) node[above] {$C_t$} -- +(0,-\msg);
    \draw[<->,>=latex] (Pt) -- node[pos=.7] {\bitwidth} node[pos=.7,bitwidth] {$r$} +(1.6,0);

    \draw[dashdotted] (1.3,1.5) -- (1.3,-1.5);

    \draw (1.7,-\rate) \tikzxor{Kf};
    \draw[next] (Kf) +(0,-\msg) node[below] {\hspace*{.3cm} $K \conc 0^*$} -- (Kf);
    \draw[next] (0,-\rate) -- node[pos=.3] {\bitwidth} node[pos=.3,bitwidth] {$c$} (Kf);
    \draw[next] (Kf) -- (2.1,-\rate);

    \draw (2.35,0) \perm{a};

    \draw (3.3,-\rate) \tikzxor{Kt};
    \draw[next] (Kt) +(0,-\msg) node[below] {$K$} -- (Kt);
    \draw[next] (2.6,-\rate) -- node[pos=.4] {\bitwidth} node[pos=.4,bitwidth] {$k$} (Kt);
    \draw (4.0,-\rate) node[name=T,sparsam] {$T$};
    \draw[next] (Kt) -- (T);
  \end{scope}

  % --- phase descriptions ---
  \iftext
    \draw (.5,-\rate-\phase) node {Initialization};
    \draw (4.0,-\rate-\phase) node {Associated Data};
    \draw (8.5,-\rate-\phase) node {Plaintext};
    \draw (12.8,-\rate-\phase) node {Finalization};
  \fi
\end{tikzpicture}

    \end{center}
\end{frame}

\begin{frame}
    \frametitle{Endianness}
    \begin{center}
    \begin{tabular}{l l}
        Ascon permutation & BE \\
        Plaintext loading & LE \\
    \end{tabular}
    \begin{tabular}{l l}
        \\\\
        Convert on permutation & 100\% \\
        Convert on load & 116\% \\
    \end{tabular}
    \end{center}
\end{frame}

\begin{frame}
    \frametitle{Questions?}
\end{frame}

\begin{frame}
    \frametitle{Permutation}
    \begin{itemize}
        \item Substitution layer
        \item Linear diffusion layer
        \item Addition of round constant
        \item Loop counter
    \end{itemize}
\end{frame}

\begin{frame}[t]
    \frametitle{Substitution Layer}
    22 operations
    \vfill
    \begin{center}
        \usetikzlibrary{calc,cipher,sponge}

\newcommand{\advanceS}[1][.5]{\draw (s) + (#1,0) coordinate (s);}
\newcommand{\updateTo}[2]{\draw[next] (#1) -- (#2); \node also [alias=#1] (#2);}

\begin{tikzpicture}[scale=1, rounded corners=2pt,
                    op/.append style={minimum size=1.25ex},
                    tee/.style={}, every node/.append style={inner sep=1pt}]
  % --- inputs ---
  \coordinate (s);
  \foreach \x in {0,...,5} {\draw (s|-0,-\x) coordinate (w\x);}
  \foreach \x in {0,...,4} {\node[left] (x\x) at (w\x) {$i_{\x}$};}

  % --- input xors ---
  \advanceS
  \draw (s|-w0) coordinate[xor] (in0);  \updateTo{w0}{in0}
  \draw (s|-w4) coordinate[tee] (tin0); \draw[next] (tin0) -- (in0);
  \advanceS
  \draw (s|-w2) coordinate[xor] (in2);  \updateTo{w2}{in2}
  \draw (s|-w1) coordinate[tee] (tin2); \draw[next] (tin2) -- (in2);
  \draw (s|-w4) coordinate[xor] (in4);  \updateTo{w4}{in4}
  \draw (s|-w3) coordinate[tee] (tin4); \draw[next] (tin4) -- (in4);

  % --- chi nots ---
  \advanceS
  \draw (s|-w0) coordinate[tee] (tand4);
  \advanceS
  \foreach \x in {0,...,4} {\draw (s|-w\x) +(0,-.5) node[left] (cnot\x) {$1$};}
  \advanceS
  \foreach \x in {0,...,4} {\draw (s|-w\x)    coordinate[tee] (tnot\x);
                            \draw (s|-cnot\x) coordinate[xor] (not\x);
                            \draw[next] (cnot\x) -- (not\x);
                            \draw[next] (tnot\x) -- (not\x);}

  % --- chi ands ---
  \advanceS[.75]
  \foreach \x in {0,...,4} {\draw (s|-cnot\x) coordinate[andalt] (and\x);
                            \draw[next] (not\x) -- (and\x);}
  \foreach \x/\X in {0/1,1/2,2/3,3/4} {\draw (and\x|-w\X) coordinate[tee] (tand\x); 
                                       \draw[next] (tand\x) -- (and\x);}
  \draw[next] (tand4) |- (not4|-w5) -| (and4);

  % --- chi xors ---
  \advanceS[1.25]
  \foreach \x in {0,...,4} {\draw (s|-w\x) coordinate[xor] (xor\x);
                            \updateTo{w\x}{xor\x}}
  \foreach \x/\X in {0/1,1/2,2/3,3/4} {\draw[next] (and\X) -- +(.3,0) -- (xor\x|-and\x) -- (xor\x);}
  \draw[next] (and0) -- +(.3,0) -- ($(and4)!.5!(xor4|-and4)$) -- (xor4|-and4) -- (xor4);

  % --- output xors ---
  \advanceS[.75]
  \draw (s|-w1) coordinate[xor] (out1);  \updateTo{w1}{out1}
  \draw (s|-w0) coordinate[tee] (tout1); \draw[next] (tout1) -- (out1);
  \draw (s|-w3) coordinate[xor] (out3);  \updateTo{w3}{out3}
  \draw (s|-w2) coordinate[tee] (tout3); \draw[next] (tout3) -- (out3);
  \advanceS
  \draw (s|-w0) coordinate[xor] (out0);  \updateTo{w0}{out0}
  \draw (s|-w4) coordinate[tee] (tout0); \draw[next] (tout0) -- (out0);
  \advanceS
  \draw (s|-w2) coordinate[xor] (out2);  \updateTo{w2}{out2}
  \draw (s|-and2) node[below] (cout2) {$1$};
  \draw[next] (cout2) -- (out2);
  %\draw (s|-and2) node[left=.15] (cout2) {$1$};
  %\draw[next] (cout2) -| (out2);

  % --- outputs ---
  \advanceS
  \foreach \x in {0,...,4} {\node[right] (y\x) at (s|-w\x) {$o_{\x}$}; \updateTo{w\x}{y\x}}
\end{tikzpicture}

    \end{center}
\end{frame}

\begin{frame}[t]
    \frametitle{Substitution Layer}
    23 operations
    \begin{align*}
        o_0 & = i_3 \xor i_4 \xor (i_1 \vee (i_0 \xor i_4 \xor i_2)) \\
        o_1 & = i_0 \xor i_4 \xor ((i_1 \xor i_2) \vee (i_2 \xor i_3)) \\
        o_2 & = i_1 \xor i_2 \xor (i_3 \vee \neg i_4) \\
        o_3 & = i_1 \xor i_2 \xor (i_0 \vee (i_3 \xor i_4)) \\
        o_4 & = i_3 \xor i_4 \xor (i_1 \wedge \neg (i_0 \xor i_4)) \\
    \end{align*}
\end{frame}

\begin{frame}[t]
    \frametitle{Substitution Layer}
    23 operations
    \begin{align*}
        o_0 & = \red{i_3 \xor i_4} \xor (i_1 \vee (\blu{i_0 \xor i_4} \xor i_2)) \\
        o_1 & = \blu{i_0 \xor i_4} \xor ((\grn{i_1 \xor i_2}) \vee (i_2 \xor i_3)) \\
        o_2 & = \grn{i_1 \xor i_2} \xor (i_3 \vee \neg i_4) \\
        o_3 & = \grn{i_1 \xor i_2} \xor (i_0 \vee (\red{i_3 \xor i_4})) \\
        o_4 & = \red{i_3 \xor i_4} \xor (i_1 \wedge \neg (\blu{i_0 \xor i_4})) \\
    \end{align*}
\end{frame}

\newcommand{\tde}[0]{\red{t_{34}}}
\newcommand{\tae}[0]{\blu{t_{04}}}
\newcommand{\tbc}[0]{\grn{t_{12}}}

\begin{frame}[t]
    \frametitle{Substitution Layer}
    17 operations
    \begin{align*}
        o_0 & = \tde \xor (i_1 \vee (\tae \xor i_2)) \\
        o_1 & = \tae \xor (\tbc \vee (i_2 \xor i_3)) \\
        o_2 & = \tbc \xor (i_3 \vee \neg i_4) \\
        o_3 & = \tbc \xor (i_0 \vee \tde) \\
        o_4 & = \tde \xor (i_1 \wedge \neg \tae) \\
        \tae & = i_0 \xor i_4 \quad
        \tbc = i_1 \xor i_2 \quad
        \tde = i_3 \xor i_4 \\
    \end{align*}
\end{frame}

\begin{frame}
    \frametitle{Bitslicing}
    \begin{itemize}
        \item Need to run 64 S-boxes
        \item Bitwise instructions run 32
        \item 34 operations total
    \end{itemize}
\end{frame}

\begin{frame}
    \frametitle{Questions?}
\end{frame}

\begin{frame}[t]
    \frametitle{Minimizing Register Usage}
    Operation order
    \begin{align*}
        r_4 & = o_2 = t_{12} \xor (r_3 \vee \neg r_4) \\
        r_3 & = o_1 = t_{04} \xor (t_{12} \vee (r_2 \xor r_3)) \\
        r_2 & = o_0 = t_{34} \xor (r_1 \vee (r_2 \xor t_{04})) \\
        r_1 & = o_4 = t_{34} \xor (r_1 \wedge \neg t_{04}) \\
        \red{t_0} & = o_3 = t_{12} \xor (r_0 \vee t_{34}) \\
    \end{align*}
\end{frame}

\begin{frame}
    \frametitle{Minimizing Register Usage}
    \begin{itemize}
        \item Used 13 registers instead of 15
        \item But: registers got shuffled
    \end{itemize}
\end{frame}

\begin{frame}
    \frametitle{Optimizing S-boxes}
    \begin{itemize}
        \item Find minimum operation count
        \item Use a SAT solver
    \end{itemize}
\end{frame}

\begin{frame}
    \frametitle{Optimizing S-boxes}
    \begin{center}
        \begin{tabular}{c c}
            Operations & Satisfiable \\ \hline
            10 & N \\
            11 & \\
            \vdots \\
            16 & \\
            17 & Y \\
        \end{tabular}
    \end{center}
\end{frame}

\begin{frame}
    \frametitle{Questions?}
\end{frame}

\begin{frame}[t]
    \frametitle{Linear Diffusion Layer}
    \begin{alignat*}{3}
        o_0 & = i_0 \xor (i_0 \ror \: & 19) & \xor (i_0 \ror \: & 28) \\
        o_1 & = i_1 \xor (i_1 \ror \: & 61) & \xor (i_1 \ror \: & 39) \\
        o_2 & = i_2 \xor (i_2 \ror \: &  1) & \xor (i_2 \ror \: &  6) \\
        o_3 & = i_3 \xor (i_3 \ror \: & 10) & \xor (i_3 \ror \: & 17) \\
        o_4 & = i_4 \xor (i_4 \ror \: &  7) & \xor (i_4 \ror \: & 41) \\
    \end{alignat*}
\end{frame}

\begin{frame}
    \frametitle{32-bit Rotate and Add}
    \begin{enumerate}
        \item Shift left
        \item Shift right
        \item Combine
        \item Add
    \end{enumerate}
\end{frame}

\begin{frame}
    \frametitle{Linear Diffusion Layer}
    \begin{itemize}
        \item 2 operations for $\ror$ and $\xor$
        \item $\times 2$ to emulate with shifts
        \item $\times 2$ for 64-bit
        \item $\times 2$ per word
        \item $\times 5$ for five words
        \item 80 operations total
    \end{itemize}
\end{frame}

\begin{frame}
    \frametitle{Loop and Round Constant}
    \begin{itemize}
        \item Final round constant is \hex{4b}
        \item Decreases by \hex{f} each round
        \item Round constant = loop counter
    \end{itemize}
\end{frame}

\begin{frame}
    \frametitle{Loop and Round Constant}
    \begin{enumerate}
        \item Add round constant
        \item Decrease
        \item Compare and loop if not done
    \end{enumerate}
    \vfill
    \begin{itemize}
        \item 3 operations total
    \end{itemize}
\end{frame}

\begin{frame}
    \frametitle{Questions?}
\end{frame}

\begin{frame}[t]
    \frametitle{Fixing Shuffled Registers}
    Operation order
    \begin{alignat*}{3}
        r_0 & = r_2 \xor (r_2 \ror \: & 19) & \xor (r_2 \ror \: & 28) \\
        r_2 & = r_4 \xor (r_4 \ror \: &  1) & \xor (r_4 \ror \: &  6) \\
        r_4 & = r_1 \xor (r_1 \ror \: &  7) & \xor (r_1 \ror \: & 41) \\
        r_1 & = r_3 \xor (r_3 \ror \: & 61) & \xor (r_3 \ror \: & 39) \\
        r_3 & = \red{t_0} \xor (\red{t_0} \ror \: & 10) & \xor (\red{t_0} \ror \: & 17) \\
    \end{alignat*}
\end{frame}

\begin{frame}
    \frametitle{Permutation Cycle Counts}
    \begin{center}
        \begin{tabular}{l c c}
            Substitution layer & 34 & 34 \\
            Linear diffusion layer & 80 & 80 \\
            Loop and round constant & 3 & 4 \\ \hline
            Total & 117 & 118
        \end{tabular}
    \end{center}
\end{frame}

\begin{frame}
    \frametitle{Register Usage}
    \begin{center}
        \begin{tabular}{l c}
            Ascon state & 10 \\
            Temporary & 3 \\
            Shuffle registers & 2 \\
            Current round constant & 1 \\
            Final round constant & 1 \\ \hline
            Total & 17
        \end{tabular}
    \end{center}
\end{frame}

\begin{frame}
    \frametitle{Relative Performance}
    \begin{center}
        \begin{tabular}{l c c c}
            Reference implementation & 100\% \\
            Big endian state & 116\% \\
            Inner loop in assembly & 129\% \\
        \end{tabular}
    \end{center}
\end{frame}

\begin{frame}
    \frametitle{Questions?}
\end{frame}

\end{document}
