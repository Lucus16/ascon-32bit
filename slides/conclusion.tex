\section{Discussion}

We have optimized the Ascon internal loop for the RISC-V RV32IMAC architecture.
While doing this, we have noted that both Ascon and RISC-V are very simple and
have little room for optimization. Nonetheless, we have found improvements in
the boolean formulas used to compute the substitution layer and an oppertunity
to combine the round constant with the loop counter. We have also switched the
representation of the state to big endian in order to reduce the number of
endian conversions necessary. Our improvements have provided a speedup of 29%
relative to the reference implementation.

There is still room for more improvement however. We have omitted to optimize
the Ascon outer loop. Because RISC-V offers plenty of registers, writing the
outer loop in assembly makes it possible to avoid loading and storing the state
to memory altogether. It is also possible that the boolean formulas of the
substitution layer are not yet optimal. Finding and proving the optimal formulas
will require some careful analysis however, as the search-space is quite large.

Finally, it is important to note that our optimizations are quite general and
should be portable to other architectures. Our new boolean formulas do not only
improve speed from 22 to 17 cycles, but also register usage from 5 to 3
temporary registers. This makes it plausible to implement the substitution layer
without needing the stack on 32-bit ARM architectures which often only have 14
registers available for implementation. It is also of interest to improve our
implementation using the RISC-V bit manipulation extension, once it is frozen
and more widely available.
