\section{Introduction}

In recent years, many critical vulnerabilities have been found in various kinds
of software and hardware. The software industry continues to grow and the value
of hacking grows with it. Making software secure is hard and many companies
consider it a waste of money. This has lead to botnets, among other things.
Botnets usually consist of a large number of cheap internet-connected consumer
devices that are easily hacked. Because the profit margins are smaller for cheap
devices, there is often less money available for implementing good security.

One way to make good security cheaper is by providing security primitives with
simple interfaces that nonetheless ensure all desirable properties. An example
of this is \emph{authenticated encryption} systems, which provide all of the
most important security properties for symmetric encryption. By using
authenticated encryption, a developer does not need to introduce authentication
to an encryption system manually, leading to fewer possibilities to make
mistakes.

Authenticated encryption has traditionally been implemented by combining
privacy-only encryption with message authentication codes. This combination
turned out to be hard to make without introducing new vulnerabilities. In
addition, performance suffers because two passes need to be made over the
plaintext, one for encryption and one for authentication. To resolve this,
cryptographers have called for new primitives that integrate all desirable
properties.

This call has taken the shape of several competitions, where a diverse set of
teams each publish a family of new primitives, followed by public analysis of
these new primitives by many other researchers. New primitives are then selected
based on how well they withstood public analysis.

Ascon is one of these new primitives, aimed at lightweight applications. It was
first submitted to the CAESAR competition~\cite{caesar} and has been selected as
the first choice for lightweight applications in the final portfolio. It was
also recently submitted to be considered for standardization in NIST Lightweight
Cryptography~\cite{nistlc} and has been selected as a round~1 candidate.

While secure software is important, the hardware it runs on must also be secure.
The integrated circuit industry has tradionally been a closed ecosystem. Chip
manufacturers keep their implementations secret in order to monopolize the
market. This means that when a vulnerability is discovered in a chip, consumers
are entirely dependent on the manufacturer to fix them. It also makes it hard
for third parties to verify a chips behavior.

RISC-V is a hardware instruction set architecture that intends to solve this
issue by being completely open-source. This means anyone is free to use,
implement, extend and adapt it. As a result, RISC-V has been gaining popularity
with researchers as well as companies. RISC-V is an architecture based on RISC
principles, which makes it a good fit for low-power applications.

So far, there has been little work towards optimizing crypto algorithms for
RISC-V in software. Most efforts have been focused on creating hardware
extensions for RISC-V as this was not possible with closed-ecosystem
architectures. Despite the fact that RISC-V extensions do not have licensing
costs and have reduced development costs, chip manufacturing remains expensive,
so fast software implementations remain important. We optimize the Ascon
authenticated encryption system for the RISC-V instruction set architecture.
